\documentclass{notes}

\title{Test for the Amp\`ere-Faraday equation}
\author{Mathias Hoppe}
\date{2023-02-27}

\begin{document}
    \maketitle

    \noindent
    As noted in [Harvey {\em et al}, Nucl.\ Fusion {\bf 59} (2019) 106046], the
    Amp\`ere-Faraday equation in cylindrical geometry reduces to
    \begin{equation}
        \mu_0\frac{\partial j_\Omega}{\partial t} = \frac{1}{r}\frac{\partial}{\partial r}\left(
            r\frac{\partial E_\parallel}{\partial r}
        \right).
    \end{equation}
    Using Ohm's law $j_\Omega = \sigma E_\parallel$, we can further write this
    equation as
    \begin{equation}
        \mu_0\sigma\frac{\partial E_\parallel}{\partial t} = \frac{1}{r}\frac{\partial}{\partial r}\left(
            r\frac{\partial E_\parallel}{\partial r}
        \right),
    \end{equation}
    which has a well-known solution in terms of Bessel modes:
    \begin{equation}
        E_\parallel(r,t) = \sum_k A_k J_0\left(\lambda_k r\right)
        \exp\left( -\frac{\lambda_k^2 t}{\mu_0\sigma} \right).
    \end{equation}
    Here, $\lambda_k$ is the $k$'th zero of $J_0(x)$ and $A_k$ are coefficients
    such that
    \begin{equation}
        E_\parallel(r,t=0) = \sum_k A_k J_0(\lambda_k r).
    \end{equation}
    Given the time evolution of the parallel electric field $E_\parallel$, and
    assuming that $A_k=0$ except for $k=l$, the conductivity may therefore be
    calculated from
    \begin{equation}
        \sigma = -\frac{\lambda_l^2 t}{\mu_0\log\left(\frac{E_\parallel(t,r)}{A_l J_0(\lambda_lr)}\right)}
    \end{equation}

\end{document}
