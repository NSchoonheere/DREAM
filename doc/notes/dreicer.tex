\documentclass{notes}

\title{Implementation of the Dreicer runaway rate}
\author{Mathias Hoppe}

\usepackage{amssymb}

\newcommand{\Ec}{E_{\rm c}}
\newcommand{\ED}{E_{\rm D}}
\newcommand{\Zeff}{Z_{\rm eff}}

\begin{document}
    \maketitle

    \noindent
    This document describes the details of the Connor-Hastie Dreicer runaway
    rate was implemented \DREAM.

    \section*{Connor-Hastie runaway rate}
    The expression derived by Connor \& Hastie for the Dreicer runaway rate is
    \begin{equation}\label{eq:gammaCH}
        \gamma = C\frac{n_e}{\tau_{ee}} \left( \frac{E}{\ED} \right)^{-\frac{3}{16}(1+\Zeff)h}
        \exp\left[-\lambda\frac{\ED}{4E} - \sqrt{\eta\frac{(1+\Zeff)\ED}{E}} \right],
    \end{equation}
    where $C$ is an undetermined constant (which we will set to 1\footnote{
    According to Ola, this is the value obtained from simulations.}), $n_e$ is
    the electron density, $\tau_{ee}$ is the thermal electron-electron collision
    time, $\Zeff$ is the plasma effective charge. The coefficients $\lambda$,
    $\eta$ and $h$ all approach unity when $E\gg E_{\rm c}$, and the option
    \texttt{DREICER\_RATE\_CONNOR\_HASTIE\_NOCORR} sets $\lambda=\eta=h=1$.
    In general, these coefficients are given by
    \begin{equation}
        \label{eq:gammaCH_var}
        \begin{aligned}
            \lambda &= 8\frac{E^2}{\Ec^2}\left[ 1-\frac{\Ec}{2E}-\sqrt{1-\frac{\Ec}{E}} \right],\\
            \eta &= \frac{1}{4}\frac{E^2/\Ec}{E/\Ec-1}\arccos^2\left( 1-\frac{2\Ec}{E} \right),\\
            h &= \frac{1}{3}\frac{1}{E/\Ec-1}\left[ \frac{E}{\Ec}+
                2\left(\frac{E}{\Ec}-2\right)\sqrt{\frac{E/\Ec}{E/\Ec-1}} - \frac{\Zeff-7}{\Zeff+1}
            \right].
        \end{aligned}
    \end{equation}
    The Dreicer field $\ED$ and Connor-Hastie threshold field $\Ec$ are given by
    \begin{align}
        \ED &= \frac{n_e e^3\ln\Lambda}{4\pi\epsilon_0^2 T_e},\\
        \Ec &= \frac{n_e e^3\ln\Lambda}{4\pi\epsilon_0^2 m_ec^2} = \ED\frac{T_e}{m_ec^2}.
    \end{align}


    \subsection*{Derivatives}
    When implementing the runaway rate, we also need to evaluate as many of its
    partial derivatives as possible in order to build the Jacobian for the
    equation system. The runaway rate $\gamma$ is most sensitive to the
    quantity $E/\ED$, and so we can focus on the derivatives with respect
    to $E$, $n_e$ and $T_e$ which affect $E/\ED$. Further, in order to avoid the
    complex dependencies on various quantities in the expressions for $\lambda$,
    $\eta$ and $h$, we will differentiate the runaway rate as if these
    coefficients were all set to one (even when the full expressions are used).

    Differentiating $\gamma$ with respect to the main parameters $E$, $n_e$,
    $T_e$ which we solve for in \DREAM, we find that
    \begin{equation}
        \begin{aligned}
            \frac{\partial\gamma}{\partial E} &\approx
                \frac{\partial x}{\partial E}\frac{\partial\gamma}{\partial x} =
                \frac{x}{E}\frac{\partial\gamma}{\partial x},\\
            %
            \frac{\partial\gamma}{\partial n_e} &\approx \frac{\gamma}{n_e} +
                \frac{\partial x}{\partial n_e}\frac{\partial\gamma}{\partial x} =
                \frac{\gamma}{n_e} - \frac{x}{n_e}\frac{\partial\gamma}{\partial x},\\
            %
            \frac{\partial\gamma}{\partial T_e} &\approx
                \frac{\partial x}{\partial T_e}\frac{\partial\gamma}{\partial x} =
                \frac{x}{T_e}\frac{\partial\gamma}{\partial x},
        \end{aligned}
    \end{equation}
    where we used that $\partial(1/\ED)/\partial n_e = -1/n_e\ED$ and
    $\partial(1/\ED)/\partial T_e = 1/T_e\ED$. This means that by only
    evaluating one derivative analytically we can easily retrieve the three
    derivatives of physical quantities which we are most interested. The
    derivative of $\gamma$ with respect to $x=E/\ED$ is given by
    \begin{equation}
        \begin{aligned}
            \frac{\partial\gamma}{\partial x} &=
                \frac{Cn_e}{\tau_{ee}}x^{-\frac{3}{16}(1+\Zeff)h}
                \exp\left[ -\frac{\lambda}{4x} - \sqrt{\eta\frac{1+\Zeff}{x}} \right]
                \left( \frac{3}{16}\frac{(1+\Zeff)h}{x} + \frac{\lambda}{4x^2} +
                \frac{1}{2x}\sqrt{\eta\frac{1+\Zeff}{x}}\right) =\\
            %
            &= \frac{\gamma}{x}\left(
                \frac{3}{16}(1+\Zeff)h + \frac{\lambda}{4x} +
                \frac{1}{2}\sqrt{\eta\frac{1+\Zeff}{x}}
            \right).
        \end{aligned}
    \end{equation}
    (again keeping $\lambda=\eta=h=1$).

\end{document}
