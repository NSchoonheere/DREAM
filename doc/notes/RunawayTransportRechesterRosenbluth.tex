\documentclass{notes}

\title{Rechester-Rosenbluth-like transport for fluid runaways}
\author{Mathias Hoppe}
\date{2024-11-05}

\newcommand{\Vp}{\mathcal{V}'}

\begin{document}
	\maketitle

	Runaway transport in ergodic magnetic fields can be treated using the
	Rechester-Rosenbluth model. This can already be done in \DREAM\ when the
	runaways are treated kinetically. For fluid runaways, however, no model
	is currently implemented in the code. In this document, we derive a
	Rechester-Rosenbluth transport term for the runaway electron density
	evolution equation.

	\section*{Derivation}
	The Rechester-Rosenbluth model prescribes a diffusion coefficient
	\begin{equation}
		\left\{D_{rr}\right\} = \pi q R_{\rm m}\left(
			\frac{\delta B}{B}
		\right)^2
		v\left|\xi_0\right|
		\left\{\frac{\xi}{\xi_0}\right\},
	\end{equation}
	where $q$ is the safety factor, $R_{\rm m}$ the plasma major radius,
	$\delta B/B$ the magnetic perturbation magnitude, $v_\parallel$ the parallel
	speed of the electron, and $\mathcal{H}$ a step function defined such that
	\begin{equation}
		\mathcal{H}\left(\xi_0\right) =
		\begin{cases}
			1, \quad &\left|\xi_0\right| > \xi_{\rm T}\\
			0, \quad &\left|\xi_0\right|\leq \xi_{\rm T}
		\end{cases}.
	\end{equation}
	The corresponding operator for the kinetic equation is
	\begin{equation}\label{eq:dfre}
		\frac{\partial f}{\partial t} =
			\frac{1}{\Vp}\frac{\partial}{\partial r}\left(
				\Vp \left\{D_{rr}\right\}\frac{\partial f}{\partial r}
			\right),
	\end{equation}
	where $f$ is the distribution function. To obtain a transport operator for
	the evolution equation of the runaway density, $n_{\rm re}$, we first assume
	that $f$ describes only the runaway electrons, and that all runaway
	electrons have momentum $p=p_{\rm re}$ (in normalized units) and pitch
	$\xi_0=1$. Mathematically, $f$ then takes the form
	\begin{equation}
		f\equiv f_{\rm re}\left(p,\xi_0\right) =
			\frac{V'}{\Vp}n_{\rm re}\delta\left(p-p_{\rm re}\right)
				\delta\left(\xi_0-1\right).
	\end{equation}
	With $n_{\rm re}$ defined as the density moment of $f_{\rm re}$, we can then
	find the time variation of $n_{\rm re}$ from equation~\eqref{eq:dfre} as
	\begin{equation}
		\int_0^\infty\dd p\int_{-1}^1\dd\xi_0\,\frac{\Vp}{V'}\frac{\partial f_{\rm re}}{\partial t} =
		\int_0^\infty\dd p\int_{-1}^1\dd\xi_0\,\frac{1}{V'}\frac{\partial}{\partial r}\left(
			\Vp \left\{D_{rr}\right\}\frac{\partial f_{\rm re}}{\partial r}
		\right).
	\end{equation}
	On the LHS we can use the definition of $n_{\rm re}$ and obtain a partial
	time derivative on $n_{\rm re}$. The RHS can in turn be re-ordered as
	follows:
	\begin{equation}\label{eq:momentRHS}
		\begin{gathered}
			\int_0^\infty\dd p\int_{-1}^1\dd\xi_0\,\frac{1}{V'}\frac{\partial}{\partial r}\left(
				\Vp \left\{D_{rr}\right\}\frac{\partial f_{\rm re}}{\partial r}
			\right) =
			%
			\frac{1}{V'}\frac{\partial}{\partial r}\left(
				\int_0^\infty\dd p\int_{-1}^1\dd\xi_0\,
				\Vp\left\{ D_{rr} \right\} \frac{\partial f_{\rm re}}{\partial r}
			\right) =\\
			%
			=
			\frac{1}{V'}\frac{\partial}{\partial r}\left[
				\pi q R_{\rm m}\left(
					\frac{\delta B}{B}
				\right)^2
				\int_0^\infty\dd p\int_{-1}^1\dd\xi_0\,
				\Vp \frac{cp\left|\xi_0\right|}{\sqrt{1+p^2}}
				\left\{\frac{\xi}{\xi_0}\right\}
				\frac{\partial}{\partial r}\left(
					\frac{V'}{\Vp}
					n_{\rm re}
					\delta\left(p-p_{\rm re}\right)
					\delta\left(\xi_0-1\right)
				\right)
			\right].
		\end{gathered}
	\end{equation}
	Before proceeding, we note that since
	\begin{equation}
		\xi = \pm\sqrt{1 - \frac{B}{B_{\rm min}}\left(1-\xi_0^2\right)},
	\end{equation}
	which for $\xi_0=1$ is identically $\xi=1$, the $\xi_0$ delta function
	causes the factor in the bounce average to be one, so that the entire bounce
	average evaluates to one. Allowing the delta functions to act on the other
	factors, we find that
	\begin{equation}
		\eqref{eq:momentRHS} =
			\frac{1}{V'}\frac{\partial}{\partial r}\left[
				V'\pi q R_{\rm m}
				\left(\frac{\delta B}{B}\right)^2
				v_{\rm re}
				\frac{\partial n_{\rm re}}{\partial r}
			\right],
	\end{equation}
	where $v_{\rm re}=cp_{\rm re}/\sqrt{1+p_{\rm re}^2}\approx c$. We therefore
	find that the runaway electron transport term becomes
	\begin{equation}
		\frac{\partial n_{\rm re}}{\partial t} =
			\frac{1}{V'}\frac{\partial}{\partial r}\left[
				V' D_{rr}^{(n)} \frac{\partial n_{\rm re}}{\partial r}
			\right],
	\end{equation}
	with the diffusion coefficient
	\begin{equation}
		D_{rr}^{(n)} =
			\pi q R_{\rm m}c\left(\frac{\delta B}{B}\right)^2.
	\end{equation}
\end{document}
