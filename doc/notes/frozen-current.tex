\documentclass{notes}

\usepackage{hyperref}

\title{Frozen current mode}
\author{Mathias Hoppe}
\date{2023-02-14}

\newcommand{\LUKE}{\textsc{Luke}}
\newcommand{\Vp}{\mathcal{V}'}

\begin{document}
	\maketitle

	A powerful feature of \LUKE\ is its ability to match a prescribed (often
	experimentally measured) plasma current by adapting a diffusive radial
	transport term. For experimental studies, such a term can be very useful
	to constrain the set of unknowns and extract useful information from
	experimental measurements. In this document, we describe the implementation
	of such a feature in \DREAM.

	\tableofcontents

	\section{Background}
	The unknown quantity which is perhaps least constrained by contemporary
	experimental measurements is the rate at which fast electrons are expelled
	from the plasma. Despite this, the fast electron confinement time plays a
	crucial role in shaping the electron distribution function and limiting the
	plasma current. In absence of other measurements, we may therefore use the
	latter to scale the radial transport in order to try to match the
	experimentally measured plasma current. Assuming diffusive transport, as
	may be reasonable in the case of turbulent transport, the model transport
	operator takes the form
	\begin{equation}\label{eq:transp}
		\left.\frac{\partial f}{\partial t}\right|_{\rm transp} =
			\frac{1}{\Vp}\frac{\partial}{\partial r}\left[
				\Vp D_0h(p,\xi_0)\frac{\partial f}{\partial r}
			\right],
	\end{equation}
	where $h(p,\xi_0)$ is an arbitrary function of momentum. In this work, we
	will take $h(p,\xi_0)=1$ and $h(p,\xi_0)=\beta_\parallel=v_\parallel/c$.

	The coefficient $D_0$ is unknown and can be constrained by requiring that
	\begin{equation}\label{eq:Ip}
		I_{\rm p} = I_{\rm p,presc}.
	\end{equation}

	\section{Implementation}
	The frozen current mode should be implemented in the non-linear solver by
	adding the condition~\eqref{eq:Ip} as the equation for $D_0$. However, doing
	so results in numerical instabilities which prevent the solver from
	converging. An alternative approach, which turns out to be stable in
	practice, is to add a separate iteration step. This can also be applied to
	the linearly implicit solver, effectively turning it into a non-linear
	solver.

	The idea of the extra iteration step---here dubbed the ``external
	iterator''---is to solve a subset of the equation system \emph{after}
	obtaining an approximation to the solution of the rest of the equation
	system. This way, the equation for $D_0$ can be formulated in a somewhat
	more explicit manner, hopefully providing better stability (as is also
	observed to be the case in practice).

	The algorithm used for constraining $D_0$ using the external iterator can
	be derived by considering $I_{\rm p}$ in equation~\eqref{eq:Ip} as
	depending only on $D_0$. Since the dependence on $D_0$ is, in general,
	complicated and no explicit analytical expression is available, we can
	employ a secant method to gradually converge towards the exact solution
	$D_0^\star$:
	\begin{equation}\label{eq:D0iter1}
		D_0^{(k+1)} = D_0^{(k)} +
			\left(D_0^{(k)} - D_0^{(k-1)}\right)
			\frac{I_{\rm p}^{(k)}-I_{\rm p,presc}}{I_{\rm p}^{(k)}-I_{\rm p}^{(k-1)}}.
	\end{equation}
	This method can be used to obtain $k+1\geq 2$. For $k=0$ and $k=1$ we need
	other estimates. In \DREAM, we take $D_0^{(0)}$ to be equal to the value of
	$D_0$ in the previous time step. For $D_0^{(1)}$, we either rescale
	$D_0^{(0)}$ according to how the plasma current changed from iteration 0 to
	iteration 1 (simple ratio of plasma currents times $D_0^{(0)}$) if
	$D_0^{(0)} > 0$, or use equation~\eqref{eq:D0app} to obtain an initial guess.

	\subsection{Quadratic iteration methods}
	Alternative methods can be developed to solve for $D_0$. The iterative
	method~\eqref{eq:D0iter1} assumes that $I_p\propto D_0$, which is true close
	to $D_0^\star$. When starting further away from $D_0^\star$ though, this
	method may have difficulties converging.

	The second most straightforward assumption for $I_{\rm p}(D_0)$ is that
	$I_{\rm p}$ depends quadratically on $D_0$. Taylor-expanding to second-order
	around the true solution $D_0^\star$, we find
	\begin{equation}
		I_{\rm p} - I_{\rm p,presc} \equiv \Delta I_{\rm p} \approx
			I_{\rm p}'\left(D_0-D_0^\star\right) +
			\frac{I_{\rm p}''}{2}\left(D_0-D_0^\star\right)^2.
	\end{equation}
	This equation can be solved for $D_0^\star$ to find
	\begin{equation}
		\begin{aligned}
			D_{0\pm}^\star &= D_0 + \frac{I_{\rm p}'}{I_{\rm p}''} \pm
				\sqrt{
					\left(D_0+\frac{I_{\rm p}'}{I_{\rm p}''}\right)^2
					+2\frac{\Delta I_{\rm p} - I_{\rm p}'D_0}{I_{\rm p}''}-D_0^2
				}
				\\
				&=
				D_0 + \frac{I_{\rm p}'}{I_{\rm p}''} \pm
				\sqrt{
					\frac{2\Delta I_{\rm p}}{I_{\rm p}''} +
					\left(\frac{I_{\rm p}'}{I_{\rm p}''}\right)^2
				}
				\\
				&=
				D_0 + \frac{I_{\rm p}'\pm
				\sqrt{
					\left(I_{\rm p}'\right)^2+2I_{\rm p}''\Delta I_{\rm p}
				}}{I_{\rm p}''}.
		\end{aligned}
	\end{equation}
	The corresponding iteration scheme can thus be written
	\begin{equation}
		D_{0\pm}^{(k+1)} =
			D_0^{(k)} +
			\frac{I_{\rm p}' \pm
			\sqrt{
				\left(I_{\rm p}'\right)^2 +
				2I_{\rm p}''\Delta I_{\rm p}^{(k)}
			}}{I_{\rm p}''}
	\end{equation}
	Since the exact functional dependence of $I_{\rm p}$ on $D_0$ is not known,
	we must approximate the derivatives appearing using the knowledge obtained
	in previous iterations. For $k\geq 2$, we can take
	\begin{equation}
		\begin{aligned}
			\left(I_{\rm p}'\right)^{(k)} &= \frac{I_{\rm p}^{(k)}-I_{\rm p}^{(k-1)}}{D_0^{(k)}-D_0^{(k-1)}},\\
			%
			\left(I_{\rm p}''\right)^{(k)} &=
				\frac{\left(I_{\rm p}'\right)^{(k)}-\left(I_{\rm p}'\right)^{(k-1)}}
					{D_0^{(k)}-D_0^{(k-1)}} =
				\\
				&=
				\frac{
					\left(D_0^{(k-1)}-D_0^{(k-2)}\right)I_{\rm p}^{(k)}
					-\left(D_0^{(k)}-2D_0^{(k-1)}+D_0^{(k-2)}\right)I_{\rm p}^{(k-1)}
					+\left(D_0^{(k)}-D_0^{(k-1)}\right)I_{\rm p}^{(k-2)}
				}
				{\left(D_0^{(k)}-D_0^{(k-1)}\right)^2\left(D_0^{(k-1)}-D_0^{(k-2)}\right)}
				.
		\end{aligned}
	\end{equation}
	For $I_{\rm p}'$, since we use a standard finite difference, we can only
	expect an accuracy of at most $\sqrt{\varepsilon}$ in double precision. This
	means that since $I_{\rm p}''$ is evaluated from approximations of
	$I_{\rm p}'$, the best accuracy achievable is merely
	$\sqrt[4]{\varepsilon}\approx 1.22\cdot 10^{-4}$. When attempting to satisfy
	a relative tolerance less than this, one might therefore want to switch to
	the linear method (which should also converge rapidly close to the root).

	\section{Analytical estimate}
	A first approximation to the coefficient $D_0$ can be obtained by assuming
	that $h(p,\xi_0)\approx 1$ and calculating the change in the plasma current
	resulting from adding the term~\eqref{eq:transp} to the Fokker--Planck
	equation. Taking first the current moment of equation~\eqref{eq:transp},
	we obtain the change in the current density due to the transport
	\begin{equation}
		\begin{gathered}
			\frac{2\pi e}{\left\langle1/R^2\right\rangle}
			\int
				v\left\{\frac{\xi}{BR^2}\right\}
				\left.\frac{\partial f}{\partial t}\right|_{\rm transp}
				\frac{\Vp}{V'}\,\dd p\dd\xi_0 =
			\left.\frac{\partial (j_\parallel/B)}{\partial t}\right|_{\rm transp} =\\
			%
			=
			\frac{2\pi e}{\left\langle 1/R^2 \right\rangle V'}
			\int
				v\left\{\frac{\xi}{BR^2}\right\}
				\frac{\partial}{\partial r}\left(
					\Vp D_0 \frac{\partial f}{\partial r}
				\right)\,
				\dd p\dd\xi_0
			\approx
				\frac{1}{V'}\frac{\partial}{\partial r}\left[
					V' D_0 \frac{\partial(j_\parallel/B)}{\partial r}
				\right],
		\end{gathered}
	\end{equation}
	where in the last step it was assumed that
	\begin{equation}
		\frac{\partial}{\partial r}\left(
			\left\langle \frac{1}{R^2}\right\rangle^{-1}
			\left\{\frac{\xi}{BR^2}\right\}
		\right)\approx 0,
	\end{equation}
	which is exactly true in the cylindrical limit. The corresponding change
	to the total plasma current is then
	\begin{equation}
		\begin{gathered}
			\frac{1}{2\pi}\int_0^a V'
				\left.\frac{\partial(j_\parallel/B)}{\partial t}\right|_{\rm transp}
				G(r)\left\langle\frac{1}{R^2}\right\rangle\,\dd r
				=
				\left.\frac{\dd I_{\rm p}}{\dd t}\right|_{\rm transp}
				\approx\\
				%
				\approx
				\frac{1}{2\pi}\int_0^a
					G(r)\left\langle\frac{1}{R^2}\right\rangle
					\frac{\partial}{\partial r}\left[
						V' D_0\frac{\partial(j_\parallel/B)}{\partial r}
					\right]
				\approx
				D_0
				\frac{V'(a)}{2\pi R_0}
				\left.\frac{\partial j_\parallel}{\partial r}\right|_{r=a},
		\end{gathered}
	\end{equation}
	where the last approximate equality holds true in the cylindrical limit.
	This can be used to obtain an initial guess for $D_0$ based on the
	current-rise without transport. From equation~\eqref{eq:Ip} it namely
	follows that
	\begin{equation}\label{eq:dIdt}
		\frac{\dd I_{\rm p}}{\dd t} =
			\left.\frac{\dd I_{\rm p}}{\dd t}\right|_{\rm transp} +
			\left.\frac{\dd I_{\rm p}}{\dd t}\right|_{\rm other}
			= 0.
	\end{equation}
	A first approximation can therefore be obtained when $\dd I_{\rm p}/\dd t$
	at $D_0=0$ is known:
	\begin{equation}\label{eq:D0app}
		D_0\approx \frac{2\pi R_0}{V'(a)}
			\frac{\dd I_{\rm p} / \dd t}
			{\partial j_\parallel/\partial r}.
	\end{equation}

	\appendix

	\section{Quadratic iteration method}
	We start from the second-degree equation
	\begin{equation}
		\Delta I_{\rm p} = I_{\rm p}'\left(D_0-D_0^\star\right) +
			\frac{I_{\rm p}''}{2}\left(D_0-D_0^\star\right)^2.
	\end{equation}
	First we reorganize this into
	\begin{equation}
		\begin{gathered}
			\Delta I_{\rm p} =
			I_{\rm p}'\left(D_0-D_0^\star\right) +
			\frac{I_{\rm p}''}{2}\left[
				D_0^2+\left(D_0^\star\right)^2 - 2D_0D_0^\star
			\right]^2
			=\\
			%
			= \frac{I_{\rm p}''}{2}\left(D_0^\star\right)^2 -
			D_0^\star\left( I_{\rm p}' + I_{\rm p}''D_0 \right) +
			I_{\rm p}'D_0 + \frac{I_{\rm p}''}{2}D_0^2\\
			%
			\Longleftrightarrow\\
			%
			\left(D_0^\star\right)^2 - 2D_0^\star\frac{I_{\rm p}'+I_{\rm p}''D_0}{I_{\rm p}''} =
			2\frac{\Delta I_{\rm p} - I_{\rm p}'D_0 - I_{\rm p}''D_0^2/2}{I_{\rm p}''}\\
			%
			\Longleftrightarrow\\
			%
			\left(D_0^\star-\frac{I_{\rm p}'+I_{\rm p}''D_0}{I_{\rm p}''}\right)^2
			- \left(\frac{I_{\rm p}'+I_{\rm p}''D_0}{I_{\rm p}''}\right)^2 =
			2\frac{\Delta I_{\rm p} - I_{\rm p}'D_0 - I_{\rm p}''D_0^2/2}
			{I_{\rm p}''}\\
			%
			\Longleftrightarrow\\
			%
			D_0^\star = D_0 + \frac{I_{\rm p}'}{I_{\rm p}''}\pm
				\sqrt{
					D_0^2 +
					\left(\frac{I_{\rm p}'}{I_{\rm p}''}\right)^2 +
					\frac{2I_{\rm p}'D_0}{I_{\rm p}''} +
					2\frac{\Delta I_{\rm p} - I_{\rm p}'D_0 - I_{\rm p}''D_0^2/2}
					{I_{\rm p}''}
				}\\
			%
			\Longleftrightarrow\\
			%
			D_0^\star =
				D_0 + \frac{I_{\rm p}'}{I_{\rm p}''} \pm
				\sqrt{
					\left(\frac{I_{\rm p}'}{I_{\rm p}''}\right)^2 +
					\frac{2\Delta I_{\rm p}}{I_{\rm p}''}
				}
				=
				D_0 + \frac{I_{\rm p}'\pm\sqrt{
					\left(I_{\rm p}'\right)^2+2I_{\rm p}''\Delta I_{\rm p}
				}}{I_{\rm p}''}.
		\end{gathered}
	\end{equation}

\end{document}
