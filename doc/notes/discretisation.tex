\documentclass{notes}

\title{Discretisation of position and momentum}
\author{}

\usepackage{hyperref}

\newcommand{\DREAM}{\textsc{DREAM}}
\newcommand{\Vp}{\mathcal{V}'}

\begin{document}
    \maketitle

    \noindent
    This document describes the (spatial/momentum) discretisations used in \DREAM.
    The overarching scheme used for deriving discretisations is the so-called
    finite-volume method (FVM).

    \tableofcontents

    \section{Grid definition}
    The most general grid used in \DREAM\ consists of three coordinates: one
    spatial coordinate $x$ and two momentum coordinates, $p_1$ and $p_2$. For
    a general coordinate $z^{(\alpha)}$, we introduce $N_\alpha$ cell grid points and
    $N_\alpha+1$ flux grid points and impose the definitions
    \begin{equation}
        \begin{aligned}
            \text{Flux grid:} & \qquad z^{(\alpha)}_{\rm min} = z^{(\alpha)}_{1/2} < z^{(\alpha)}_{3/2} < \ldots < z^{(\alpha)}_{N_\alpha+1/2} = z^{(\alpha)}_{\rm max},\\
            \text{Cell grid:} & \qquad z^{(\alpha)}_i = \frac{z^{(\alpha)}_{i+1/2} + z^{(\alpha)}_{i-1/2}}{2}, \quad (1\leq i \leq N_\alpha)\\
            &\qquad \Delta z^{(\alpha)}_i = z^{(\alpha)}_{i+1/2} - z^{(\alpha)}_{i-1/2}, \quad (1\leq i \leq N_\alpha)\\
            &\qquad \Delta z^{(\alpha)}_{i-1/2} = z^{(\alpha)}_i - z^{(\alpha)}_{i-1},\quad (2\leq i \leq N_\alpha).
        \end{aligned}
    \end{equation}

    \section{Boundary conditions}
    This section describes the various boundary conditions used in the code.
    In momentum space, the internal boundary conditions can be described from
    the distribution function symmetry relations
    \begin{equation}\label{eq:fsymmetry}
        \begin{aligned}
            f(-p,\theta) &= f(p,\pi-\theta),\\
            f(p,-\theta) &= f(p,\theta).
        \end{aligned}
    \end{equation}

    \subsection{Boundary conditions on $p/\xi$ grids}
    From the first of the symmetry relations~\eqref{eq:fsymmetry}, we find that
    that the flux through $p=0$ along a given pitch direction must be equal on
    both sides, implying that
    \begin{equation}
        \Phi\left(p=0^+, \xi\right) = \Phi\left(p=0^-, -\xi\right).
    \end{equation}

    From the second of the symmetry relations~\eqref{eq:fsymmetry}, we
    immediately find that the fluxes across the $\xi=\pm1$ boundaries vanish,
    i.e.\
    \begin{equation}\label{eq:bc:xiInternal}
        \Phi(p, \xi=1) = \Phi(p, \xi=-1) = 0.
    \end{equation}

    \section{Spatial discretisations}
    In this section we describe the various discretisations used in \DREAM. We
    however begin by describing the general approach taken to discretising
    equations in the code.

    \subsection{Finite volume method}
    In order to achieve mass conservation, we use the finite-volume method to
    discretise our equations and consider the fluxes between cells on the
    computational grid. For a general flux-conservative term
    $\nabla\cdot\bb{\Phi}$, we take the matrix row corresponding to element
    $i_1\cdots i_D$ ($D$ being the problem dimensionality) of the unknown
    quantity to be
    \begin{equation}
        \left[ \nabla\cdot\bb{\Phi} \right]_{i_1\cdots i_D}  =
            \frac{1}{\Vp_{i_1\cdots i_D}}\sum_\beta \frac{
                \Phi^{(\beta)}_{i_\beta+1/2} \Vp_{i_\beta+1/2} -
                \Phi^{(\beta)}_{i_\beta-1/2} \Vp_{i_\beta-1/2}
            }{\Delta z^{(\beta)}_{i_\beta}},
    \end{equation}
    where $\Vp$ denotes the phase space Jacobian, Greek indices denote the
    coordinate (i.e.\ $z_\alpha, z_\beta, \ldots$), Latin indices denote grid
    points and the matrix indices which are not shifted have been suppressed for
    clarity (i.e.\
    $\Phi^{(\beta)}_{i_\beta+1/2}\equiv\Phi^{(\beta)}_{i_1\cdots i_\beta+1/2\cdots i_D}$).

    \subsection{Advection term}
    A general advection term can be written on the form
    \begin{equation*}
        \nabla\cdot\left( \bb{F} f \right),
    \end{equation*}
    with $\bb{F} = \zhat^{(\alpha)}F^{(\alpha)} + \zhat^{(\beta)}F^{(\beta)}+\ldots$
    denoting the advection coefficient and $f$ the distribution function. For
    momentum-space friction, we discretise such terms according to
    \begin{equation}\label{eq:advection:discr}
        \left[\nabla\cdot\left( \bb{F}f \right) \right]_{i_1\cdots i_D} =
            \frac{1}{\Vp_{i_1\cdots i_D}}\sum_\beta \frac{
                F^{(\beta)}_{i_\beta+1/2}f_{i_\beta+1/2} \Vp_{i_\beta+1/2} -
                F^{(\beta)}_{i_\beta-1/2}f_{i_\beta+1/2} \Vp_{i_\beta-1/2}
            }{\Delta z^{(\beta)}_{i_\beta}},
    \end{equation}
    Since the distribution function is only known on the cell grid, but is here
    required on the flux grid, we must interpolate. We interpolate linearly in
    $f$ according to
    \begin{equation}
        f_{i_\beta-1/2} = \delta^{(\beta)}_{i_\beta} f_{i_\beta} + \left( 1 - \delta^{(\beta)}_{i_\beta} \right) f_{i_\beta-1},
    \end{equation}
    where the interpolation coefficients $\delta^{(\beta)}_{i_\beta}$ are
    determined with the help of any of the methods described in
    section~\ref{sec:interp}. (Note that unnecessary indices have been
    suppressed, so that really
    $\delta^{(\beta)}_{i_\beta} = \delta^{(\beta)}_{i_1\cdots i_\beta\cdots i_D}$).

    \subsubsection{Boundary conditions on $p/\xi$ grid}
    Internal boundary conditions at $p=p_{1/2}$ must be specified explicitly.
    They must then be inserted appropriately into~\eqref{eq:advection:discr} so
    that
    \begin{equation}
        \left[ \nabla\cdot\left( \bb{F}f \right) \right]_{1,j} = \frac{
            \Vp_{3/2} F_{3/2}^{(p)} f_{3/2} - \Vp_{1/2}\Phi^{(p)}_{1/2}
        }{\Delta p_{1}}
    \end{equation}
    where $\Phi_{1/2}^{(p)}$ is the phase space particle flux into or out of
    the inner boundary. At $\xi=\xi_{1/2}$ and $\xi=\xi_{N_\xi+1/2}$, however,
    we use the result~\eqref{eq:bc:xiInternal} and obtain (for the $\xi$ term)
    \begin{equation}
        \begin{aligned}
            \left[ \nabla\cdot\left( \bb{F} f \right) \right]_{i,1} &=
                \frac{\Vp_{3/2} F^{(\xi)}_{i,3/2} f_{i,3/2}}{\Delta\xi_{1}},\\
            %
            \left[ \nabla\cdot\left( \bb{F} f \right) \right]_{i,N_\xi} &=
                -\frac{\Vp_{N_\xi-1/2} F^{(\xi)}_{i,N_\xi-1/2} f_{i,N_\xi-1/2}}{\Delta\xi_{N_\xi}}.
        \end{aligned}
    \end{equation}

    \subsubsection{Boundary conditions on $p_\parallel / p_\perp$ grid}
    \emph{To be written...}

    \section{Interpolating in distribution function}\label{sec:interp}
    In advection terms, as well as diffusion cross terms, we must evaluate the
    distribution function on the flux grid. Since the distribution function is
    only explicitly computed on the cell grid, this means that we must
    interpolate in the distribution function. We use linear interpolation and
    write generally
    \begin{equation}
        f_{i-1/2} = \delta_{i} f_i + \left( 1 - \delta_i \right) f_{i-1}.
    \end{equation}
    In the simplest approach, we take $\delta_i\equiv 1/2$, making $f_{i-1/2}$
    a simple average of the value of $f$ in the adjacent cells. While simple and
    often accurate enough, other schemes for choosing $\delta_i$ can provide
    better stability and even desirable physical properties such as preservation
    of positivity. These schemes are referred to as ``flux limiter schemes'',
    and we also provide those in the code.

    \subsection{Flux limiter schemes}

\end{document}

