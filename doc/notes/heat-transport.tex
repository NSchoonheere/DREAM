\documentclass{notes}

\title{Heat transport in DREAM}
\author{}

\usepackage{amssymb}
\usepackage{hyperref}
\usepackage[
    backend=biber,
    sorting=none,
    style=numeric-comp,
    citestyle=numeric-comp
]{biblatex}
\addbibresource{ref.bib}

\newcommand{\ee}{\mathrm{e}}
\newcommand{\fMe}{f_{\rm Me}}
\newcommand{\ncold}{n_{\rm cold}}
\newcommand{\Te}{T_e}
\newcommand{\Vp}{\mathcal{V}'}
\newcommand{\VpVol}{V'}

\begin{document}
    \maketitle

    \noindent
    This document describes the implementation of various heat transport
    operators in \DREAM. First, we describe the derivation of a heat transport
    operator resulting from constant diffusive radial particle transport.
    After this, we derive the heat transport operator corresponding
    Rechester-Rosenbluth transport in a stochastic magnetic field.

    \tableofcontents

    \section{Constant particle diffusion}\label{sec:constant}
    A general diffusive particle transport operator takes the form
    \begin{equation}\label{eq:transport}
        T\left\{ f \right\} = \frac{1}{\Vp}\frac{\partial}{\partial r}\left(
            \Vp \left\{D_{rr}\right\}\frac{\partial f}{\partial r}
        \right),
    \end{equation}
    where $D_{rr} = D_{rr}(r,p,\xi_0)$ is the diffusion coefficient. In this
    section, we take $D_{rr} = \text{constant}$ and seek the corresponding heat
    transport occuring as a result of this diffusive particle transport. As
    shown in the theory notes, the bounce average of diffusion coefficient which
    is independent of the poloidal angle is
    \begin{equation}
        \left\{ D_{rr} \right\} = D_{rr}\frac{\left\langle B\right\rangle}{\left\langle B/\sqrt{1-\lambda B} \right\rangle}.
    \end{equation}
    The heat transport operator is obtained by taking the energy moment of
    $T\{f\}$ and evaluating $f$ at a Maxwellian with temperature $\Te$ and
    density $\ncold$,
    \begin{equation}
        \fMe(r,p) = \frac{\ncold(r)}{4\pi\Theta(r) K_2(1/\Theta(r))}\exp\left(-\frac{\gamma}{\Theta(r)}\right),
    \end{equation}
    where $\Theta\equiv\Te/mc^2$ and $K_2(x)$ is a modified Bessel function of
    the second kind. The energy moment of~\eqref{eq:transport} thus becomes
    \begin{equation}
        2\pi\int(\gamma-1) T\left\{f\right\}\,\frac{\Vp}{\VpVol}\,\dd p\dd\xi_0 =
        \frac{1}{\VpVol}\frac{\partial}{\partial r}\left[
            \int \Vp\left\{D_{rr}\right\}\frac{\partial\fMe}{\partial r}\,\dd p\dd\xi_0
        \right].
    \end{equation}
    To evaluate the integral within the first radial derivative, we first use
    that
    \begin{equation}
        \begin{aligned}
            2\pi \int\mathrm{d} \xi_0 \,\frac{\mathcal{V}'}{V'} \{D\} &=
            \frac{2\pi}{V'}\int_{-1}^1\mathrm{d} \xi_0  \oint \mathrm{d} \theta \,\sqrt{g}D\\
            %
            &= p^2\frac{2\pi}{V'}\int_{0}^{2\pi}\mathrm{d} \theta  \, \mathcal{J} \oint \mathrm{d} \xi_0 \,\frac{B}{B_\mathrm{min}}\frac{\xi_0}{\xi}D\\
            %
            &= 2\pi p^2 \left\langle \oint \mathrm{d} \xi_0 \,\frac{B}{B_\mathrm{min}}\frac{\xi_0}{\xi}D \right\rangle.
        \end{aligned}
    \end{equation}
    with
    \begin{equation}
        \begin{aligned}
            \xi &= \mathrm{sgn}(\xi_0)\sqrt{1-(1-\xi_0^2)\frac{B}{B_\mathrm{min}}}, \\
            \oint \mathrm{d} \xi_0 &= \int_{-1}^{-\xi_T(\theta)} \mathrm{d}\xi_0+ \int_{\xi_T(\theta)}^1 \mathrm{d}\xi_0,\\
            \xi_T(\theta) &= \sqrt{1-\frac{B_\mathrm{min}}{B}}: \quad \xi(\xi_T) = 0.
        \end{aligned}
    \end{equation}
    For a constant diffusion coefficient, $D\equiv D_0$, we obtain
    \begin{equation}
        \begin{aligned}
            D_0 \oint \mathrm{d} \xi_0 \,\frac{B}{B_\mathrm{min}}\frac{\xi_0}{\xi} &= D_0 \oint \mathrm{d} \xi_0 \,\frac{\partial \xi}{\partial \xi_0} \nonumber \\
            &= D_0 \left( \int_{\xi(-1)}^{\xi(-\xi_T(\theta))} \mathrm{d}\xi + \int_{\xi(\xi_T(\theta))}^{\xi(1)} \mathrm{d}\xi \right) \nonumber \\
            &= D_0 \int_{-1}^1 \mathrm{d}\xi = 2D_0,
        \end{aligned}
    \end{equation}
    so that the energy moment of $T$ can be written
    \begin{equation}
        2\pi\int(\gamma-1) T\left\{f\right\}\,\frac{\Vp}{\VpVol}\,\dd p\dd\xi_0 =
            \frac{1}{\VpVol}\frac{\partial}{\partial r}\left[
                \VpVol D_0\frac{\partial}{\partial r} \left(
                    \frac{\ncold}{\Theta K_2(1/\Theta)} \int_0^\infty (\gamma-1) \ee^{-\gamma/\Theta}\,p^2\,\dd p
                \right)
            \right].
    \end{equation}
    We next change integration variables from $p$ to $\gamma$ and obtain
    \begin{equation}
        \int_0^\infty p^2\left(\gamma-1\right) \ee^{-\gamma/\Theta}\,\dd p =
        \int_1^\infty \gamma\left(\gamma-1\right)\sqrt{\gamma^2-1}\ee^{-\gamma/\Theta}\,\dd\gamma.
    \end{equation}
    We first note that
    \begin{equation}
        \begin{aligned}
            \gamma\sqrt{\gamma^2-1} &= \frac{1}{3}\frac{\dd}{\dd\gamma}\left( \gamma^2 - 1\right)^{3/2},\\
            \gamma^2\sqrt{\gamma^2-1} &= \frac{1}{15}\frac{\dd^2}{\dd\gamma^2}\left( \gamma^2-1 \right)^{5/2}
            - \frac{1}{3}\left( \gamma^2-1 \right)^{3/2}.
        \end{aligned}
    \end{equation}
    Furthermore, integral 8.432.3 of Gradshteyn \& Ryzhik is
    \begin{equation}
        \int_0^\infty \ee^{-zt}\left( t^2-1 \right)^{\nu-1/2}\,\dd t =
        \frac{\Gamma\left(\nu+1/2\right)}{\left(z/2\right)^\nu\Gamma\left(1/2\right)} K_\nu(z).
    \end{equation}
    Integrating by parts, we find
    \begin{equation}
        \begin{gathered}
            \int_1^\infty \gamma\left(\gamma-1\right)\sqrt{\gamma^2-1}\ee^{-\gamma/\Theta}\,\dd\gamma =\\
            %
            \int_1^\infty \left[
                \frac{1}{15}\frac{\dd^2}{\dd\gamma^2}\left( \gamma^2-1 \right)^{5/2}
                - \frac{1}{3}\left( \gamma^2-1 \right)^{3/2}
            \right] \ee^{-\gamma/\Theta}\,\dd\gamma - 
            \frac{1}{3}\int_1^\infty\frac{\dd}{\dd\gamma}\left( \gamma^2 - 1\right)^{3/2} \ee^{-\gamma/\Theta}\,\dd\gamma =\\
            %
            \int_1^\infty\left[ \frac{1}{15\Theta^2}\left(\gamma^2-1\right)^{5/2} -
                \frac{1}{3}\left( \gamma^2 - 1\right)^{3/2} \right] \ee^{-\gamma/\Theta}\,\dd\gamma -
            \frac{1}{3\Theta}\int_1^\infty \left( \gamma^2 - 1\right)^{3/2} e^{-\gamma/\Theta}\,\dd\gamma =\\
            %
            \Theta K_3(1/\Theta) - \Theta^2 K_2(1/\Theta) - \Theta K_2(1/\Theta).
        \end{gathered}
    \end{equation}
    For a Maxwellian we therefore have that the energy moment is
    \begin{equation}
        \begin{gathered}
            2\pi\int\left( \gamma-1 \right)\fMe\,p^2\,\dd p\dd\xi_0 =
            \frac{\ncold}{\Theta K_2(1/\Theta)} \left[
                \Theta K_3\left(1/\Theta\right) - \Theta^2 K_2(1/\Theta) - \Theta K_2(1/\Theta)
            \right] =\\
            %
            \ncold\left[ \frac{K_3(1/\Theta)}{K_2(1/\Theta)} - 1 - \Theta \right].
        \end{gathered}
    \end{equation}
    In the limit of low temperatures ($\Theta\ll 1$) the Bessel function ratio
    is approximately
    \begin{equation}
        \frac{K_3(1/\Theta)}{K_2(1/\Theta)}\approx 1 + \frac{5}{2}\Theta,
    \end{equation}
    such that the energy moment becomes the familiar $3\ncold\Theta/2$.

    The heat transport term corresponding to a constant diffusion coefficient
    is then
    \begin{equation}
        2\pi\int\left(\gamma-1\right) T\left\{f\right\}\,\frac{\Vp}{\VpVol}\,\dd p\dd\xi_0 =
        %
        \frac{1}{\VpVol}\frac{\partial}{\partial r}\left[
            \VpVol\left\{ D_{rr} \right\}\frac{\partial}{\partial r} \left( \frac{3}{2}\ncold\Theta \right)
        \right].
    \end{equation}

    \section{Rechester-Rosenbluth diffusion}
    For Rechester-Rosenbluth diffusion, the diffusion coefficient takes the form
    \begin{equation}
        D_{rr} = \pi qR_0\left( \frac{\delta B}{B} \right)^2 \left| v_\parallel \right|\equiv
        D_0 \left| v_\parallel \right|.
    \end{equation}
    where $q=q(r)$ is the safety factor, $\delta B$ is the magnetic perturbation
    and $v_\parallel$ is the parallel speed. Following the prescription in the
    previous section, we find that the $\xi_0$ integral of the bounce average of
    this coefficient is
    \begin{equation}
        \begin{aligned}
            \int\dd\xi_0\,\frac{\Vp}{\VpVol}\left\{ D_{rr} \right\} &=
            %
            D_0v\left\langle\oint\dd\xi_0\,\frac{B}{B_{\rm min}}\frac{\xi_0}{\xi}\left|\xi\right|\right\rangle =\\
            %
            &= 2D_0v\left\langle \frac{B}{B_{\rm min}}\int_{\xi_{T0}}^1\xi_0\,\dd\xi_0\right\rangle =\\
            %
            &= 2D_0v\left( 1 - \xi_{T0} \right)\left\langle \frac{B}{B_{\rm min}} \right\rangle.
        \end{aligned}
    \end{equation}

    \subsection{Heat transport}
    The heat transport corresponding to the Rechester-Rosenbluth particle
    diffusion is obtained, as in section~\ref{sec:constant}, by calculating the
    energy moment of the transport operator:
    \begin{equation}
        \begin{gathered}
            2\pi\int\left(\gamma-1\right) T\left\{f\right\}\,p^2\,\dd p\dd\xi_0 =
                \frac{2\pi}{V'}\frac{\partial}{\partial r}\left[ V'
                    \int\left(\gamma-1\right)\left\{D_{rr}\right\} \frac{\partial\fMe}{\partial r}\,p^2\,\dd p\dd\xi_0
                \right] =\\
            %
            = \frac{4\pi}{V'}\frac{\partial}{\partial r}\left[ V'D_0
                \left(1-\xi_{T0}\right)\left\langle\frac{B}{B_{\rm min}}\right\rangle
                \frac{\partial}{\partial r}\int_0^\infty c\left(\gamma-1\right)p^2\fMe\,\dd p
            \right],
        \end{gathered}
    \end{equation}
    With a change of variables from $p$ to $\gamma$, the integral over $p$
    becomes
    \begin{equation}
        \begin{gathered}
            \int_0^\infty c\left(\gamma-1\right)\left(\gamma^2-1\right)\fMe\,\dd p =\\
            %
            = \frac{c\ncold}{4\pi\Theta K_2(1/\Theta)}\int_1^\infty\left(\gamma^3-\gamma^2-\gamma+1\right)\ee^{-\gamma/\Theta}\,\dd\gamma =\\
            %
            = \frac{c\ncold\ee^{-1/\Theta}}{4\pi\Theta K_2(1/\Theta)}\left(
                6\Theta^4 + 6\Theta^3 + 3\Theta^2 + \Theta - 2\Theta^3 - 2\Theta^2 - \Theta -
                \Theta^2 - \Theta + \Theta
            \right) =\\
            %
            = \frac{c\ncold\ee^{-1/\Theta}}{4\pi K_2(1/\Theta)}\left(
                6\Theta^3 + 4\Theta^2
            \right).
        \end{gathered}
    \end{equation}
    In the low temperature limit, the Bessel function expands to
    \begin{equation}
        K_2\approx \sqrt{\frac{\pi\Theta}{2}}\ee^{-1/\Theta}\left( 1 + \frac{15}{8}\Theta \right),
    \end{equation}
    yielding the following form for the expanded operator:
    \begin{equation}
        T_{\rm heat} = \frac{1}{\VpVol}\frac{\partial}{\partial r}\left[
            \VpVol cD_0\sqrt{\frac{2}{\pi}}\left(1-\xi_{\rm T0}\right)\left\langle\frac{B}{B_{\rm min}} \right\rangle
            \frac{\partial}{\partial r}\left( 
                \ncold \Theta^{3/2}\left[ 1 - \frac{3}{8}\Theta + \frac{45}{16}\Theta^2 + \ldots \right]
            \right)
        \right].
    \end{equation}
    To get this into a form suitable for implementation in \DREAM, we will use
    the chain rule to obtain $T_{\rm heat}$ as an operator that is applied to
    $\Theta$. We will further assume that $\partial\ncold/\partial r$ which,
    although not generally true, can be interpreted as neglecting the radial
    transport of cold electrons as not assuming this would give rise to a
    diffusion term acting on $\ncold$. Since we assume that the ions are not
    transported (in this model), the $\ncold$ cannot be transported either,
    causing the assumption $\partial\ncold/\partial r$ to be somewhat motivated
    by physics. In reality, the $\ncold$ transport would be exactly balanced by
    some source term which ensures quasi-neutrality.

    With the above, we find
    \begin{equation}
        \frac{\partial}{\partial r}\left[ \ncold\Theta^{3/2} \left(
            1 - \frac{3}{8}\Theta
        \right)\right] =
        \ncold\Theta^{1/2}\left( \frac{3}{2} - \frac{15}{16}\Theta \right)
        \frac{\partial\Theta}{\partial r}.
    \end{equation}
    Bringing everything together, the diffusion coefficient for the
    Rechester-Rosenbluth heat transport is
    \begin{equation}
        \left\{ D_{rr} \right\} =
        \sqrt{2\pi}cqR_0\left(\frac{\delta B}{B}\right)^2\left(1-\xi_{\rm T0}\right)
        \left\langle\frac{B}{B_{\rm min}}\right\rangle\ncold
        \frac{3\Theta^{1/2}}{2}\left( 1 - \frac{5}{8}\Theta \right).
    \end{equation}
    We will also need partial derivatives of the diffusion coefficient when
    building the Jacobian matrix. The derivatives needed are
    \begin{equation}
        \begin{aligned}
            \frac{\partial}{\partial\ncold}\left\{ D_{rr} \right\} &= \frac{\left\{ D_{rr} \right\}}{\ncold},\\
            \frac{\partial}{\partial\Te}\left\{ D_{rr} \right\} &=
                \sqrt{2\pi}cqR_0\left(\frac{\delta B}{B}\right)^2\left(1-\xi_{\rm T0}\right)
                \left\langle\frac{B}{B_{\rm min}}\right\rangle\ncold
                \frac{3}{4\Theta^{1/2}}\left( 1 - \frac{15}{8}\Theta \right).
        \end{aligned}
    \end{equation}

\textcolor{blue}{
Corrected geometric factor [Eq (19) for constant diffusion and (21) for Rechester-Rosenbluth]: 
the pitch integral of the bounce average is given by
\begin{align}
2\pi \int\mathrm{d} \xi_0 \,\frac{\mathcal{V}'}{V'} \{D\} &= \frac{2\pi}{V'}\int_{-1}^1\mathrm{d} \xi_0  \oint \mathrm{d} \theta \,\sqrt{g}D\nonumber\\
&= p^2\frac{2\pi}{V'}\int_{0}^{2\pi}\mathrm{d} \theta  \, \mathcal{J} \oint \mathrm{d} \xi_0 \,\frac{B}{B_\mathrm{min}}\frac{\xi_0}{\xi}D \nonumber \\
&= 2\pi p^2 \left\langle \oint \mathrm{d} \xi_0 \,\frac{B}{B_\mathrm{min}}\frac{\xi_0}{\xi}D \right\rangle.
\end{align}
Here,
\begin{align}
\xi &= \mathrm{sgn}(\xi_0)\sqrt{1-(1-\xi_0^2)\frac{B}{B_\mathrm{min}}}, \nonumber \\
\oint \mathrm{d} \xi_0 &= \int_{-1}^{-\xi_T(\theta)} \mathrm{d}\xi_0+ \int_{\xi_T(\theta)}^1 \mathrm{d}\xi_0, \nonumber \\
\xi_T(\theta) &= \sqrt{1-\frac{B_\mathrm{min}}{B}}: \quad \xi(\xi_T) = 0.
\end{align}
Furthermore, a useful relation is given by
\begin{equation}
\frac{B}{B_\mathrm{min}}\frac{\xi_0}{\xi} = \frac{\partial \xi}{\partial \xi_0}.
\end{equation}
For constant $D = D_0$, we obtain
\begin{align}
D_0 \oint \mathrm{d} \xi_0 \,\frac{B}{B_\mathrm{min}}\frac{\xi_0}{\xi} &= D_0 \oint \mathrm{d} \xi_0 \,\frac{\partial \xi}{\partial \xi_0} \nonumber \\
&= D_0 \left( \int_{\xi(-1)}^{\xi(-\xi_T(\theta))} \mathrm{d}\xi + \int_{\xi(\xi_T(\theta))}^{\xi(1)} \mathrm{d}\xi \right) \nonumber \\
&= D_0 \int_{-1}^1 \mathrm{d}\xi = 2D_0,
\end{align}
such that the net momentum space integral becomes $4\pi p^2 D_0$, as used in (3), yielding no geometric correction.
}

\textcolor{blue}{ 
For the Rechester-Rosenbluth model, on the other hand, we have $D = D_0 |\xi|$ when $|\xi_0| > \xi_{T0}$ (eg for all passing particles) and otherwise 0. Here, $\xi_{T0} = \sqrt{1-B_\mathrm{min}/B_\mathrm{max}}$. Then, the integral becomes
\begin{align}
D_0 \oint \mathrm{d} \xi_0 \,\frac{B}{B_\mathrm{min}}\frac{\xi_0}{\xi} |\xi|  
&= 2D_0 \frac{B}{B_\mathrm{min}} \int_{\xi_{T0}}^1 \xi_0 \, \mathrm{d}\xi_0 \nonumber \\
&= 2D_0 \frac{B}{B_\mathrm{min}}(1-\xi_{T0}).
\end{align}
Therefore, the pitch integral over the RR model yields an additional geometric factor of
\begin{equation}
(1-\xi_{T0}) \left\langle \frac{B}{B_\mathrm{min}} \right\rangle.
\end{equation}
In the large-aspect-ratio limit, with $\epsilon = r/R$, this geometric correction is given by $(\sqrt{1+\epsilon}-\sqrt{2\epsilon})/\sqrt{1-\epsilon}$, which decreases monotonically from 1 at $\epsilon=0$ to 0.5 at $\epsilon \approx 0.2$. 
}

\end{document}

