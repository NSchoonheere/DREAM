\documentclass[11pt,a4paper]{article}
\usepackage{diagbox}
\usepackage{wrapfig}
\usepackage[utf8]{inputenc}
%\usepackage[swedish]{babel}
\usepackage{graphicx}
\usepackage{amsmath}
\usepackage{amssymb}
\usepackage{units}
\usepackage{ae}
\usepackage{icomma}
\usepackage{color}
\usepackage{graphics} 
\usepackage{bbm}
\usepackage{float}

\usepackage{caption}
\usepackage{subcaption}

\usepackage{hyperref}
\usepackage{epstopdf}
\usepackage{epsfig}
\usepackage{braket}
\usepackage{pdfpages}

\usepackage{tcolorbox}

\newcommand{\N}{\ensuremath{\mathbbm{N}}}
\newcommand{\Z}{\ensuremath{\mathbbm{Z}}}
\newcommand{\Q}{\ensuremath{\mathbbm{Q}}}
\newcommand{\R}{\ensuremath{\mathbbm{R}}}
\newcommand{\C}{\ensuremath{\mathbbm{C}}}
\newcommand{\id}{\ensuremath{\,\mathrm{d}}}
\newcommand{\rd}{\ensuremath{\mathrm{d}}}
\newcommand{\Ordo}{\ensuremath{\mathcal{O}}}% Stora Ordo
\renewcommand{\L}{\ensuremath{\mathcal{L}}}% Stora Ordo
\newcommand{\sub}[1]{\ensuremath{_{\text{#1}}}}
\newcommand{\ddx}[1]{\ensuremath{ \frac{\partial}{\partial #1} }}
\newcommand{\ddxx}[2]{\ensuremath{ \frac{\partial^2}{\partial #1 \partial #2} }}
%\newcommand{\sup}[1]{\ensuremath{^{\text{#1}}}}
\renewcommand{\b}[1]{\ensuremath{ {\bf #1 } }}
\renewcommand{\arraystretch}{1.5}

\begin{document}

\begin{center}
\Large \bf Background theory for 1D disruption simulation framework.
\end{center}

\vspace{10mm}

\textsc{Dream} is a self-consistent 1D+2P bounce-averaged Fokker-Planck solver specialized for modelling runaway electrons generated during tokamak disruptions. It is a modular framework that can be run in various different modes, where the electron dynamics can be captured in various levels of approximation: using a full 2D momentum space description (like the \textsc{Luke} or \textsc{Cql3D} codes); a reduced 1D model averaged over pitch angle using a novel approach that has not yet been used in the literature (but based on the same approximations as used to derive the classical analytical growth rates); or a 0D fluid model (as an extended version of the existing \textsc{Go} code).

Main features of the model:
\begin{itemize}
\item Bounce-averaged Fokker-Planck equation for the electron population with non-linear collisions and arbitrary prescribed advection-diffusion radial transport coefficients.
\item Non-equilibrium atomic physics model where all ion charge states are treated as separate fluids, interacting via atomic rate coefficients (from the \textsc{Adas} database) and having arbitrary prescribed advection-diffusion radial transport, including also fast-electron impact ionisation using the model by (N Garland \emph{et al} PoP 2020).
\item Electron temperature evolution including ohmic heating, collisional energy transfer from hot electrons and ions, radial transport with advection-diffusion as well as radiated power from atomic rate coefficients. 
\item Resistive poloidal flux evolution, including the option to model rapid current flattening via a helicity-conserving hyperresistive term.
\item Supports arbitrary prescribed 2D magnetic field geometries
\end{itemize}


\tableofcontents





\section{Magnetic geometry, moments, flux surface averages and bounce average}
We will describe dynamics in an axisymmetric magnetic field, which forms nested flux surfaces that we can parametrize with coordinates $(r,\,\theta,\,\varphi)$ defined by
\begin{align}
\b{x} &= R\hat{R} + z\hat{z}, \nonumber \\
R &= R_m + \Delta(r) + r\cos[\theta +\delta(r) \sin\theta], \nonumber \\
z &= r \kappa(r) \sin\theta, \nonumber \\
\hat{R} &= \cos\varphi \hat{x} + \sin\varphi \hat{y}.
\label{eq:geometry coordinates}
\end{align}
With this parametrisation we assume that the flux surfaces can be characterised by the major radius of the magnetic axis $R_m$, the elongation $\kappa(r)$, the Shafranov shift $\Delta(r)$ and the triangularity $\delta(r)$.
With such a parametrization, an arbitrary magnetic field can be represented by
\begin{align}
\b{B} &= G(r) \nabla \varphi + \nabla \varphi \times \nabla \psi(r),
\end{align}
where $\psi$ denotes the poloidal magnetic flux and $G$ describes the toroidal magnetic field.

\subsection{Flux surface average}
The volume integral of an arbitrary quantity $X$ can be written
\begin{align}
\int X \,\rd V &= \int\rd r  \int_0^{2\pi}\rd\theta \int_0^{2\pi}\rd \varphi \,\mathcal{J}X \nonumber \\
&= \int \rd r \, V'\langle X \rangle
\end{align}
Here, we have defined the \emph{flux surface average} denoted $\langle X \rangle$, which we define as
\begin{align}
\langle X \rangle &= \frac{1}{V'} \int_0^{2\pi}\rd \theta \int_0^{2\pi}\rd \varphi \, \mathcal{J}X, \nonumber \\
V' &= \int_0^{2\pi}\rd \theta \int_0^{2\pi}\rd \varphi  \, \mathcal{J},
\end{align}
where $\mathcal{J}$ denotes the Jacobian for the coordinate system $(r,\,\theta,\,\varphi)$, defined as
\begin{align}
\mathcal{J} =\left| \frac{\partial \b{x}}{\partial r}\cdot \left( \frac{\partial \b{x}}{\partial \theta}\times \frac{\partial\b{x}}{\partial \varphi}\right)\right| = \frac{1}{|\nabla r \cdot(\nabla \theta \times \nabla \varphi)|},
\end{align}
where the last relation allows us to write $\mathcal{J} = (\partial \psi/\partial r) /|\b{B}\cdot\nabla \theta|$.

\subsection{Spatial Jacobian and $\nabla r$}
In terms of a parametrization $R=R(r,\,\theta)$ and $z = z(r,\,\theta)$, we obtain
\begin{align}
\frac{\partial \b{x}}{\partial r} &= \frac{\partial R}{\partial r}\hat{R} + \frac{\partial z}{\partial r}\hat{z} , \nonumber \\
\frac{\partial \b{x}}{\partial \theta} &= \frac{\partial R}{\partial \theta}\hat{R} + \frac{\partial z}{\partial \theta}\hat{z} , \nonumber \\
\frac{\partial \b{x}}{\partial \varphi} &= R\frac{\partial \hat{R}}{\partial \varphi} = R\hat{\varphi}.
\end{align}
The Jacobian can then be expressed as
\begin{align}
\mathcal{J} &= \text{det}\left|\begin{matrix}
0 & 0 & R \\
\frac{\partial R}{\partial r} & \frac{\partial z}{\partial r} & 0 \\
\frac{\partial R}{\partial \theta} & \frac{\partial z}{\partial \theta} & 0 
\end{matrix}\right| \nonumber \\
&= R\left| \frac{\partial R}{\partial r}\frac{\partial z}{\partial \theta} - \frac{\partial R}{\partial \theta}\frac{\partial z}{\partial r}\right|.
\end{align}
In terms of our specific parametrization, we can evaluate
\begin{align}
\frac{\partial R}{\partial r} &= \Delta' + \cos[\theta+\delta\sin\theta]-r\delta'\sin\theta\sin[\theta+\delta\sin\theta], \nonumber \\
\frac{\partial R}{\partial \theta} &= -r(1+\delta\cos\theta)\sin[\theta+\delta\sin\theta],  \nonumber \\
\frac{\partial z}{\partial r} &=  \kappa\left(1+ \frac{r\kappa'}{\kappa}\right)\sin\theta, \nonumber \\
\frac{\partial z}{\partial \theta} &= r\kappa\cos\theta,
\end{align}
and obtain
\begin{align}
\hspace{-10mm} \mathcal{J} %&= \kappa r R \biggl\{ \Bigl(\Delta' + \cos[\theta + \delta\sin\theta] \Bigr)\cos\theta + \sin\theta\sin[\theta+\delta\sin\theta] \biggl[(1+\delta\cos\theta)\left(1+\frac{r\kappa'}{\kappa}\right) -r\delta'\cos\theta \biggr]  \biggr\} \nonumber \\
= \kappa r R \left\{ \cos(\delta\sin\theta) + \Delta'\cos\theta + \sin\theta\sin[\theta+\delta\sin\theta]\left[\frac{r\kappa'}{\kappa} + \delta\cos\theta\left(1+\frac{r\kappa'}{\kappa}-\frac{r\delta'}{\delta} \right) \right] \right\}.
\end{align}

For some calculations we will need to evaluate $\nabla r$. We can do so in the following way; taking the gradient of $R=R(r,\,\theta)$ and $z = z(r,\,\theta)$ yields
\begin{align}
\nabla R &= \frac{\partial R}{\partial r}\nabla r + \frac{\partial R}{\partial \theta}\nabla\theta, \nonumber \\
\nabla z &= \frac{\partial z}{\partial r} \nabla r + \frac{\partial z}{\partial \theta}\nabla \theta,
\end{align}
which we can invert to yield
\begin{align}
\begin{pmatrix}
\nabla r \\
\nabla \theta
\end{pmatrix} &= 
\frac{1}{\frac{\partial R}{\partial r}\frac{\partial z}{\partial \theta} - \frac{\partial R}{\partial \theta}\frac{\partial z}{\partial r}}
\begin{pmatrix}
\frac{\partial z}{\partial \theta} & -\frac{\partial R}{\partial \theta} \\
-\frac{\partial z}{\partial r} & \frac{\partial R}{\partial r}
\end{pmatrix}
\begin{pmatrix}
\nabla R \\
\nabla z
\end{pmatrix}
 \nonumber \\
&=\frac{R}{\mathcal{J}}\begin{pmatrix}
\frac{\partial z}{\partial \theta} & -\frac{\partial R}{\partial \theta} \\
-\frac{\partial z}{\partial r} & \frac{\partial R}{\partial r}
\end{pmatrix} \begin{pmatrix}
\nabla R \\
\nabla z
\end{pmatrix}
.
\end{align}
For example, in particular, this allows us to evaluate
\begin{align}
|\nabla r|^2 &= \frac{R^2}{\mathcal{J}^2} \left[ \left(\frac{\partial z}{\partial \theta}\right)^2 + \left(\frac{\partial R}{\partial \theta}\right)^2\right] \nonumber \\
&= \frac{\kappa^2 r^2R^2}{\mathcal{J}^2}\Big[  \cos^2\theta + \frac{1}{\kappa^2}(1+\delta\cos\theta)^2\sin^2(\theta+\delta\sin\theta)\Big]
\end{align}


\subsection{Bounce average}
A similar averaging procedure can be used for phase space densities, where the phase--space volume integral of an arbitrary quantity $X(\b{x},\,\b{p})$ is
\begin{align}
\int X \,\rd\b{x} \rd\b{p}.
\end{align}
We shall describe phase-space dynamics in a zero-orbit width drift kinetic description in the banana limit (where the orbit time is the shortest time scale in the system after the gyrofrequency), where we transform to a guiding-center phase space $\rd\b{x} \rd\b{p} = 2\pi \rd \b{X} p^2 \rd p \rd \xi$ where $\b{X}$ denotes the guiding-center position, $p$ is the momentum and $\xi = \b{p}\cdot\b{b}/p$ denotes the pitch-angle cosine. In drift-kinetics the energy $p^2/2m$ and magnetic moment $\mu = p_\perp^2/2mB$ are conserved quantities, and it is convenient to switch to the magnetic moment as a pitch-angle like coordinate, which we do by changing variables from $\xi$ to $\lambda$ defined by
\begin{align}
\lambda &= \frac{2m \mu}{p^2} = \frac{p_\perp^2}{B p^2} = \frac{1-\xi^2}{B}, \nonumber \\
\rd \lambda &= -\frac{2\xi}{B}\rd \xi = -2\sigma\frac{\sqrt{1-\lambda B}}{B} \rd \xi,
\end{align}
where we have introduced $\sigma = \text{sgn}(\xi)$ since $\lambda$ does not distinguish positive from negative $\xi$. Particles with $\lambda > \lambda_T = 1/B\sub{max}$ will be trapped, where the bounce points are given by $\theta\sub{bounce}(\lambda):~\lambda B(\theta\sub{bounce}) = 1$.

In this case, we can write the phase-space volume integral as
\begin{align}
\int X \,\rd\b{x} \rd\b{p} &= 2\pi \int \rd r \int_0^{2\pi} \rd \varphi \oint \rd \theta \, \int_0^\infty \rd p \int_0^{1/B\sub{max}} \!\rd \lambda \, \mathcal{J}p^2\frac{B}{\sqrt{1-\lambda B}}\frac{1}{2}\sum_\sigma X \nonumber \\
&= \int \rd r \rd p \rd \lambda \sum_\sigma \mathcal{V}' \{ X \},
\end{align}
where we introduced the bounce average $\{X\}$ (which integrates over all quantities upon which the distribution does not depend, and therefore in the trapping region also sums over the signs of $v_\parallel$) as
\begin{align}
\{ X\} &= \frac{1}{\mathcal{V}'} \int_0^{2\pi} \rd \varphi \oint \rd \theta \sqrt{g}X, \nonumber \\
\mathcal{V}' &= \int_0^{2\pi} \rd \varphi \oint \rd \theta \sqrt{g}, \nonumber \\
\oint\rd\theta &= \begin{cases}
\int_0^{2\pi}\rd\theta, & \text{passing }(\lambda \leq \lambda_T)\\
\sum_\sigma \int_{-\theta\sub{bounce}}^{\theta\sub{bounce}} \rd\theta, & \text{trapped } (\lambda > \lambda_T)
\end{cases} \\
\sqrt{g} &= \frac{p^2}{2}\frac{B}{\sqrt{1-\lambda B}}\mathcal{J} ,
\end{align}
where we let $\sqrt{g}$ denote the phase-space Jacobian and $\mathcal{J}$ the spatial Jacobian. Note that in the trapping region $\lambda > \lambda_T$, the integrand is summed over $\sigma$ both inside and outside the bounce average, so that the outer sum will always yield a factor of 2.
It can be pointed out that for a passing particle in the $(p,\,\lambda,\,\sigma)$ coordinate system, the bounce average is related to the flux average via 
\begin{align}
\{X\}\sub{passing} &= \frac{p^2}{2}\frac{V'}{\mathcal{V}'}\left\langle \frac{B}{\sqrt{1-\lambda B}}X\right\rangle \nonumber \\
&= \frac{1}{\langle B/\sqrt{1-\lambda B} \rangle}\left\langle \frac{B}{\sqrt{1-\lambda B}}X\right\rangle 
\end{align}

\subsection{Bounce averaged kinetic equation: general procedure}
We consider a kinetic equation taking the form of a general advection-diffusion equation in phase space with an arbitrary source $S$,
\begin{align}
\frac{\partial f}{\partial t} &= \frac{\partial}{\partial \b{z}} \cdot \left( -\b{A} f + \mathsf{D}\cdot \frac{\partial f}{\partial \b{z}}\right) + S(\b{z}) \nonumber \\
\frac{\partial f}{\partial t} &= \frac{1}{\sqrt{g}}\frac{\partial}{\partial z^i}\left[ \sqrt{g} \left(- A^i f +  D^{ij}\frac{\partial f}{\partial z^j}\right)\right] + S(\b{z}),
\end{align}
where $A^i = (\partial z^i/\partial \b{z}) \cdot \b{A}$ and $D^{ij} = (\partial z^i/\partial \b{z}) \cdot \mathsf{D} \cdot (\partial z^j/\partial \b{z})$.
Since in our ordering, $f$ is only a function of the three orbit constants of motion, for example $p,\,\lambda,\,r$, we can bounce average this equation (averaging over the remaining coordinates), yielding
\begin{align}
\frac{\partial f}{\partial t} &= \frac{1}{\mathcal{V}'}\frac{\partial}{\partial z^i} \left[ \mathcal{V}'\left( -\{A^i\} f + \{D^{ij}\} \frac{\partial f}{\partial z^j} \right)\right] + \{S\},
\label{eq:general bounce average}
\end{align} 
where the sums now only go over the constants of motion ($p$, $\lambda$ and $r$) as all other derivatives are annihilated by the bounce average. The only subtlety concerns the $\lambda$ integral, since the bounce average depends on $\lambda$ via the integration limits $\theta\sub{bounce}(\lambda)$ for the trapped orbits. However, this is resolved by the fact that $A^\lambda = (\partial\lambda/\partial \b{z}) \cdot \b{A}$ vanishes at the end points of the integral, because  $\nabla \lambda = -2\xi\nabla \xi/B = 0$ vanishes as $\xi=0$ at those points by definition; likewise for the diffusion term.

Then, the bounce averaging procedure reduces to removing all derivatives with respect to $\theta$ and $\varphi$ and replacing the coefficients with their bounce-averaged counterpart, and change the Jacobian from $\sqrt{g}$ to $\mathcal{V}'$.


\subsection{Moments of the distribution function, plasma current}
The total particle number is given by 
\begin{align}
N &= \int f \, \rd \b{x} \rd \b{p}  = \sum_\sigma 2\pi\int \rd r \rd p \rd \lambda \mathcal{V}' f \nonumber \\
&= \int n \,\rd\b{x} = \int \rd r \, V' \langle n \rangle
\end{align}
where the particle number density $n$ is given by
\begin{align}
n &= \int f \rd \b{p} = 2\pi \sum_\sigma \int \rd p \rd\lambda \frac{p^2}{2}\frac{B}{\sqrt{1-\lambda B}} f \nonumber \\
&=\frac{2\pi}{\mathcal{J}} \sum_\sigma \int \rd p \rd\lambda \sqrt{g}f.
\end{align}
For a beam-like distribution with $\lambda \approx 0$, we see that $n/B$ is constant on flux surfaces.

\subsubsection*{Plasma current}
The plasma current density is given by
\begin{align}
j_\parallel = \int ev \xi f \,\rd\b{p} = 2\pi eB\int_0^{p\sub{max}} \rd p \, vp^2 \int_0^{1/B\sub{max}} \rd \lambda \, \frac{1}{2}\sum_\sigma  \sigma f,
\end{align}
where we only integrate over passing orbits, $\lambda \leq 1/B\sub{max}$, since the sum over $\sigma$ kills the trapped contribution since $f$ is even in $\sigma$ in the trapped region.

The toroidal plasma current enclosed within the flux surface $r$ is given by
\begin{align}
I(r) &= \int \b{j}\cdot \rd\b{s} = \int \b{j}\cdot \left(\frac{\partial \b{x}}{\partial r}\times \frac{\partial \b{x}}{\partial \theta}\right) \rd r \rd \theta \nonumber \\
&= \int (\b{j}\cdot\nabla \varphi )R\left|\frac{\partial \b{x}}{\partial r}\times \frac{\partial \b{x}}{\partial \theta} \right| \rd r \rd \theta \nonumber \\
&= \frac{1}{2\pi} \int \rd r \rd\theta \rd\varphi \, (\b{j}\cdot\nabla \varphi )\left|\frac{\partial \b{x}}{\partial \varphi}\cdot\left(\frac{\partial \b{x}}{\partial r}\times \frac{\partial \b{x}}{\partial \theta} \right) \right|   \nonumber \\
&=\frac{1}{2\pi} \int \rd r \rd\theta \rd\varphi \, \mathcal{J}(\b{j}\cdot\nabla \varphi ),
\end{align}
where we explicitly used the fact that $\partial \b{x}/\partial r$ and $\partial \b{x}/\partial \theta$ are orthogonal to $\partial \b{x}/\partial \varphi$. In a force-free (pressureless) equilibrium, the perpendicular current vanishes and one can write 
\begin{align}
\b{j} = \frac{j_\parallel}{B}\b{B},
\end{align}
where, as we just showed, $j_\parallel/B$ is constant on flux surfaces. In this case, we can express
\begin{align}
I(r) &= \frac{1}{2\pi}\int_0^r \,V'\frac{j_\parallel}{B} \langle \b{B}\cdot\nabla \varphi \rangle \, \rd r, \nonumber \\
&=\frac{1}{2\pi}\int_0^r \,V'\frac{j_\parallel}{B}G(r) \left\langle \frac{1}{R^2}\right\rangle \rd r .
\end{align}
Alternatively, by inserting the definition of the local current density, we obtain
\begin{align}
I(r) &= \int \rd r  \rd p \rd \xi_0  \rd\theta\rd\varphi\,\sqrt{g}e v f  \frac{G}{B}\frac{\xi}{R^2} \nonumber \\
&= \int \rd r \rd p \rd \xi_0 \,\mathcal{V}' ev f G \left\{ \frac{\xi}{B R^2}\right\}.
\end{align}
We can thus identify the (essentially) coordinate invariant expression
\begin{align}
\frac{j_\parallel}{B} = \frac{2\pi e}{\left\langle\frac{1}{R^2}\right\rangle}\int \rd p \rd \xi_0 \frac{\mathcal{V}'}{V'} vf \left\{ \frac{\xi}{B R^2}\right\}.
\end{align}



%\section{Magnetic geometry, flux surface averages and bounce average}
%
%We will describe flux surfaces as $\b{x} = \b{x}(r,\,\theta,\,\varphi)$ where $r$ is a radius-like flux surface label, $\theta$ a poloidal-like angle and $\varphi$ is the toroidal angle. For example, if we introduce a Cartesian coordinate system $(x,\,y,\,z)$ where the $z$-axis is the symmetry axis and $\varphi = \text{atan}(y/x)$ denotes the toroidal angle, we can introduce the major radius $R = \sqrt{x^2+y^2}$ with unit vector $\hat{R} = \cos\varphi \hat{x} + \sin\varphi \hat{y}$. A flux surface with arbitrary major radius $R_m$, elongation $\kappa(r)$, Shafranov shift $\Delta(r)$ and triangularity $\delta(r)$ can be parametrized via
%\begin{align}
%\b{x} &= R\hat{R} + z\hat{z}, \nonumber \\
%R &= R_m + \Delta(r) + r[\cos\theta -\delta(r) \sin^2\theta], \nonumber \\
%z &= r \kappa(r) \sin\theta.
%\label{eq:geometry coordinates}
%\end{align}
%
%An axisymmetric magnetic field can generally be represented by
%\begin{align}
%\b{B} = G(\psi) \nabla \varphi + \nabla \varphi \times \nabla \psi,
%\end{align}
%where $\psi = \psi(t,\,r)$ is the poloidal flux and $G$ describes the toroidal magnetic field (where $|\nabla \varphi | = 1/R$). The poloidal contravariant component of the magnetic field $B^\theta = \nabla\theta \cdot \b{B}$is given by
%\begin{align}
%B^\theta = \nabla \theta \cdot \b{B} = \nabla\varphi\cdot(\nabla \psi \times \nabla \theta) = \frac{\partial \psi}{\partial r} \nabla \varphi\cdot(\nabla r \times \nabla \theta) = \frac{1}{\mathcal{J}} \frac{\partial \psi}{\partial r},
%\end{align}
%where $\mathcal{J}$ is the Jacobian (see below).
%
%\subsection{Jacobian and $V'$}
%
%The Jacobian for the coordinates $(r,\,\theta,\,\varphi)$ is given by
%\begin{align}
%\mathcal{J} &= \frac{1}{|\nabla \varphi \cdot( \nabla r \times \nabla \theta)|} = \left| \frac{\partial \b{x}}{\partial \varphi}\cdot \left(\frac{\partial \b{x}}{\partial r}\times\frac{\partial \b{x}}{\partial \theta}\right)\right|,
%\end{align}
%where we can evaluate
%\begin{align}
%\frac{\partial \b{x}}{\partial \varphi} &= R\frac{\partial \hat{R}}{\partial \varphi} = R\hat{\varphi}, \nonumber \\
%\frac{\partial \b{x}}{\partial \theta} &= -r\sin\theta(1+2\delta \cos\theta) \hat{R} + r\kappa\cos\theta \hat{z} \nonumber \\
%\frac{\partial \b{x}}{\partial r} &=(\cos\theta + \Delta' -\delta\sin^2\theta - r\delta'\sin^2\theta)\hat{R} + \sin\theta(\kappa + r\kappa')\hat{z}.
%\end{align}
%The Jacobian then takes the form
%\begin{align}
%\mathcal{J} &= \left| \begin{matrix}
%%\hat{R} & \hat{\varphi}  & \hat{z}  \\
%0 & R & 0 \\
%-r\sin\theta(1+2\delta\cos\theta) & 0 & r\kappa\cos\theta \\
%(\cos\theta + \Delta' - \sin^2\theta(\delta+ r\delta')\hat{R}  & 0 & \sin\theta(\kappa + r\kappa')
%%0 & R & 
%\end{matrix}\right| \nonumber \\
%&= R\Big[ r\sin^2\theta(1+2\delta\cos\theta)(\kappa+r\kappa') + r\kappa\cos\theta(\cos\theta + \Delta' - \sin^2\theta(\delta+r\delta')\Big] \nonumber \\
%&= \kappa rR \left[ 1+r \frac{\kappa'}{\kappa}\sin^2 \theta  + 2\delta \cos\theta\sin^2\theta \left(1+r\frac{\kappa'}{\kappa}\right)  + \cos\theta(\Delta'-\sin^2\theta(\delta + r\delta') \right] \nonumber \\
%&=  \kappa rR \left[ 1+\frac{r\kappa'}{\kappa}\sin^2 \theta + \Delta'\cos\theta + \delta\cos\theta\sin^2\theta \left(1+2\frac{r\kappa'}{\kappa} - \frac{r\delta'}{\delta}\right)\right]%  - \cos\theta \sin^2\theta(\delta + r\delta') \right] \nonumber \\
%\end{align}
%
%In the disruption model, the volume $V(r) = \int_0^r \rd r \int_0^{2\pi}\rd \theta \int_0^{2\pi} \rd \varphi \, \mathcal{J} $ enclosed within the flux surface labeled $r$, and in particular its derivative $V' = \partial V/\partial r$ will become important. This is given by
%\begin{align}
%V'(r) &= \int_0^{2\pi}\rd \theta \int_0^{2\pi} \rd \varphi \, \mathcal{J} \nonumber \\
%&= 2\pi \kappa r  \int_0^{2\pi}\rd \theta  \,R\left[ 1+\frac{r\kappa'}{\kappa}\sin^2 \theta + \Delta'\cos\theta + \delta\cos\theta\sin^2\theta \left(1+2\frac{r\kappa'}{\kappa} - \frac{r\delta'}{\delta}\right)\right] %\nonumber \\
%%&= (2\pi)^2 \kappa rR \left(1+\frac{1}{2}\frac{r\kappa'}{\kappa}\right).
%\end{align}
%
%\subsubsection*{Triangularity model with $\cos(\theta+\delta\sin\theta)$}
%A different way of including triangularity is via the parametrization
%\begin{align}
%R &= R_m + \Delta(r) + r\cos[\theta +\delta(r) \sin\theta], \nonumber \\
%z &= r \kappa(r) \sin\theta.
%\end{align}
%With this choice, in a completely analogous manner, the Jacobian can be evaluated as
%\begin{align}
%%\mathcal{J} &= \kappa r R \Biggl\{\cos(\delta\sin\theta) + \Delta'\cos\theta + r\frac{\kappa'}{\kappa}\sin\theta\sin(\theta+\delta\sin\theta) \nonumber \\
%%&+ \delta\cos\theta\sin\theta\sin(\theta+\delta\sin\theta)\left[1+r\left(\frac{\kappa'}{\kappa}-\frac{\delta'}{\delta}\right) \right] \Biggr\}.
%\hspace{-10mm} \mathcal{J} &= \kappa r R \Biggl\{\cos(\delta\sin\theta) + \Delta'\cos\theta +\sin\theta\sin(\theta+\delta\sin\theta) \left[ r\frac{\kappa'}{\kappa} + \delta\cos\theta \left(1+r\frac{\kappa'}{\kappa}-\frac{\delta'}{\delta}\right)\right]\Biggr\}.
%%+ \delta\cos\theta\sin\theta\sin(\theta+\delta\sin\theta)\left[1+r\left(\frac{\kappa'}{\kappa}-\frac{\delta'}{\delta}\right) \right] \Biggr\}.
%\end{align}
%Then, $V'$ must be modified accordingly. This triangularity model is probably more well established in the literature, and should preferentially be used when triangularity effects are being investigated.
%
%
%\subsection{Flux surface average $\langle X \rangle$ and calculation of $\langle |\nabla r|^2 \rangle$ and $\langle |\nabla r |^2 / R^2 \rangle$}
%Throughout these notes, we will define the flux-surface integral $\langle X \rangle$ as 
%\begin{align}
%\langle X \rangle(t,\,r) = \int_0^{2\pi}\rd \theta \int_0^{2\pi} \rd \varphi \, \mathcal{J} X(t,\,r,\,\theta,\,\varphi),
%\end{align}
%which is not really an average at all since I choose not to divide by the total (differential) volume $V' = \langle 1 \rangle$ of the flux surface. The dimension $[\langle X \rangle]$ is $[X]\text{m}^2$.
%
%In the calculation of the divergence of a flux, we need to evaluate $\langle |\nabla r|^2 \rangle$ as well as $\langle |\nabla r|^2/R^2\rangle$. In order to do this we need to calculate $\nabla r$, which  we can do by inverting equations (\ref{eq:geometry coordinates}) for $r$:
%\begin{align}
%\sin\theta &= \frac{z}{r\kappa} \nonumber \\
%%\cos\theta &= \sqrt{1-\frac{z^2}{r^2\kappa^2}} \nonumber \\
%R = \sqrt{x^2+y^2} &= R_m + \Delta + r[\cos\theta-\delta\sin^2\theta] %\nonumber \\
%%&= R_m + \Delta + r\left[ \sqrt{1-\frac{z^2}{r^2\kappa^2}} - \delta\frac{z^2}{r^2\kappa^2}\right].
%\end{align}
%Taking the gradient of these two equations yields
%\begin{align}
%\nabla \theta &= \frac{1}{\cos\theta} \left( \frac{\nabla z}{r\kappa} - \frac{z}{r^2\kappa}\nabla r - \frac{z\kappa'}{r\kappa^2}\nabla r\right)\nonumber \\
%&= \frac{1}{\kappa r^2\cos\theta}\left[r\nabla z -z \left( 1+\frac{r\kappa'}{\kappa}\right)\nabla r\right] \nonumber \\
%\frac{x\nabla x + y\nabla y}{R} &= \Delta' \nabla r + \nabla r (\cos\theta-\delta\sin^2\theta)  \nonumber \\
%&-r(\sin\theta+2\delta\cos\theta\sin\theta)\nabla \theta - r\delta' \nabla r \sin^2\theta \nonumber \\
%&=\Big(\Delta' + \cos\theta - \delta\sin^2\theta - r\delta' \sin^2\theta\Big) \nabla r - r\sin\theta(1+2\delta\cos\theta)\nabla \theta \nonumber \\
%&= \left[\Delta' + \cos\theta - \delta\sin^2\theta - r\delta' \sin^2\theta + \frac{\sin^2\theta}{\cos\theta}(1+2\delta\cos\theta)\left(1+\frac{r\kappa'}{\kappa}\right) \right]\nabla r \nonumber \\
%& - \sin\theta(1+2\delta\cos\theta)\frac{\nabla z}{\kappa \cos\theta},
%\end{align}
%or 
%\begin{align}
%&\left[1 + \Delta'\cos\theta +\sin^2\theta\frac{r\kappa'}{\kappa}+\delta\cos\theta\sin^2\theta \left(1-\frac{ r\delta'}{\delta}+ 2\frac{r\kappa'}{\kappa}\right)\right]\nabla r \nonumber \\
%&= \cos\theta\frac{x\nabla x + y\nabla y}{R} + \frac{1}{\kappa}\sin\theta(1+2\delta\cos\theta)\nabla z ,
%\end{align}
%or, finally,
%\begin{align}
%|\nabla r|^2 &= \frac{\cos^2\theta + \frac{(1+2\delta\cos\theta)^2}{\kappa^2}\sin^2\theta}{\left[1 + \Delta'\cos\theta +\sin^2\theta\frac{r\kappa'}{\kappa}+\delta\cos\theta\sin^2\theta \left(1+ 2\frac{r\kappa'}{\kappa}-\frac{ r\delta'}{\delta}\right)\right]^2} \nonumber \\
%&= \frac{\kappa^2 r^2 R^2}{\mathcal{J}^2} \left(\cos^2\theta + \frac{(1+2\delta\cos\theta)^2}{\kappa^2}\sin^2\theta\right).
%\end{align}
%Then, we need to determine
%\begin{align}
%\langle |\nabla r |^2 \rangle &= 2\pi \int_0^{2\pi} \rd \theta \,\mathcal{J}|\nabla r|^2 \nonumber \\
%&= 2\pi \kappa^2 r^2  \int_0^{2\pi} \rd \theta \, R^2\frac{\cos^2\theta + \frac{(1+2\delta\cos\theta)^2}{\kappa^2}\sin^2\theta}{\mathcal{J}}
%%R=R_m+\Delta(r) + r[\cos\theta -\delta(r) \sin^2\theta]
%\end{align}
%and analogously $\langle |\nabla r|^2 /R^2\rangle$, which do not have analytical closed-form expressions in general, and must be evaluated numerically unless $\delta = 0$ and $\kappa' = 0$.
%
%
%\subsubsection*{Triangularity model with $\cos(\theta+\delta\sin\theta)$}
%For the other triangularity model, we get
%\begin{align}
%|\nabla r|^2 = \frac{R^2}{\mathcal{J}^2}\left[ \left(\frac{\partial z}{\partial \theta}\right)^2 + \left(\frac{\partial R}{\partial \theta}\right)^2\right]
%\end{align}
%(which is also a handy formula for an arbitrary parametrization), which yields
%\begin{align}
%|\nabla r|^2 = \frac{\kappa^2 r^2R^2}{\mathcal{J}^2}\Big[  \cos^2\theta + \frac{1}{\kappa^2}(1+\delta\cos\theta)^2\sin^2(\theta+\delta\sin\theta)\Big]
%\end{align}
%
%
%
%
%\subsection{Bounce average $\{X\}$ in the zero-orbit-width limit}
%We define the bounce average $\{X\}$ as the integral of the quantity $X$ (assumed to be axisymmetric) in time along a closed guiding center trajectory (a full poloidal orbit) 
%\begin{align}
%\{X\}(t,\,r,\,E,\,\mu) &= \oint \rd \tau \, X(t,\,r\,\theta(\tau),\,E,\,\mu) \nonumber \\
%&= \oint \frac{\rd \theta}{\rd \theta/\rd \tau} X = \oint \frac{\rd \theta}{\nabla \theta \cdot \b{v}\sub{gc}} X,
%\end{align}
%where $E$ is the energy and $\mu$ the magnetic moment of the guiding center, assumed to be conserved along the orbit. In the zero-orbit-width limit 
%\begin{align}
%\b{v}\sub{gc} = v_\parallel \frac{\b{B}}{B},
%\end{align}
%we obtain
%\begin{align}
%\{X\} &= \oint \rd \theta \frac{B}{v_\parallel} \frac{1}{\nabla\theta\cdot\b{B}} X \nonumber \\
%&= \frac{1}{\partial \psi/\partial r} \oint \rd \theta \,\mathcal{J} \frac{B}{v_\parallel}X.
%\end{align}
%The orbit along which we integrate is different for passing and trapped particles; it is given by
%\begin{align}
%\oint \rd \theta &= 
%\begin{cases}
%\int_0^{2\pi} \rd \theta , & \text{passing,} \\
%\sum_\sigma \int_{-\theta\sub{bounce}}^{\theta\sub{bounce}} \rd\theta \,\sigma , & \text{trapped}.
%\end{cases} \\
%\theta\sub{bounce}: & \quad v_\parallel(\theta\sub{bounce})= 0, \nonumber
%\end{align}
%where $v_\parallel(\theta)$ is determined by the conservation of magnetic moment and energy,
%\begin{align}
%E &= \frac{p_\parallel^2+p_\perp^2}{2m} \\
%\mu &= \frac{p_\perp^2}{2mB},
%\end{align}
%and we can see that the bounce integral for a trapped orbit will annihilate any function that is even in $\sigma$.
%%For a passing particle, the integral is taken from $0$ to $2\pi$ and we find
%%\begin{align}
%%\{ X \}\sub{passing} &= \frac{1}{2\pi \partial \psi/\partial r}\left \langle  \frac{B}{v_\parallel} X \right\rangle,
%%\end{align}
%%whereas for a passing particle, the integral is taken from $-\theta\sub{bounce}$ to $\theta\sub{bounce}$ and summed over the signs of $v_\parallel = v\xi = \sigma v |\xi|$. If we use $v$, $|\xi|$ and $\sigma$ as variables for our momentum space, we should sum over $\sigma$ in the integral and obtain
%%\begin{align}
%%\{ X \}\sub{trapped} = \frac{1}{\partial \psi/\partial r}\int_{-\theta\sub{bounce}}^{\theta\sub{bounce}} \rd \theta \,\mathcal{J}\frac{B}{v|\xi|} (X^++X^-),
%%\end{align}
%%where $X^+ = X(\sigma=+1)$ and $X^- = X(\sigma=-1)$, and we see explicitly that $\{ X \}\sub{trapped}$ is even in $\sigma$; the bounce average will annihilate any function that is odd in $\sigma$.
%


\section{Kinetic equation}
We study the kinetic equation (ignoring the energy-diffusion term, strictly valid in the superthermal limit) which can be given in local ($r,\,p,\,\xi$) coordinates as 
\begin{align}
\hspace{-1mm}\frac{\partial f}{\partial t} + eE\left[\frac{1}{p^2}\frac{\partial}{\partial p}(p^2\xi f)  + \frac{1}{p}\frac{\partial}{\partial \xi}[(1-\xi^2) f]\right] &= \frac{1}{p^2}\frac{\partial}{\partial p}\Big(p^3 \nu_s  f\Big) + \frac{\nu_D}{2}\frac{\partial}{\partial \xi}\left[(1-\xi^2)\frac{\partial f}{\partial \xi}\right] \nonumber \\
&\hspace{-15mm} + \frac{1}{\sqrt{g}}\frac{\partial}{\partial r}\left[ \sqrt{g}\left( -Af+D\frac{\partial f}{\partial r}\right)\right] + S,
\end{align}
and a spatial $\partial/\partial \theta$ term that we will kill with the bounce average in a second.
Here, $f=f(t,\,p,\,\xi)$ is the distribution function, $\xi = \b{p}\cdot\b{b}/p$ denotes the pitch, $\nu_s$ is the slowing-down frequency, $\nu_D$ the deflection frequency. We use the model [L~Hesslow \emph{et al.} JPP 84 (2018)]
\begin{align}
\nu_s &= 4\pi c r_0^2 \frac{\gamma^2}{p^3} \left\{n\sub{cold} \ln\Lambda^\text{ee} + \sum_j n_j N_{e,j}\left[\frac{1}{k}\ln(1+ h_j^k)  - \frac{p^2}{\gamma^2}\right]\right\} , \nonumber \\
\nu_D &= 4\pi c r_0^2\frac{\gamma}{p^3} \left( n\sub{cold} Z\sub{eff} \ln\Lambda^\text{ei} + \sum_j n_j g_j(p) \right), \nonumber \\
g_j &= \frac{2}{3} (Z_j^2-Z_{0j}^2)\ln[1+ (\bar{a}_jp)^{3/2}] - \frac{2}{3}N_{ej}^2\frac{(\bar{a}_j p)^{3/2}}{1+(\bar{a}_j p)^{3/2}}, \nonumber \\
h_j &= \frac{m_e c^2}{I_j}p\sqrt{\gamma-1}\nonumber \\
N_{e,j} &= Z_j-Z_{0j}, \nonumber \\
Z\sub{eff} &= \sum_j n_j Z_{0j}^2/n\sub{cold}  \nonumber \\
\ln\Lambda^{ee} &= \ln\Lambda_c + \ln\sqrt{\gamma-1}, \nonumber \\
\ln\Lambda^{ei} &= \ln\Lambda_c + \ln(\sqrt{2}p), 
\end{align}
where $Z_j$ is the atomic number and $Z_{0j}$ the charge number of particle species $j$, where $n_j$ denotes the corresponding number density, $I_j$ is an ionic mean stopping power (tabulated), $\bar{a}_j$ an ion-specific parameter (tabulated) and $n\sub{cold}$ the number density of free cold electrons. Momenta $p$ are normalized to $m_e c$, and $\gamma = \sqrt{1+p^2}$ is the Lorentz factor. We assume that the background density is uniformly distributed on flux surfaces.

The transport model lacks a first-principle description, but we shall generally assume that the transport is dominated by losses along stochastic field lines, making it proportional to the parallel speed:
\begin{align}
A &= A_0(t,\,r)v |\xi|, \nonumber \\
D &= D_0(t,\,r) v|\xi|.
\end{align}


\subsection{Common coordinate systems}
We shall only describe spatial dynamics using the coordinates $(r,\,\theta,\,\varphi)$ with Jacobian $\mathcal{J}$. For momentum space, there are multiple choices that can be useful. The local variables $(p,\,\xi)$ with Jacobian $p^2$ provide a natural description of collisions. Local variables $(p_\parallel,\,p_\perp)$ with Jacobian $p_\perp$ are natural to describe electric-field acceleration, which is dominant during the current quench. For the bounce averaged equation, we must choose coordinates that are functions of the orbit-constants-of-motion $(p,\,\lambda,\,\sigma)$ with Jacobian $p^2B/(2\sqrt{1-\lambda B})$. Analogues to the above can be constructed by considering \emph{the value at the minimum magnetic field}, i.e. $(p,\,\xi_0 = \xi(B=B\sub{min}))$ or $(p_{\parallel,0} = p_\parallel(B=B\sub{min}),\,p_{\perp,0} = p_\perp(B=B\sub{min}))$. We can find explicit expressions for these by noting
\begin{align}
\lambda &= \frac{1-\xi^2}{B}, \nonumber \\
\xi &= \sigma \sqrt{1-\lambda B}, \nonumber \\
&= \frac{p_\parallel}{p}, \nonumber\\
\xi_0 &= \sigma \sqrt{1-\lambda B\sub{min}}\nonumber \\
&= \sigma \sqrt{1-\frac{B\sub{min}}{B}(1-\xi^2)}, \nonumber \\
p_{\parallel,0} &= p\xi_0, \nonumber \\
p_{\perp,0} &= p\sqrt{1-\xi_0^2}.
\end{align}
In calculating the contravariant components $A^i = \nabla z^i \cdot \b{A}$, it is useful to give (using the notation $\nabla_\b{p} = \partial/\partial \b{p}$)
\begin{align}
\nabla_\b{p}\xi  &= \frac{\partial \xi}{\partial p_\parallel}\hat{p_\parallel} + \frac{\partial \xi}{\partial p_\perp}\hat{p_\perp} = \left\{\xi = p_\parallel/\sqrt{p_\parallel^2+p_\perp^2}\right\}\nonumber \\
& = \left(\frac{1}{p}-\frac{p_\parallel^2}{p^3}\right)\hat{p}_\parallel - \frac{p_\parallel p_\perp}{p^3} \hat{p}_\perp \nonumber \\
&= \frac{\sqrt{1-\xi^2}}{p}\left(\sqrt{1-\xi^2}\hat{p}_\parallel - \xi \hat{p}_\perp\right), \nonumber \\
\nabla_\b{p} \lambda &= -\frac{2\xi}{B}\nabla_\b{p}\xi, \nonumber \\
 \nabla_\b{p}\xi_0&=\frac{B\sub{min}}{B}\frac{\xi}{\xi_0} \nabla_\b{p}\xi, \nonumber \\
\nabla_\b{p} p_{\parallel,0} &= \xi_0 \nabla_\b{p} p + p \nabla_\b{p}\xi_0 \nonumber \\
&=\xi_0\hat{p} +\frac{B\sub{min}}{B}p\frac{\xi}{\xi_0}\nabla_\b{p}\xi, \nonumber \\
%&=\frac{1}{\xi_0}\left(1-\frac{B\sub{min}}{B}\right)\hat{p} + \frac{B\sub{min}}{B}\frac{\xi}{\xi_0}\hat{p}_\parallel \nonumber \\
&= \frac{\xi}{\xi_0}\hat{p}_\parallel + \frac{\sqrt{1-\xi^2}}{\xi_0}\left(1-\frac{B\sub{min}}{B}\right)\hat{p}_\perp, \nonumber \\
\nabla_\b{p}p_{\perp,0} &= \sqrt{1-\xi_0^2}\hat{p} - \frac{B\sub{min}}{B}\frac{p \xi}{\sqrt{1-\xi_0^2}}  \nabla_\b{p}\xi \nonumber \\
&= \sqrt{\frac{B\sub{min}}{B}} \hat{p}_\perp
\end{align}

For the various coordinate systems, the table summarises expressions for the Jacobian $\sqrt{g}$ as well as $\xi$, which often naturally appears in the bounce-average coefficients.
In the following subsections, we will explicitly carry out the bounce average in the $(p,\,\lambda)$ system in order to illustrate how it works out. 


\begin{table}
\begin{center}
\caption{Properties of three useful momentum-space coordinate systems.}
\begin{tabular}{|c|c|c|}\hline 
Coordinate system & $\sqrt{g}/\mathcal{J}$ & $\xi$ \\ \hline
$p$, $\lambda$ & $\frac{p^2 B}{2\xi}$ & $\sigma\sqrt{1-\lambda B}$ \\\hline
$p$, $\xi_0$ & $ p^2 \frac{B}{B\sub{min}}\frac{\xi_0}{\xi}$ & $\sigma\sqrt{1-\frac{B}{B\sub{min}}(1-\xi_0^2)}$ \\\hline
$p_{\parallel,0}$, $p_{\perp,0}$ & $\frac{B}{B\sub{min}} \frac{p_{\perp,0}}{\sqrt{p_{\parallel,0}^2+p_{\perp,0}^2}}\frac{p_{\parallel,0}}{\xi}$ & $\sigma\sqrt{1-\frac{B}{B\sub{min}}\frac{p_{\perp,0}^2}{p_{\parallel,0}^2+p_{\perp,0}^2}}$\\
\hline
\end{tabular}
\end{center}
\end{table}

\subsection{Change of variables to magnetic moment $\lambda$}
To facilitate the bounce average we change pitch variable from $\xi$ to $\lambda$
\begin{align}
\lambda &= \frac{1-\xi^2}{B}, \nonumber \\
\xi &= \sigma\sqrt{1-\lambda B}, \nonumber \\
\frac{\partial}{\partial \xi} &= -\frac{2\xi}{B}\frac{\partial}{\partial \lambda} \nonumber \\
&= -\sigma\frac{\sqrt{1-\lambda B}}{B}\frac{\partial}{\partial \lambda} .
\end{align}
The kinetic equation then takes the form
\begin{align}
\frac{\partial f}{\partial t} + eE \sigma\sqrt{1-\lambda B}\left[\frac{1}{p^2}\frac{\partial p^2 f}{\partial p}  - \frac{2}{p}\frac{\partial \lambda f}{\partial \lambda}\right] &= \frac{1}{p^2}\frac{\partial}{\partial p}\Big(p^3 \nu_s  f\Big) + 2\nu_D\frac{\sqrt{1-\lambda B}}{B}\frac{\partial}{\partial \lambda}\left[\lambda \sqrt{1-\lambda B}\frac{\partial f}{\partial \lambda}\right] \nonumber \\
&\hspace{-31mm} + \frac{1}{\sqrt{g}}\frac{\partial}{\partial r}\left[ \sqrt{g}\left( -Af+D\frac{\partial f}{\partial r}\right)\right] + S(t,\,r,\,\theta,\,p,\,\xi).
\end{align}
 
\subsection{Bounce average of the kinetic equation}
In this section we give explicit expressions for each of the terms appearing in the bounce averaged kinetic equation.

\subsubsection*{Electric field term}
The bounce average of the electric-field term yields
\begin{align}
\left\{eE \sigma\sqrt{1-\lambda B}\left[\frac{1}{p^2}\frac{\partial p^2 f}{\partial p}  - \frac{2}{p}\frac{\partial \lambda f}{\partial \lambda}\right]  \right\} = \left\{eE \sigma\sqrt{1-\lambda B}\right\}\left[\frac{1}{p^2}\frac{\partial p^2 f}{\partial p}  - \frac{2}{p}\frac{\partial \lambda f}{\partial \lambda}\right].
\end{align}
Since the bounce average contains a $\sigma$, the contribution will vanish identically in the trapped region. For the remaining contribution in the passing region, we can relate the bounce average to the flux-surface average as
\begin{align}
\left\{eE \sigma\sqrt{1-\lambda B}\right\} &= \sigma \Theta(\lambda) \frac{p^2}{2}\frac{V'}{\mathcal{V}'}e\left\langle EB\right\rangle \nonumber \\
&= \sigma \Theta(\lambda) \frac{e \left\langle \b{E}\cdot\b{B}\right\rangle}{\langle B/\sqrt{1-\lambda B} \rangle}, \nonumber \\
\Theta &= \begin{cases}
1 & \text{passing} ~(\lambda \leq \lambda_T) \\
0 & \text{trapped} ~(\lambda > \lambda_T)
\end{cases}.
\end{align}
We may further express this in terms of the loop voltage via
\begin{align}
\langle \b{E}\cdot\b{B} \rangle =  \left\langle \b{B}\cdot\nabla \varphi \right\rangle \frac{V\sub{loop}}{2\pi} = G \left\langle\frac{1}{R^2}\right\rangle \frac{V\sub{loop}}{2\pi},
\end{align}
yielding the bounce averaged electric-field term
\begin{align}
 \left\{eE \sigma\sqrt{1-\lambda B}\right\}\left[\frac{1}{p^2}\frac{\partial p^2 f}{\partial p}  - \frac{2}{p}\frac{\partial \lambda f}{\partial \lambda}\right] &= \sigma \Theta \frac{e \left\langle \b{E}\cdot\b{B}\right\rangle}{\langle B/\sqrt{1-\lambda B} \rangle}\left[\frac{1}{p^2}\frac{\partial p^2 f}{\partial p}  - \frac{2}{p}\frac{\partial \lambda f}{\partial \lambda}\right]  \nonumber \\
&=\frac{\sigma \Theta G \left\langle\frac{1}{R^2}\right\rangle}{\langle B/\sqrt{1-\lambda B}\rangle} \frac{eV\sub{loop}}{2\pi}  \left[\frac{1}{p^2}\frac{\partial p^2 f}{\partial p}  - \frac{2}{p}\frac{\partial \lambda f}{\partial \lambda}\right] .
\end{align}


\subsubsection*{Pitch-angle scattering term}
The pitch-angle term is treated analogously, but noting that we may pass the bounce average inside the $\lambda$ derivatives despite the fact that the integration limits $\theta\sub{bounce}$ depend on $\lambda$. However, the function that is being differentiated vanishes at the bounce points due to the term $\sqrt{1-\lambda B}$, allowing this procedure. We write
\begin{align}
\left\{2\nu_D\frac{\sqrt{1-\lambda B}}{B}\frac{\partial}{\partial \lambda}\left[\lambda \sqrt{1-\lambda B}\frac{\partial f}{\partial \lambda}\right] \right\} = \frac{2\nu_D}{\mathcal{V}'}\frac{\partial}{\partial \lambda}\left(  \mathcal{V'}\left\{\frac{1-\lambda B}{B}\right\} \lambda\frac{\partial f}{\partial \lambda}\right).
\end{align}
The average can be given explicitly as
\begin{align}
\mathcal{V'}\left\{\frac{1-\lambda B}{B}\right\} &= \int_0^{2\pi}\rd \varphi \oint \rd \theta \, \mathcal{J}\sqrt{1-\lambda B}\nonumber \\
&=\begin{cases}
V' \langle \sqrt{1-\lambda B} \rangle, & \text{passing}~(\lambda \leq \lambda_T) \\
2\int_0^{2\pi}\rd\varphi\int_{-\theta\sub{bounce}}^{\theta\sub{bounce}} \rd\theta \,\mathcal{J}\sqrt{1-\lambda B}, & \text{trapped} ~(\lambda > \lambda_T)
\end{cases}.
\end{align}

\subsubsection*{Time derivative and slowing down term}
These terms are constant across the flux surface, and yield
\begin{align}
\left\{ \frac{\partial f}{\partial t}\right\} &= \frac{\partial f}{\partial t} , \nonumber \\
\left\{\frac{1}{p^2}\frac{\partial p^3\nu_s f }{\partial p}\right\} &= \frac{1}{p^2}\frac{\partial p^3\nu_s f }{\partial p}.
\end{align}

\subsubsection*{Transport term}
In a previous section, we gave the general flux-surface averaged transport term as
\begin{align}
\left\{\frac{1}{\sqrt{g}}\frac{\partial}{\partial r}\left[ \sqrt{g}\left( -Af+D\frac{\partial f}{\partial r}\right)\right]\right\} = \frac{1}{\mathcal{V}'}\frac{\partial}{\partial r}\left[ \mathcal{V'}\left(-\{A\}f+ \{D\}\frac{\partial f}{\partial r} \right) \right].
\end{align}
For our proposed transport describing parallel losses along open magnetic field lines, we have
\begin{align}
A &= A_0 v |\xi| = A_0(t,\,r) v \sqrt{1-\lambda B}
\end{align}
and likewise for $D$. The bounce average of this yields
\begin{align}
\{A\} = A_0 v \{\sqrt{1-\lambda B}\},
\end{align}
which for passing particles is given by  $A_0 v \langle B \rangle/\langle B/\sqrt{1-\lambda B}\rangle$.


\subsection{Full bounce-averaged kinetic equation}
We can write all the terms out explicitly, which yields the equation
\begin{align}
\frac{\partial f}{\partial t} &+ \sigma \Theta \frac{e \left\langle \b{E}\cdot\b{B}\right\rangle}{\langle B/\sqrt{1-\lambda B} \rangle}\left[\frac{1}{p^2}\frac{\partial p^2 f}{\partial p}  - \frac{2}{p}\frac{\partial \lambda f}{\partial \lambda}\right]  = \frac{1}{p^2}\frac{\partial(p^3 \nu_s f)}{\partial p} \nonumber \\
&\hspace{-15mm}+ \frac{2\nu_D}{\mathcal{V}'}\frac{\partial}{\partial \lambda}\left(  \mathcal{V'}\left\{\frac{1-\lambda B}{B}\right\} \lambda\frac{\partial f}{\partial \lambda}\right)+\frac{1}{\mathcal{V}'}\frac{\partial}{\partial r}\left[ \mathcal{V'}\left(-\{A\}f+ \{D\}\frac{\partial f}{\partial r} \right) \right] + \{S\},
\label{eq:full bounce averaged}
\end{align}
with the definitions given in the previous subsection.


\subsection{Bounce-averaged effective critical field}
Hesslow ecritpaper considers the kinetic equation with the addition of bremsstrahlung and synchrotron radiation losses
\begin{align}
\frac{\partial f}{\partial t} &= \frac{1}{p^2}\frac{\partial}{\partial p}\left[ p^2\left(-\xi eE + p\nu_s + F\sub{br}+\frac{p\gamma}{\tau_s}(1-\xi^2)\right)f\right] \nonumber \\
&+\frac{\partial}{\partial \xi}\left[ (1-\xi^2)\left( -\frac{1}{p}eE + \frac{\nu_D}{2}\frac{\partial}{\partial \xi} - \frac{\xi}{\tau_s \gamma} \right)f\right].
\end{align}
In terms of the notation in these notes, in $(p,\,\xi_0)$ coordinates the equation can be written
\begin{align}
\frac{\partial f}{\partial t} &= -\frac{1}{\mathcal{V}'}\frac{\partial}{\partial p}\left(\mathcal{V'} \{A^p\}f\right) + \frac{1}{\mathcal{V}'}\frac{\partial}{\partial \xi_0}\left[\mathcal{V'}\left( -\{A^{\xi_0}\}f + \{D^{\xi_0\xi_0}\} \frac{\partial f}{\partial \xi_0}\right)\right],
\end{align}
where for relativistic energies, the $pp$-diffusion is negligible. Assuming that near the effective critical field the momentum flux is much smaller than the characteristic pitch flux, $\{A^p\} \ll \{A^{\xi_ 0}\} \sim \{D^{\xi_0 \xi_0}\}$, we find an equilibrium solution
\begin{align}
0 &= -\{A^{\xi_0}\}f + \{D^{\xi_0\xi_0}\} \frac{\partial f}{\partial \xi_0} \nonumber \\
f &= F(p) \frac{\exp\left( -\int_{\xi_0}^1 \frac{\{A^{\xi_0}\}}{\{D^{\xi_0\xi_0}\}} \rd \xi_0' \right)}{\int_{-1}^1 \rd \xi_0 \frac{\mathcal{V}'}{V'}\exp\left( -\int_{\xi_0}^1 \frac{\{A^{\xi_0}\}}{\{D^{\xi_0\xi_0}\}} \rd \xi_0' \right) },
\end{align}
where we have normalized the momentum distribution $F$ such that the flux-surface averaged runaway density is given by $\langle n\sub{RE} \rangle = \int F(p) \,\rd p$.
Integrating the kinetic equation over $\rd \xi_0 \mathcal{V'}$ yields
\begin{align}
0 &= V' \frac{\partial}{\partial p} \left[ F(p)  \frac{\int_{-1}^1\rd \xi_0 \, \mathcal{V}'  \{A^p\} \exp\left( -\int_{\xi_0}^1\rd \xi_0' \,\frac{\{A^{\xi_0}\}}{\{D^{\xi_0\xi_0}\}} \right)}{\int_{-1}^1\rd \xi_0 \, \mathcal{V}' \exp\left( -\int_{\xi_0}^1\rd \xi_0' \,\frac{\{A^{\xi_0}\}}{\{D^{\xi_0\xi_0}\}} \right)}\right] \nonumber \\
&\equiv V' \frac{\partial \bigl[ F(p) U(p) \bigr]}{\partial p} 
\end{align}
We identify the integrals, which take the form of a pitch-distribution-averaged momentum advection coefficient, as the effective momentum particle flux accounting for finite pitch angles, denoted $U(p)$. The critical electric field is the minimum field for which $U(p;\,V\sub{loop})=0$ has a real positive root, and the (largest) solution $p\sub{max}$ for which $U(p\sub{max}; \,V\sub{loop}) = 0$ estimates the maximum energy that can be reached by runaways, i.e. where a bump would form in the steady-state runaway distribution tail.

We further assume that the pitch-angle distribution is dominated by the balance between electric-field acceleration and pitch-angle scattering, for which we find
\begin{align}
\frac{\{A^{\xi_0}\}}{\{D^{\xi_0\xi_0}\}} = \frac{2}{\nu_D} \frac{\xi_0}{p} \frac{\{eE\xi \}}{\left\{\frac{B\sub{min}}{B}\xi^2\right\}},
\end{align}
where it should be pointed out that this vanishes for identically for trapped particles, having $\xi_0 < \xi\sub{T}$, with the trapped-passing boundary occuring at
\begin{align}
\xi_T(r) = \sqrt{1-B\sub{min}/B\sub{max}}.
\end{align}



\newpage
A simpler approximate expression can be found by noting that we can split the integral into the contributions from passing particles and those of trapped particles,
\begin{align}
0 &= \int_{\xi_T}^1\rd \xi_0 \left\{  A_p   \right\} \exp \left[-\int_{\xi_0}^1\rd \xi_0' \,\frac{\{A^{\xi_0}\}}{\{D^{\xi_0\xi_0}\}} \right]\nonumber \\
&+  \exp \left[-\int_{\xi_T}^1\rd \xi_0' \,\frac{\{A^{\xi_0}\}}{\{D^{\xi_0\xi_0}\}} \right]\left\{\int_{\xi_c}^{\xi_T}\rd \xi_0 \, \bigl[ A_p(-\xi_0) + A_p(\xi_0) \bigr] \right\}\nonumber \\
& +  \exp \left[-\int_{\xi_T}^1\rd \xi_0' \,\frac{\{A^{\xi_0}\}}{\{D^{\xi_0\xi_0}\}} \right] \int_{-1}^{-\xi_T} \rd \xi_0  \, \left\{ A_p \right\} \exp \left[-\int_{\xi_0}^{-\xi_T}\rd \xi_0' \,\frac{\{A^{\xi_0}\}}{\{D^{\xi_0\xi_0}\}} \right].
\end{align}
Typically only the first of these three terms will contribute significantly; if not, the validity of the approach is anyway dubious since the assumption of negligible momentum flux only holds for $\xi_0$ near unity. In that case, we need only solve the simpler problem
\begin{align}
0 &= \int_{\xi_T}^1\rd \xi_0 \left\{ A_p   \right\} \exp \left[-\int_{\xi_0}^1\rd \xi_0' \,\frac{\{A^{\xi_0}\}}{\{D^{\xi_0\xi_0}\}} \right], \nonumber \\
&=  \int_{\xi_T}^1\rd \xi_0 \left\{ A_p   \right\} \exp \left[-\int_{\xi_0}^1\rd \xi_0' \,\frac{2}{\nu_D} \frac{\xi_0}{p} \frac{\{eE\xi \}}{\left\{\frac{B\sub{min}}{B}\xi^2\right\}} \right], \nonumber \\
&= \int_{\xi_T}^1\rd \xi_0 \left\{ A_p   \right\} \exp \left[-\frac{2e\langle \b{E}\cdot\b{B}\rangle}{B\sub{min} p \nu_D}\int_{\xi_0}^1\rd \xi_0' \,  \frac{\xi_0'}{\left\langle \xi\right\rangle} \right], \nonumber \\
&= \int_{\xi_T}^1\rd \xi_0 \left\{ A_p   \right\} \exp \left[-\frac{2e\langle \b{E}\cdot\b{B}\rangle}{B\sub{min} p \nu_D}\int_{\xi_0}^1\rd \xi_0' \,  \frac{\xi_0'}{\left\langle \sqrt{1-\frac{B}{B\sub{min}}(1-\xi_0'^2)}\right\rangle} \right], \nonumber \\
&= \int_{\xi_T}^1\rd \xi_0 \left\{ A_p   \right\}  \exp \left[-\frac{2e\langle \b{E}\cdot\b{B}\rangle}{B\sub{min} p \nu_D}(1-\xi_0)\int_0^{1} \rd x   \frac{1-(1-\xi_0)x}{\left\langle \sqrt{1-\frac{B}{B\sub{min}} (1-\xi_0)x[2-(1-\xi_0)x]}\right\rangle} \right].
\end{align}
The integral, which can be interpreted as the relative error in setting $\int_{\xi_0}^1 \rd \xi_0'/\langle \xi \rangle = 1-\xi_0$, equals 1 when $\xi_0=1$. The error is maximal at the minimum value $\xi_0 = \xi_T$. For flux surfaces where $R = R_0 (1+\epsilon \cos\theta)$ with $\epsilon=r/R$, and $B = B_0(r)/R$, the maximum deviation occurs at $\epsilon=0.318$, where the integral is 1.12, i.e. there is a maximum relative error of approximately $10\%$ caused by replacing the $\xi_0 / \langle \xi \rangle$ integral by $1-\xi_0$. The maximum error (i.e. at $\xi_T$) scales approximately as $0.426\sqrt{\epsilon}$ for small $\epsilon$ and as $0.2\sqrt{1-\epsilon}$ when $\epsilon \to 1$. By setting the integral to 1, we obtain the relatively simple condition for the critical electric field:
\begin{align}
0 = \int_{\xi_T}^1\rd \xi_0 \left\langle \frac{\xi_0}{\xi}\frac{B}{B\sub{min}} A_p   \right\rangle \exp \left[-\frac{2e\langle \b{E}\cdot\b{B}\rangle}{B\sub{min} p \nu_D}(1-\xi_0) \right] .
\end{align}
The remaining flux surface averages are given by
\begin{align}
\left\langle \frac{\xi_0}{\xi}\frac{B}{B\sub{min}} A_p   \right\rangle_E &= \xi_0\frac{e \langle\b{E}\cdot\b{B}\rangle}{B\sub{min}}, \nonumber \\
\left\langle \frac{\xi_0}{\xi}\frac{B}{B\sub{min}} A_p   \right\rangle\sub{friction} &= -[p\nu_s + F\sub{brems}(p)]\left\langle \frac{\xi_0}{\xi}\frac{B}{B\sub{min}} \right\rangle, \nonumber \\
\left\langle \frac{\xi_0}{\xi}\frac{B}{B\sub{min}} A_p   \right\rangle\sub{synchr} &= -\frac{1-\xi_0^2}{\gamma \tau_{s,\text{min}}}\left\langle \xi\frac{B^3}{B\sub{min}^3}  \right\rangle,
\end{align}
where $1/\tau_{s,\text{min}} = e^4 B\sub{min}^2/(6\pi\varepsilon_0 m_e^3 c^3)$.

% xi_T = sqrt( 1 - (1-r)/(1+r) ) = sqrt( 2r/(1+r) )
% 1-xi_T = 1- sqrt(2r/(1+r)) = (1+2r/(1+r)) / (1 + sqrt[2r/(1+r)] )
%x = (1-xi0')/(1-xi0)
%dx = -dxi0'/(1-xi0)
%xi0' = 1 -  (1-xi0) x
%1-xi0'^2 = 2(1-xi0) x - (1-xi0)^2 x^2 = (1-xi0)x(2-(1-xi0)x)
% int_xi0^1 dxi0' -> (1-xi0) int_0^1 dx

% xi0 = sqrt(1-Bmin*lambda)
% 1-xi0^2 = Bmin*lambda
%
% d xi0 = Bmin* d lambda / 2xi0
% int d lambda / <xi>
% int d lambda / <1-lambda*B>


\section{Reduced kinetic equation in strong-pitch-angle scattering limit}
We proceed to solve the equation perturbatively in the limit of strong pitch angle scattering. We order $\nu_D \sim \delta^0$, $E \sim \delta$ and $\partial/\partial t \sim \nu_s \sim A \sim D \sim S \sim \delta^2$. In that case, writing $f=f_0+\delta f_1+\delta^2 f_2 + ...$, we obtain the system of equations
\begin{align}
\frac{\partial}{\partial \lambda}\left(  \mathcal{V'}\left\{\frac{1-\lambda B}{B}\right\} \lambda\frac{\partial f_0}{\partial \lambda}\right) &= 0, \\
\sigma \Theta \frac{e \left\langle \b{E}\cdot\b{B}\right\rangle}{\langle B/\sqrt{1-\lambda B} \rangle}\left[\frac{1}{p^2}\frac{\partial p^2 f_0}{\partial p}  - \frac{2}{p}\frac{\partial \lambda f_0}{\partial \lambda}\right] &=  \frac{2\nu_D}{\mathcal{V}'}\frac{\partial}{\partial \lambda}\left(  \mathcal{V'}\left\{\frac{1-\lambda B}{B}\right\} \lambda\frac{\partial f_1}{\partial \lambda}\right) \\
%
\frac{\partial f_0}{\partial t} + \sigma \Theta \frac{e \left\langle \b{E}\cdot\b{B}\right\rangle}{\langle B/\sqrt{1-\lambda B} \rangle}\left[\frac{1}{p^2}\frac{\partial p^2 f_1}{\partial p}  - \frac{2}{p}\frac{\partial \lambda f_1}{\partial \lambda}\right]  &= \frac{1}{p^2}\frac{\partial(p^3 \nu_s f_0)}{\partial p} \nonumber \\
&\hspace{-90mm}+ \frac{2\nu_D}{\mathcal{V}'}\frac{\partial}{\partial \lambda}\left(  \mathcal{V'}\left\{\frac{1-\lambda B}{B}\right\} \lambda\frac{\partial f_2}{\partial \lambda}\right)+\frac{1}{\mathcal{V}'}\frac{\partial}{\partial r}\left[ \mathcal{V'}\left(-\{A\}f_0+ \{D\}\frac{\partial f_0}{\partial r} \right) \right] + \{S\},
\end{align}
The first equation yields the general solution
\begin{align}
f_0 = f_0(t,r,p),
\end{align}
i.e. the leading-order distribution is isotropic, upon which the second equation takes the form
\begin{align}
\sigma \Theta \frac{V'e \left\langle \b{E}\cdot\b{B}\right\rangle}{4\nu_D}p^2\frac{\partial  f_0}{\partial p} &=  \frac{\partial}{\partial \lambda}\left(  \mathcal{V'}\left\{\frac{1-\lambda B}{B}\right\} \lambda\frac{\partial f_1}{\partial \lambda}\right) 
\end{align}
If we multiply the equation by, and sum over, $\sigma$, we get
\begin{align}
\Theta  p^2\frac{V'e \left\langle \b{E}\cdot\b{B}\right\rangle}{2\nu_D}\frac{\partial  f_0}{\partial p } &=  \frac{\partial}{\partial \lambda}\left(  \mathcal{V'}\left\{\frac{1-\lambda B}{B}\right\} \lambda\frac{\partial( f_1^+ - f_1^-)}{\partial \lambda}\right), 
\end{align} 
where $f_1^{\pm} = f_1(\sigma = 0.5\pm0.5)$.
Note that the LHS only depends on $\lambda$ via $\Theta$, which is zero in the trapped region and one in the passing region. The RHS also vanishes in the trapped region, since the distribution must be even in $\sigma$ there. Integrating the equation over $\lambda$ from 0 to $\lambda$ yields
\begin{align}
\frac{\partial (f_1^+-f_1^-)}{\partial \lambda} &=  p^2\frac{V'e \left\langle \b{E}\cdot\b{B}\right\rangle}{2\nu_D}\frac{1}{\mathcal{V}' \left\{\frac{1-\lambda B}{B}\right\}}\frac{\partial f_0}{\partial p } \nonumber \\
&= \frac{e \left\langle \b{E}\cdot\b{B}\right\rangle}{\nu_D}\frac{1}{\langle \sqrt{1-\lambda B}\rangle} \frac{\partial f_0}{\partial p}.
\end{align}
Integrating a final time over $\lambda$, from $\lambda$ to the trapped-region boundary $\lambda=\lambda_T = 1/B\sub{max}$ where the LHS vanishes, yields
\begin{align}
f_1^+-f_1^- = -\int_\lambda^{\lambda_T} \frac{\rd \lambda}{\langle\sqrt{1-\lambda B}\rangle} \frac{ e \langle \b{E}\cdot\b{B}\rangle}{\nu_D}\frac{\partial f_0}{\partial p}.
\end{align}

If we now sum the third equation over $\sigma$, multiply by $\mathcal{V}'/2$ and integrate over all $\lambda$ (from 0 to $1/B\sub{min}$), we obtain
\begin{align}
\left(\int \rd \lambda \, \mathcal{V}' \right)\frac{\partial f_0}{\partial t} &+\frac{1}{4} V' e\langle \b{E}\cdot\b{B}\rangle \frac{\partial p^2\int_0^{\lambda_T}(f_1^+-f_1^-)\rd\lambda}{\partial p} +\left(\int \rd \lambda \, \mathcal{V}' \right)\frac{1}{p^2}\frac{\partial(p^3 \nu_s f_0)}{\partial p}\nonumber \\
&\hspace{-30mm} +\frac{\partial}{\partial r}\left[ -\left(\int \rd \lambda \, \mathcal{V}' \frac{1}{2}\sum_\sigma\{A\}\right)f_0+  \left(\int \rd \lambda \, \mathcal{V}' \frac{1}{2}\sum_\sigma\{D\}\right)\frac{\partial f_0}{\partial r}\right] + \frac{1}{2}\sum_\sigma\int\rd\lambda\,\mathcal{V}' \{S\}
\end{align}

The electric-field term becomes
\begin{align}
\frac{1}{4} V' e\langle \b{E}\cdot\b{B}\rangle \frac{\partial p^2\int_0^{\lambda_T}(f_1^+-f_1^-)\rd\lambda}{\partial p} &= -\frac{1}{4}\int_0^{\lambda_T}\rd \lambda \int_\lambda^{\lambda_T} \frac{\rd \lambda'}{\langle\sqrt{1-\lambda' B}\rangle} V'(e\langle \b{E}\cdot\b{B}\rangle)^2\frac{\partial}{\partial p} \left(\frac{p^2}{\nu_D} \frac{\partial f_0}{\partial p}\right) \nonumber \\
&= \hspace{-25mm} -\int_0^{\lambda_T}\frac{\lambda\,\rd \lambda}{\langle\sqrt{1-\lambda B}\rangle} \frac{V'(e\langle \b{E}\cdot\b{B}\rangle)^2}{4} \frac{\partial}{\partial p} \left(\frac{p^2}{\nu_D}\frac{\partial f_0}{\partial p}\right) ,
\end{align}
and the $\lambda$-averaged Jacobian becomes
\begin{align}
\int_0^{B\sub{min}}\!\! \rd\lambda \,\mathcal{V}' &= \int_0^{2\pi}\rd\varphi \int_0^{B\sub{min}}\rd\lambda \oint \rd \theta \, \sqrt{g} \nonumber \\
&= \int_0^{2\pi} \rd\varphi \int_0^{2\pi}\rd\theta \int_0^{1/B(\theta)}\rd\lambda \,\sqrt{g} \nonumber \\
&= \frac{p^2}{2}\int_0^{2\pi} \rd\varphi \int_0^{2\pi}\rd\theta \,\mathcal{J}\int_0^{1/B(\theta)}\rd\lambda \,\frac{B}{\sqrt{1-\lambda B}} \nonumber \\
&= p^2V'.
\end{align}


For the Rechester-Rosenbluth like transport model, we can also evaluate
\begin{align}
\int \rd \lambda \, \mathcal{V}' \frac{1}{2}\sum_\sigma\{A\} &= A_0 v\frac{p^2}{2}\int_0^{2\pi} \rd\varphi \int_0^{2\pi}\rd\theta \,\mathcal{J}\int_0^{1/B(\theta)}\rd\lambda \, B \nonumber \\
&= \frac{A_0 v}{2}p^2 V'.
\end{align}

\subsection{Reduced kinetic expression: final expression}
Inserting the above relations and dividing the reduced kinetic equation by $\int \mathcal{V}'\rd\lambda = p^2 V'$ yields
\begin{align}
&\frac{\partial f_0}{\partial t} - \int_0^{\lambda_T}\frac{\lambda\,\rd \lambda}{\langle\sqrt{1-\lambda B}\rangle} \frac{(e\langle \b{E}\cdot\b{B}\rangle)^2}{4} \frac{1}{p^2}\frac{\partial}{\partial p} \left(\frac{p^2}{\nu_D}\frac{\partial f_0}{\partial p}\right) - \frac{1}{p^2}\frac{\partial(p^3 \nu_s f_0)}{\partial p} =  \nonumber \\
&\frac{\frac{1}{2}\sum_\sigma\int\rd\lambda \,\mathcal{V'}\{S\}}{\int \rd\lambda\,\mathcal{V}'}+\frac{1}{\int \rd \lambda \, \mathcal{V}' }\frac{\partial}{\partial r}\left[- \left(\int \rd \lambda \, \mathcal{V}' \frac{1}{2}\sum_\sigma\{A\}\right)f_0+  \left(\int \rd \lambda \, \mathcal{V}' \frac{1}{2}\sum_\sigma\{D\}\right)\frac{\partial f_0}{\partial r}\right] \nonumber \\
&=\left\langle \frac{1}{2}\int_{-1}^1  S\,\rd \xi \right\rangle +  \frac{1}{V'}\frac{\partial}{\partial r}\left[V' \left( -\frac{A_0 v}{2} f_0+  \frac{D_0 v}{2}\frac{\partial f_0}{\partial r}\right)\right],
\end{align}
where the last line is valid explicitly for the transport model $A,D \propto v|\xi|$ and an arbitrary source function $S=S(t,\,r,\,\theta,\,p,\,\xi)$. The equation can also be given explicitly on flux form as
\begin{align}
\frac{\partial f_0}{\partial t} &= \sum_{i=(r,\,p)}  \frac{1}{\sqrt{g}}\frac{\partial}{\partial z^i}\left[\sqrt{g}\left(A^if +\sum_{j=(r,\,p)} D^{ij}\frac{\partial f}{\partial z^j}\right)\right] + S, \nonumber \\
\sqrt{g} &= p^2V', \nonumber \\
A^p &= p\nu_s , \nonumber \\
D^{pp} &=  \int_0^{\lambda_T}\frac{\lambda\,\rd \lambda}{\langle\sqrt{1-\lambda B}\rangle} \frac{(e\langle \b{E}\cdot\b{B}\rangle)^2}{4\nu_D}, \nonumber \\
A^r &= \left\langle \frac{1}{2}\int_{-1}^1  A(t,\,r,\,\theta,\,p,\,\xi) \,\rd\xi \right\rangle, \nonumber \\
D^{rr} &= \left\langle \frac{1}{2}\int_{-1}^1  D(t,\,r,\,\theta,\,p,\,\xi) \,\rd\xi \right\rangle, \nonumber \\
S &= \left\langle \frac{1}{2}\int_{-1}^1  S(t,\,r,\,\theta,\,p,\,\xi) \,\rd\xi \right\rangle.
\end{align}

\subsection{Plasma current in reduced kinetic description}
The parallel current density can in this approximation be evaluated as
\begin{align}
\frac{j_\parallel}{B} &= \pi\int_0^\infty \rd p \int_0^{1/B\sub{max}}\rd\lambda\, evp^2(f^+-f^-) \nonumber \\
&=\pi e\int_0^\infty \rd p\,vp^2 \int_0^{1/B\sub{max}}\rd\lambda\, (f_1^+-f_1^-) \nonumber \\
&= -\pi  e\langle \b{E}\cdot\b{B}\rangle \int_0^{\lambda_T}\frac{\lambda\,\rd \lambda}{\langle\sqrt{1-\lambda B}\rangle}\int \rd p \, \frac{vp^2}{\nu_D}\frac{\partial f_0}{\partial p}.
\end{align}

Similarly, the electric conductivity $\sigma$ is often defined in terms of the parallel Ohmic current $j_\Omega$ as
\begin{align}
\frac{j_\Omega}{B} = \sigma \frac{\langle \b{E}\cdot\b{B} \rangle}{\langle B^2\rangle},
\end{align}
and the runaway current (where we assume that the runaway distribution is characterized by $v\approx c$, $\lambda \ll 1/B$ and $f\sub{RE}(\sigma=-1) \ll f\sub{RE}(\sigma=+1)$). In that case, the runaway current and density are related via
\begin{align}
n\sub{RE} &= \pi\int_0^\infty \rd p \,p^2\int_0^{1/B}\rd \lambda \,\frac{B}{\sqrt{1-\lambda B}}(f\sub{RE}^+ + f\sub{RE}^-) \nonumber \\
&\approx B \pi \int_0^\infty \rd p \,p^2\int_0^{1/B}\rd \lambda \, f\sub{RE}^+, \nonumber \\
\frac{j\sub{RE}}{B} &= \pi\int_0^\infty \rd p \,vp^2 \int_0^{1/B\sub{max}}\rd\lambda \, (f\sub{RE}^+-f\sub{RE}^-)\nonumber \\
&\approx  \pi ec\int_0^\infty \rd p \,p^2 \int_0^{1/B}\rd\lambda \, f\sub{RE}^+ \nonumber \\
&\equiv \frac{ec n\sub{RE}}{B}.
\end{align}

\subsection{Particle flux through the boundaries}
Consider an equation of the form
\begin{align}
\frac{\partial f}{\partial t} &= \frac{1}{4\pi p^2}\frac{\partial F_p}{\partial p},
\end{align}
where $F_p$ denotes the momentum flux, assumed to be isotropic ($f$ and $F_p$ uniform in $\lambda$ and $\sigma$). In that case, the particle flux through the boundaries is given by
\begin{align}
\frac{\partial n}{\partial t} &=[F_p(p\sub{max}) - F_p(p\sub{min})]\frac{B}{2} \int_0^{1/B} \rd \lambda \frac{1}{\sqrt{1-\lambda B}}, \nonumber \\
&=F_p(p\sub{max}) - F_p(p\sub{min}).
\end{align}
Thus, $F_p$ describes the flux of local particle density, which is uniform on the flux surfaces for an isotropic distribution. We wish to match this to the runaway density, which is not uniform on flux surfaces, so we match the flux-surface averaged density instead:
\begin{align}
\frac{\partial \langle n\sub{fast} \rangle}{\partial t} &= F_p(p\sub{max}) - F_p(p\sub{min}), \nonumber \\
\frac{\partial \langle n\sub{RE} \rangle}{\partial t} &= \langle B \rangle \frac{\partial}{\partial t}\frac{n\sub{RE}}{B} = - F_p(p\sub{max}) .
\end{align}
Therefore, to match the flux from the isotropic fast distribution to the beam-like runaway distribution in a way that conserves total particle number, we set
\begin{align}
\frac{\partial}{\partial t}\frac{n\sub{RE}}{B} =  - \frac{F_p(p\sub{max})}{\langle B\rangle} ,
\end{align}
whereas the cold population is whatever is needed to maintain quasi-neutrality -- typically uniform since the runaway density is negligible and we assume uniform impurities. In that case, in principle, the flux from fast population to cold is $\partial n\sub{cold}/\partial t = F_p(0)$.




%\section{Kinetic equation}
%
%
%We study the kinetic equation (ignoring the energy-diffusion term, strictly valid in the superthermal limit) 
%\begin{align}
%\hspace{-1mm}\frac{\partial f}{\partial t} + eE\left(\xi\frac{\partial f}{\partial p} + \frac{1-\xi^2}{p}\frac{\partial f}{\partial \xi}\right) = \frac{1}{p^2}\frac{\partial}{\partial p}\Big(p^3 \nu_s  f\Big) + \frac{\nu_D}{2}\frac{\partial}{\partial \xi}\left[(1-\xi^2)\frac{\partial f}{\partial \xi}\right].
%\end{align}
%Here, $\xi = \b{p}\cdot\b{b}/p$ denotes the pitch, $\nu_s$ is the slowing-down frequency, $\nu_D$ the deflection frequency and $f=f(t,\,p,\,\xi)$ the distribution function. We use the model [L~Hesslow \emph{et al.} JPP 84 (2018)]
%\begin{align}
%\nu_s &= 4\pi c r_0^2 \frac{\gamma^2}{p^3} \left\{n\sub{cold} \ln\Lambda^\text{ee} + \sum_j n_j N_{e,j}\left[\frac{1}{k}\ln(1+ h_j^k)  - \frac{p^2}{\gamma^2}\right]\right\} , \nonumber \\
%\nu_D &= 4\pi c r_0^2\frac{\gamma}{p^3} \left( n\sub{cold} Z\sub{eff} \ln\Lambda^\text{ei} + \sum_j n_j g_j(p) \right), \nonumber \\
%g_j &= \frac{2}{3} (Z_j^2-Z_{0j}^2)\ln[1+ (\bar{a}_jp)^{3/2}] - \frac{2}{3}N_{ej}^2\frac{(\bar{a}_j p)^{3/2}}{1+(\bar{a}_j p)^{3/2}}, \nonumber \\
%h_j &= \frac{m_e c^2}{I_j}p\sqrt{\gamma-1}\nonumber \\
%N_{e,j} &= Z_j-Z_{0j}, \nonumber \\
%Z\sub{eff} &= \sum_j n_j Z_{0j}^2/n\sub{cold}  \nonumber \\
%\ln\Lambda^{ee} &= \ln\Lambda_c + \ln\sqrt{\gamma-1}, \nonumber \\
%\ln\Lambda^{ei} &= \ln\Lambda_c + \ln(\sqrt{2}p), 
%\end{align}
%where $Z_j$ is the atomic number and $Z_{0j}$ the charge number of particle species $j$, where $n_j$ denotes the corresponding number density, $I_j$ is an ionic mean stopping power (tabulated), $\bar{a}_j$ an ion-specific parameter (tabulated) and $n\sub{cold}$ the number density of free cold electrons. Momenta $p$ are normalized to $m_e c$, and $\gamma = \sqrt{1+p^2}$ is the Lorentz factor. 
%
%
%%Here, $\tau_c = (4\pi \ln\Lambda n_e r_0^2 c)^{-1}$ is the relativistic collision time, and $\bar \nu_s$ and $\bar\nu\sub{D}$ are normalized collision frequencies that equal $1$ and $1+Z\sub{eff}$ in an ideal plasma, respectively (fully ionized, constant Coulomb logarithm, no radiation losses).
%
%\subsection{Large-$\nu_D$ reduced kinetic equation}
%
%We proceed to solve the equation perturbatively in the limit of strong pitch angle scattering. We order $\nu_D \sim \delta^0$, $E \sim \delta$ and $\partial/\partial t \sim \nu_s \sim \delta^2$. In that case, writing $f=f_0+\delta f_1+\delta^2 f_2 + ...$, we obtain the system of equations
%\begin{align}
%\frac{\partial}{\partial \xi}\left[(1-\xi^2)\frac{\partial f_0}{\partial \xi}\right] &= 0, \\
% eE\left(\xi\frac{\partial f_0}{\partial p} + \frac{1-\xi^2}{p}\frac{\partial f_0}{\partial \xi}\right) &=  \frac{\nu\sub{D}}{2}\frac{\partial}{\partial \xi}\left[(1-\xi^2)\frac{\partial f_1}{\partial \xi}\right] \\
%\hspace{-7mm} \frac{\partial f_0}{\partial t} +eE \left(\xi\frac{\partial f_1}{\partial p} + \frac{1-\xi^2}{p}\frac{\partial f_1}{\partial \xi}\right) &= \frac{1}{p^2}\frac{\partial}{\partial p}\Big(p^3 \nu_s f_0\Big) + \frac{\nu\sub{D}}{2}\frac{\partial}{\partial \xi}\left[(1-\xi^2)\frac{\partial f_2}{\partial \xi}\right].
%\end{align}
%The first equation yields the general solution
%\begin{align}
%f_0 = f_0(t,p),
%\end{align}
%i.e. the leading-order distribution is isotropic, upon which the second equation takes the form
%\begin{align}
%\frac{2eE}{\nu_D}\xi\frac{\partial f_0}{\partial p } &=\frac{\partial}{\partial \xi}\left[(1-\xi^2)\frac{\partial f_1}{\partial \xi}\right], \nonumber \\
%\Rightarrow f_1 &= eE c(t,\,p) - \xi \frac{eE}{\nu_D} \frac{\partial f_0}{\partial p},
%\end{align}
%where $c$ is an undetermined integration constant that doesn't influence the evolution of $f_0$. Indeed, it can be shown (via the third-order equation) that $c$ obeys exactly the same equation as $f_0$, and if $f_0$ is initialized such that $c=0$ at $t=0$, it will remain so for all times.
%
%Integrating the third equation over $\xi$ from -1 to 1 (and dividing by 2) yields the final kinetic equation (suppressing the subscript $0$)
%\begin{align}
%\frac{\partial f}{\partial t} - \frac{1}{p^2}\frac{\partial}{\partial p} \left[ p^2\left( \frac{1}{3}\frac{(eE)^2}{\nu_D}\frac{\partial f}{\partial p} + p \nu_s  f \right)\right] = 0.
%\end{align}
%This is the reduced 1D kinetic equation for $f_0 = f_0(t,\,p)$ that we shall solve. We will now suppress the subscript 0, and can write the equation on divergence form as 
%\begin{align}
%\frac{\partial f}{\partial t} &= \nabla \cdot \b{S} = \frac{1}{p^2}\frac{\partial p^2 S}{\partial p} = \frac{1}{4\pi p^2}\frac{\partial F}{\partial p}, 
%\label{eq:kinetic eq} \\
%F &=  4\pi p^3 \nu_s  f +  \frac{4\pi}{3}p^2\frac{(eE)^2}{\nu_D}\frac{\partial f}{\partial p} 
%\end{align}
%Since we will ultimately consider an equation for which $S$ is singular at the origin, it is convenient to work with the flux $F=p^2 S$ instead of the actual momentum-space flux $S$. 
%
%\subsection{Fast current}
%The current carried by this distribution is given by
%\begin{align}
%j \sub{fast} &=  2\pi ec\int_0^\infty \rd p \, \frac{p^3}{\gamma} \int_{-1}^1 \rd \xi \, \xi f_1   \nonumber \\
%&= -\frac{4\pi}{3} \frac{e^2 E}{m_e} \int_0^\infty \rd p \, \frac{p^3 }{\gamma \nu_D}\frac{\partial f}{\partial p}  \nonumber \\
%&= \frac{4\pi}{3} \frac{e^2 E}{m_e} \int_0^\infty  \frac{\partial}{\partial p}\left(\frac{p^3}{\gamma\nu_D}\right) f\,\rd p.
%\end{align}
%Calculating the current up to some $p=p\sub{max}$ yields
%\begin{align}
%j\sub{fast} &= \frac{4\pi}{3} \frac{e^2 E}{m_e} \left(\int_0^{p\sub{max}}  \frac{\partial}{\partial p}\left(\frac{p^3}{\gamma\nu_D}\right) f\,\rd p - \left.\frac{p^3}{\gamma\nu_D}f\right|_{p=p\sub{max}}\right).
%\end{align}
%It is probably desirable to limit the current to that which would be obtained if all particles moved with $\xi=1$, i.e. $4\pi ec \int p^3/\gamma f \, \rd p$. The reason that this value can be exceeded is that we assumed $E/\nu_D \sim \delta$, which is often valid for low momenta, but breaks down in the runaway region where electric-field acceleration starts to dominate. A way to ``match'' the solution to this region, we may assume that the evolution of the distribution in momentum is relatively accurate, yet amend the current by writing
%\begin{align}
%j = -\frac{4\pi}{3} \frac{e^2 E}{m_e} \int_0^{p\sub{cut}} \rd p \, \frac{p^3 }{\gamma \nu_D}\frac{\partial f}{\partial p}  + 4\pi ec \int_{p\sub{cut}}^\infty \frac{p^3}{\gamma }f \, \rd p,
%\end{align}
%where $p\sub{cut}=p\sub{cut}(E)$ is the momentum for which
%\begin{align}
%-\frac{1}{3}\frac{eE}{m_e c} \frac{p^3}{\gamma\nu_D}\frac{\partial f}{\partial p} = \frac{p^3}{\gamma} f, \nonumber \\
%\nu_D(p\sub{cut}) = -\frac{1}{3}\frac{eE}{m_e c} \frac{1}{f}\frac{\partial f}{\partial p}.
%\end{align}
%In this way, if $p\sub{cut} < p\sub{max}$, the current density will be matched smoothly to the runaway region where we assume that all particles move with $\xi=1$ and $v=c$.
%Since $E$ will typically depend on $j$ in a self-consistent treatment, this method would need to be solved self-consistently and the value of $p\sub{cut}$ cannot be given on closed form but must be evaluated iteratively.
%
%
%\subsection{Bounce-averaged kinetic equation}
%In order to facilitate the bounce average, we express the equation in momentum coordinates that are conserved along the orbit, i.e. functions of the energy $\gamma$ and of the magnetic moment 
%\begin{align}
%\mu = p_\perp^2/(2mB).
%\end{align}
%One such choice is to use the momentum $p$ and the normalized magnetic moment
%\begin{align}
%\lambda &= \frac{2m\mu}{p^2} = \frac{1}{B}\frac{p_\perp^2}{p^2}= \frac{1-\xi^2}{B}, \nonumber \\
%\xi &= \sigma \sqrt{1-B\lambda}
%\end{align}
%where $\lambda$ is in units of inverse magnetic field strength, and which we complement with the sign of $\xi$, denoted $\sigma = \text{sgn}(\xi)$. Expressed in terms of $r,\,p,\,\lambda,\,\sigma$, the distribution is constant along the trajectory in the banana regime (where the transit time is much shorter than the collision time and other time scales of the problem).
%
%Particles with $\mu B\sub{max} > p^2/2m$ will be trapped, where $B\sub{max}$ is the maximum value of the magnetic field along the particle trajectory. This corresponds to a trapped region $\lambda > \lambda_{T}$, where
%\begin{align}
%\lambda_T = \frac{1}{B\sub{max}},
%\end{align}
%whereas the largest possible $\lambda$ is given by $1/B\sub{min}$.
%
%\subsubsection*{Pitch-angle scattering term}
%The pitch-angle scattering term is bounce averaged via the relations
%\begin{align}
%\frac{\partial}{\partial \xi} &= \frac{\partial \lambda}{\partial \xi} \frac{\partial}{\partial \lambda} = -\frac{2\xi}{B}\frac{\partial}{\partial \lambda} = -\frac{2\sigma \sqrt{1-B\lambda}}{B}\frac{\partial}{\partial \lambda}, \nonumber \\
%\frac{\partial}{\partial \xi}\left[(1-\xi^2)\frac{\partial}{\partial \xi} \right] &= 4\frac{\xi}{B} \frac{\partial}{\partial \lambda}  \left[\lambda \sigma \sqrt{1-B\lambda}\frac{\partial}{\partial \lambda}\right]
%\end{align}
%The bounce integral of this term is
%\begin{align}
%\left\{ \frac{\partial}{\partial \xi}\left[(1-\xi^2)\frac{\partial}{\partial \xi} \right]\right\} &= \frac{1}{2\pi\partial \psi/\partial r} \frac{4}{ v}\frac{\partial}{\partial \lambda}\left(\lambda S(\lambda)\frac{\partial}{\partial \lambda}\right), \nonumber \\
%S &= 2\pi \oint \rd \theta \,\mathcal{J}\sigma\sqrt{1-B\lambda}
%\end{align}
%
%\subsubsection*{Electric field term}
%The electric field term is given by
%\begin{align}
%&eE_\parallel \left\{\frac{1}{p^2}\frac{\partial}{\partial p}(\xi p^2 f) + \frac{1}{p}\frac{\partial}{\partial \xi}\Big[(1-\xi^2) f\Big] \right\} \nonumber \\
%&= eE_\parallel \xi\left(\frac{\partial  f}{\partial p} - \frac{2\lambda}{p}\frac{\partial f}{\partial \lambda} \right)
%\end{align}
%The bounce average of this term for passing particles becomes
%\begin{align}
%%\{...\}\sub{passing} = \frac{e}{2\pi v \partial \psi/\partial r} \langle \b{E}\cdot\b{B}\rangle \left(\frac{1}{p^2}\frac{\partial p^2 f}{\partial p} - \frac{2}{p}\frac{\partial \lambda f}{\partial \lambda} \right),
%\{...\}\sub{passing} = \frac{e}{2\pi v \partial \psi/\partial r} \langle \b{E}\cdot\b{B}\rangle \left(\frac{\partial  f}{\partial p} - \frac{2\lambda}{p}\frac{\partial f}{\partial \lambda} \right),
%\end{align}
%and vanishes for trapped particles since the entire term is odd in $\sigma$ (as $f$ is always even in $\sigma$ in the trapped region). Here, the flux surface average of the electric field can be expressed in terms of the loop voltage in an arbitrary axisymmetric system as
%\begin{align}
%\langle \b{E}\cdot\b{B} \rangle = (B_\varphi R) \left\langle\frac{1}{R^2}\right\rangle \frac{V\sub{loop}}{2\pi},
%\end{align}
%where $B_\varphi R = G(r)$ is a flux function.
%
%\subsubsection*{Time derivative and friction term}
%Finally, the time derivative and friction terms pick up a prefactor
%\begin{align}
%\{1\} &= \frac{1}{2\pi v \partial \psi/\partial r} L \nonumber \\\
%L &= 2\pi \oint \rd\theta \, \mathcal{J}\frac{\sigma B}{\sqrt{1-B\lambda}}.
%\end{align}
%
%\subsubsection*{Final bounce-averaged kinetic equation}
%\textcolor{red}{[this is not quite right, redo the bounce average, and do it right (see local transport model section)]} If we multiply the entire equation by $\sigma 2\pi v \partial \psi/\partial r$, it then takes the form
%\begin{align}
%\bar{L}\frac{\partial f}{\partial t} &+ \sigma\Theta G(r) \left\langle\frac{1}{R^2}\right\rangle\frac{e V\sub{loop}}{2\pi}\left(\frac{1}{p^2}\frac{\partial  p^2 f}{\partial p} - \frac{2}{p}\frac{\partial \lambda f}{\partial \lambda} \right) \nonumber \\
%&= \frac{1}{p^2}\frac{\partial}{\partial p}( p^3 \nu_s \bar{L}f) +  2\nu_D \frac{\partial}{\partial \lambda} \left(\lambda  \bar{S}\frac{\partial f}{\partial \lambda}\right), \\
%\bar{L} &= \begin{cases}
%2\pi \int_0^{2\pi} \rd \theta \,\mathcal{J}\frac{B}{\sqrt{1-B\lambda}} = \left\langle\frac{B}{|\xi|}\right\rangle, & \text{passing} ~ (\lambda \leq \lambda_T) \\
%4\pi \sigma \int_{-\theta\sub{bounce}}^{\theta\sub{bounce}} \rd \theta \, \mathcal{J}\frac{B}{\sqrt{1-B\lambda}}, & \text{trapped}~(\lambda > \lambda_T)
%\end{cases}\nonumber \\
%%2\pi \oint \rd\theta \, \mathcal{J}\frac{\sigma B}{\sqrt{1-B\lambda}} \nonumber \\
%\bar{S} &= \begin{cases}
%2\pi \int_0^{2\pi} \rd \theta \,\mathcal{J} \sqrt{1-B\lambda}  = \langle |\xi| \rangle, & \text{passing} \\
%4\pi\sigma \int_{-\theta\sub{bounce}}^{\theta\sub{bounce}} \rd \theta \,\mathcal{J}\sigma\sqrt{1-B\lambda}, & \text{trapped} \\
%\end{cases}\nonumber \\
%\left\langle\frac{1}{R^2}\right\rangle &= 2\pi \int_0^{2\pi} \rd\theta \, \frac{\mathcal{J}}{R^2} \nonumber \\
%\Theta(\lambda) &= \begin{cases}
%1, & \text{passing} ~ \, (\lambda \leq \lambda\sub{T}) \\
%0, & \text{trapped} ~ (\lambda > \lambda\sub{T})
%\end{cases}
%\end{align}
%
%\subsection{Solution of bounce-averaged reduced kinetic equation}
%If we employ the same ordering as previously, $\nu_D \sim \delta^0$, $E\sim \delta$ and $\partial/\partial t \sim \nu_s \sim \delta^2$, we obtain the equation system
%\begin{align}
%\frac{\partial}{\partial \lambda} \left(\lambda \bar{S}(\lambda)\frac{\partial f_0}{\partial \lambda}\right) &= 0,
%\end{align}
%yielding
%\begin{align}
%f_0 = f_0(t,\,r,\,p).
%\end{align}
%The next-order equation then reads
%\begin{align}
%\sigma  \Theta G(r) \left\langle\frac{1}{R^2}\right\rangle\frac{eV\sub{loop}}{2\pi}\frac{\partial  f_0}{\partial p} &= 2\nu_D \frac{\partial}{\partial \lambda} \left(\lambda \bar{S}(\lambda)\frac{\partial f_1}{\partial \lambda}\right) .
%\end{align}
%Multiplying by $\sigma$ and summing over the signs, as well as integrating the equation over $\lambda$ from 0 to $\lambda$ yields (for $\lambda < \lambda_T$; in the trapping region the equation reads $0=0$)
%\begin{align}
% \frac{\partial (f_1^+-f_1^-)}{\partial \lambda} = \frac{1}{\langle |\xi|\rangle}\Theta G(r) \left\langle\frac{1}{R^2}\right\rangle\frac{eV\sub{loop}}{2\pi\nu_D}\frac{\partial  f_0}{\partial p}.
%\end{align}
%Integrating this from $\lambda$ to $\lambda_T=1/B\sub{max}$ yields (where the boundary term on the LHS vanishes since $f$ is even in the trapped region)
%\begin{align}
%f_1^+-f_1^- = -\int_\lambda^{\lambda\sub{T}} \frac{\rd \lambda }{\langle |\xi|\rangle} \, G(r)\left\langle\frac{1}{R^2}\right\rangle\frac{eV\sub{loop}}{2\pi\nu_D}\frac{\partial  f_0}{\partial p}.
%\end{align}
%
%Finally, the second-order equation summed over $\sigma$, divided by $2$ and integrated over all $\lambda$ (the contribution from $\lambda > \lambda_T$ being identically zero for all terms in the equation), reads
%\begin{align}
%l\frac{\partial  f_0}{\partial t} &- \frac{G(r)}{2} \left\langle\frac{1}{R^2}\right\rangle\frac{eV\sub{loop}}{2\pi}\frac{1}{p^2}\frac{\partial  p^2\int_0^{\lambda\sub{T}}(f_1^+-f_1^-)\rd\lambda}{\partial p} = \frac{1}{p^2}\frac{\partial}{\partial p}\left( p^3 \nu_s l f_0\right), \nonumber \\
%l &= \int_0^{\lambda_T} \left\langle\frac{B}{|\xi|}\right\rangle \rd\lambda
%%\int_0^{1/B\sub{min}} L\,\rd\lambda = 2\pi \int_0^{1/B\sub{min}} \rd \lambda \, \oint\rd\theta\, \mathcal{J} \frac{\sigma B}{\sqrt{1-B\lambda}} = \int_0^{1/B\sub{min}}
%\end{align}
%where we insert our expression for $f_1^+-f_1^-$ and divide by $l$, which finally yields
%\begin{align}
%\frac{\partial f_0}{\partial t} &= \frac{1}{p^2}\frac{\partial}{\partial p}\left[ p^3\nu_s f_0 +\frac{1}{2\nu_D} \frac{\int_0^{\lambda\sub{T}} \rd \lambda \frac{\lambda}{\langle |\xi|\rangle}}{\int_0^{\lambda_T} \left\langle\frac{B}{|\xi|}\right\rangle} \left(G(r)\left\langle\frac{1}{R^2}\right\rangle\frac{eV\sub{loop}}{2\pi}\right)^2\frac{\partial f_0}{\partial p}\right], \nonumber \\
% &= \frac{1}{p^2}\frac{\partial}{\partial p}\left\{p^2 \left[p\nu_s f_0 + \frac{\eta(r)}{3\nu_D}\left(\frac{eV\sub{loop}}{2\pi R_m}\right)^2 \frac{\partial f}{\partial p}\right]\right\} \nonumber \\
%\eta(r) &= \frac{3}{2}R_m^2 G(r)^2 \left\langle\frac{1}{R^2}\right\rangle^2 \frac{\int_0^{\lambda\sub{T}} \frac{\lambda}{\langle|\xi|\rangle}\,\rd \lambda }{\int_0^{\lambda_T}\left\langle\frac{B}{|\xi|}\right\rangle\,\rd\lambda}.
%%V'\int_0^{\lambda\sub{T}} \rd \lambda \int_\lambda^{\lambda\sub{T}} \frac{\rd \lambda'}{S(\lambda')} =V' \int_0^{\lambda\sub{T}} \rd \lambda \frac{\lambda}{S(\lambda)}.
%\end{align}
%The neoclassical correction factor $\eta$ captures the difference between the 0D equation and the bounce averaged equation with the replacement $E = V\sub{loop}/(2\pi R_m)$ (but note that the bounce averaged equation never explicitly refers to $R_m$; the term in $\eta$ cancels against the corresponding term under $V\sub{loop}$). For constant elongation and no triangularity, it goes like $\eta = 1-0.57\sqrt{r/R_m} - 0.44 r/R_m$.
%
%%\noindent \textcolor{red}{[TODO: evaluate for $B=\text{constant}$ and verify that it reduces to previous result]}
%
%\subsection{Plasma current in bounce-averaged description}
%The toroidal plasma current is given by the surface integral
%\begin{align}
%I_p(r) &= \int  j_\varphi \frac{\rd r \rd \theta}{|\nabla r\times\nabla\theta|} \nonumber \\ 
%&= \int \b{j}\cdot\nabla \varphi \, \frac{R\rd r \rd \theta}{|\nabla r\times\nabla\theta|} .
%\end{align}
%We can utilize that the system is axisymmetric by dividing by $2\pi$ and integrating over $\varphi$ from 0 to $2\pi$.
%Since $\nabla \varphi$ is orthogonal to $\nabla r$ and $\nabla \theta$, and $|\nabla \varphi| = 1/R$, we can rewrite the volume-element term as
%\begin{align}
%I_p(r) &= \frac{1}{2\pi} \int \b{j}\cdot\nabla \varphi \, \frac{\rd r \rd \theta \rd\varphi}{|\nabla \varphi \cdot(\nabla r\times\nabla\theta)|}  \nonumber \\
%&= \frac{1}{2\pi} \int_0^r \rd r \,\langle \b{j}\cdot\nabla \varphi \rangle.
%\end{align}
%Furthermore, in a low-beta (pressureless) plasma we can write
%\begin{align}
%\b{j} = \frac{j_\parallel}{B}\b{B},
%\end{align}
%where $j_\parallel/B$ is a flux function. In that case,
%\begin{align}
%I_p(r) &= \frac{1}{2\pi} \int_0^r \rd r \, \frac{j_\parallel}{B} \langle \b{B}\cdot\nabla \varphi \rangle \nonumber \\
%&= \frac{1}{2\pi} \int_0^r \rd r\, \frac{j_\parallel}{B} G(r) \left\langle\frac{1}{R^2}\right\rangle \nonumber \\
%&= 2\pi  \int_0^r \rd r \,  \frac{\partial \psi}{\partial r}q(r)\frac{j_\parallel}{B} ,
%\end{align}
%where $G = B_\varphi R$, and the $q$-factor is defined by
%\begin{align}
%q &= \frac{\langle \b{B}\cdot\nabla \varphi \rangle}{\langle \b{B}\cdot\nabla \theta\rangle} = \frac{\int \rd \theta \int \rd \varphi \, \frac{\b{B}\cdot\nabla\varphi}{\b{B}\cdot\nabla\theta} }{\int \rd \theta \int \rd \varphi} \nonumber \\
%&= \frac{1}{(2\pi)^2} \frac{G(r)}{\partial \psi/\partial r} \left\langle\frac{1}{R^2}\right\rangle.
%\end{align}
%
%\subsubsection*{Current density calculated from disribution function}
%The local parallel plasma current is defined by
%\begin{align}
%j_\parallel &= e\int v_\parallel f \,\rd\b{p} = 2\pi ec \int \, \frac{p^3}{\sqrt{1+p^2}}\xi f \,\rd p\rd\xi \nonumber \\
%&= \pi ec \sum_\sigma \sigma \int \frac{p^3}{\sqrt{1+p^2}}\sqrt{1-B\lambda}  f \, \rd p \frac{B}{\sqrt{1-B\lambda}}  \rd \lambda \nonumber \\
%&= B \pi ec \int \frac{p^3}{\sqrt{1+p^2}} (f^+-f^-) \,\rd p \rd \lambda.
%\end{align}
%This gives us the flux function 
%\begin{align}
%\frac{j_\parallel}{B} &= \pi ec \int_0^\infty \rd p \, \frac{p^3}{\sqrt{1+p^2}} \int_0^{1/B\sub{max}} \rd \lambda \,(f^+-f^-) \nonumber \\
%&= -\pi e c G(r)\left\langle\frac{1}{R^2}\right\rangle \int_0^{\lambda_T}\frac{\lambda}{\langle|\xi|\rangle}\rd\lambda \frac{eV\sub{loop}}{2\pi}\int_0^\infty \rd p \, \frac{p^3}{\nu_D \sqrt{1+p^2}}\frac{\partial f_0}{\partial p}.
%% &= - e \eta(r) G(r)\frac{1}{V'}\left\langle  \frac{1}{R^2} \right\rangle  V\sub{loop} \int_0^\infty \rd p \, vp^2 \frac{1}{\nu_D}\frac{\partial f_0}{\partial p}.
%\end{align}
%It then follows that the contribution to the plasma current is
%\begin{align}
%\frac{\partial I_p}{\partial r} &= \frac{1}{2\pi}\frac{j_\parallel}{B}G(r)\left\langle\frac{1}{R^2}\right\rangle \nonumber \\
%&= \eta_2(r)\frac{4\pi}{3}\frac{eV\sub{loop}}{2\pi R_m} \int_0^\infty \rd p \,\frac{p^3}{\sqrt{1+p^2}\nu_D}\frac{\partial f_0}{\partial p}, \nonumber \\
%\eta_2(r) &=\eta(r) \frac{1}{4\pi R_m}\int_0^{\lambda_T}\left\langle\frac{B}{|\xi|}\right\rangle\rd\lambda \nonumber \\
%&=\frac{3}{8\pi} R_m^2G(r)^2\left\langle\frac{1}{R^2}\right\rangle^2 \int_0^{\lambda_T} \frac{\lambda}{\langle|\xi|\rangle}\rd\lambda,
%%-\frac{\eta(r)}{3R_m}\int_0^{\lambda_T}\left\langle\frac{B}{|\xi|}\right\rangle\rd\lambda \, \frac{eV\sub{loop}}{2\pi R_m} \int_0^\infty \rd p \,\frac{vp^2}{\nu_D}\frac{\partial f_0}{\partial p}.
%\end{align}
%where the neoclassical correction $\eta_2 \sim 1-1.46\sqrt{r/R_m} + \Ordo((r/R_m)^{3/2})$ in an equilibrium with constant elongation and no triangularity.
%
%
%Similarly, the Ohmic parallel current can be given as
%\begin{align}
%\frac{j_\Omega}{B} = \frac{G(r)}{\langle B^2\rangle}\left\langle \frac{1}{R^2} \right\rangle  \sigma\frac{V\sub{loop}}{2\pi},
%\end{align}
%where $\sigma$ is the Spitzer resistivity (with or without neoclassical corrections, depending on collisionality regime).
%
%For the runaway density we wish to relate the runaway density to the runaway current. The density is given by
%\begin{align}
%n\sub{RE} = 2\pi\int_0^\infty \rd p \, p^2 \int_{-1}^1 \rd \xi \, f\sub{RE} = \pi B \int_0^\infty \rd p \, p^2 \int_0^{1/B\sub{min}} \rd\lambda \, \frac{f^+ + f^-}{\sqrt{1-\lambda B}}.
%\end{align}
%If we consider a beam-shaped runaway distribution where only an insignificant number of particles are near the trapped region, we approximate $\sqrt{1-\lambda B} \approx 1$ and $f^- \ll f^+$, in which case we find
%\begin{align}
%\frac{n\sub{RE}}{B} &\approx \pi \int_0^\infty \rd p \, p^2 \int \rd \lambda\, f^+, \nonumber \\
%\frac{j\sub{RE}}{B} &=  \pi ec \int_0^\infty \rd p \, \frac{p^3}{\sqrt{1+p^2}} \int_0^{1/B\sub{max}} \rd \lambda \,(f^+-f^-) \nonumber \\
%&\approx \pi e c \int_0^\infty \rd p \, p^2 \int \rd \lambda \,f^+ \nonumber \\
%&\equiv ec \frac{n\sub{RE}}{B},
%\end{align}
%where we assumed that the bulk of runaway current is carried by electrons with $p \gg 1$.
%
%\subsection{Particle flux through the boundaries}
%Consider an equation of the form
%\begin{align}
%\frac{\partial f}{\partial t} &= \frac{1}{4\pi p^2}\frac{\partial F_p}{\partial p},
%\end{align}
%where $F_p$ denotes the momentum flux, assumed to be isotropic ($f$ and $F_p$ uniform in $\lambda$ and $\sigma$). In that case, the particle flux through the boundaries is given by
%\begin{align}
%\frac{\partial n}{\partial t} &=[F_p(p\sub{max}) - F_p(p\sub{min})]\frac{B}{2} \int_0^{1/B} \rd \lambda \frac{1}{\sqrt{1-\lambda B}}, \nonumber \\
%&=F_p(p\sub{max}) - F_p(p\sub{min}),
%\end{align}
%Thus, $F_p$ describes the flux of local particle density, which is uniform on the flux surfaces for an isotropic distribution. We wish to match this to the runaway density, which is not uniform on flux surfaces, so we match the flux-surface integrated density instead:
%\begin{align}
%\frac{\partial \langle n\sub{fast} \rangle}{\partial t} &= V'[F_p(p\sub{max}) - F_p(p\sub{min})], \nonumber \\
%\frac{\partial \langle n\sub{RE} \rangle}{\partial t} &= \langle B \rangle \frac{\partial}{\partial t}\frac{n\sub{RE}}{B} = -V' F_p(p\sub{max}) + ...
%\end{align}
%Therefore, to match the flux from the isotropic fast distribution to the beam-like runaway distribution in a way that conserves total particle number, we set
%\begin{align}
%\frac{\partial}{\partial t}\frac{n\sub{RE}}{B} =  - \frac{F_p(p\sub{max})}{\langle B\rangle/V'} ,
%\end{align}
%whereas the cold population is whatever is needed to maintain quasi-neutrality -- typically uniform since the runaway density is negligible and we assume uniform impurities. In that case, in principle, the flux from fast population to cold is $\partial n\sub{cold}/\partial t = F_p(0)$.
%
%%and if we require net particle conservation locally on a flux surface, and assume $F_p$ to be independent of $\lambda$, we can average this to obtain
%%\begin{align}
%%\frac{\partial}{\partial t} \langle n \rangle &= [F_p(p\sub{max}) - F_p(p\sub{min}) ]\frac{1}{2} \left\langle\int_0^{1/B} \rd \lambda \,\frac{B}{\sqrt{1-\lambda B}}\right\rangle \nonumber \\
%%&=V'[F_p(p\sub{max}) - F_p(p\sub{min}) ] .
%%\end{align}
%
%\subsection{Local transport model}
%We consider a general advection-diffusion term in phase space of the form
%\begin{align}
%\mathcal{D}f = \frac{\partial}{\partial \b{z}} \cdot \left(\b{A}f + \mathsf{D}\cdot\frac{\partial}{\partial \b{z}} f\right),
%\end{align}
%where $\b{z} = (\b{x},\,\b{p})$. 
%
%In general curvilinear coordinates, we can write for the diffusion term
%\begin{align}
%\mathsf{D}\cdot\nabla f &= \sum_{jkl} D^{jk} (\b{b}_j \otimes \b{b}_k) \cdot \frac{\partial f}{\partial z^l}  \b{b}^l \nonumber \\
%&=  \sum_{jk}\b{b}_j D^{jk} \frac{\partial f}{\partial z^k}% \nonumber \\
%%&= \sum_l g_{il}\b{b}^l  \sum_j D^{ij} \frac{\partial f}{\partial z^j}
%\end{align}
%The divergence of this is given by
%\begin{align}
%\nabla \cdot \Big(\mathsf{D}\cdot\nabla f \Big) &= \sum_{ijk} \frac{1}{\sqrt{g}}\frac{\partial}{\partial z^i}\left(\sqrt{g} (\b{b}^i\cdot\b{b}_j)D^{jk}\frac{\partial f}{\partial z^k}\right) \nonumber \\
%&= \sum_{ij} \frac{1}{\sqrt{g}}\frac{\partial}{\partial z^i}\left(\sqrt{g} D^{ij}\frac{\partial f}{\partial z^j}\right)  \nonumber \\
%&= \sum_{ijk} \frac{1}{\sqrt{g}}\frac{\partial}{\partial z^i}\left(\sqrt{g} g^{jk} {D^i}_k\frac{\partial f}{\partial z^j}\right) .
%%
%%&= \frac{1}{\sqrt{g}}\frac{\partial}{\partial z^i}\left(\sqrt{g} (\nabla z^i \cdot \nabla z^j) D^{ij}\frac{\partial f}{\partial z^j}\right)
%\end{align}
%where $g^{jk} = \nabla z^j \cdot \nabla z^k$ and $\sqrt{g}$ is the Jacobian for the coordinate system $\{z^i\}$. For our coordinate system $(r,\,\theta,\,\varphi,\,p,\,\lambda,\,\zeta,\,\sigma)$, the Jacobian is 
%\begin{align}
%\sqrt{g} = \sigma \mathcal{J}\frac{B}{2\xi}p^2,
%\end{align}
%(and the phase-space volume integral is $\mathcal{V} = \int \rd \b{x} \rd\b{p} = \sum_\sigma \int \rd r \rd \theta \rd\varphi \rd p \rd \lambda \rd \zeta \,\sqrt{g}$).
%
%The advection term, in turn, takes the form
%\begin{align}
%\nabla \cdot (\b{A}f) &= \frac{1}{\sqrt{g}}\frac{\partial}{\partial z^i} \left(\sqrt{g}A^i f\right) \nonumber \\
%&= \frac{1}{\sqrt{g}}\frac{\partial}{\partial z^i} \left(\sqrt{g}(\b{A} \cdot \nabla r) f\right) \nonumber \\
%&=\sum_{ij} \frac{1}{\sqrt{g}}\frac{\partial}{\partial z^i} \left(\sqrt{g}g^{ij} A_j f\right) 
%\end{align}
%
%
%\subsubsection*{Bounce averaged transport model}
%We take the volume average of the diffusion term over all variables that the distribution function does not depend on. In the banana regime, we first assume that $f$ only depends on $r$,\,$p$ and $\lambda$. We then define the integral
%\begin{align}
%\{ X \} = \oint \rd \theta \int_0^{2\pi} \rd \varphi \oint_0^{2\pi}\rd \zeta \,\sqrt{g} X,
%\end{align}
%where again $\oint \rd \theta = \int_0^{2\pi} \rd \theta$ for passing particles and $\oint \rd \theta = \int_{-\theta\sub{bounce}}^{\theta\sub{bounce}}\rd \theta \sum_\sigma $ for trapped particles.
%The phase-space volume integral of $X$ is $\int X \,\rd\b{x}\rd \b{p} = \sum_\sigma \int \rd r \int \rd p \int \rd \lambda \,\{X\}$.
%Thus defined, the integrated transport model becomes
%\begin{align}
%\{\mathcal{D}f\} = \sum_{i = r,\,p,\,\lambda} \frac{\partial}{\partial z^i} \left[\{A^i\} f +\sum_{j=r,p,\lambda}\sum_k  \left\{(\nabla z^j \cdot \nabla z^k){D^i}_k\right\}\frac{\partial f}{\partial z^j}\right].
%\end{align}
%
%\subsubsection*{Radial transport model}
%
%Since we lack a first-principle model of the transport of runaways, we will simply have to assume some model for the transport coefficients $A^i$ and ${D^i}_k$ (note that we could just as well choose, arbitrarily, to specify for example $D^{rr}$ instead, at which point the $|\nabla r|^2$ term would disappear below). The main model we shall persue is a Rechester-Rosenbluth like model where we take
%\begin{align}
%A^r &= A_0 |\xi| v, \nonumber \\
%{D^r}_r &= D_0 |\xi| v,
%\end{align}
%where $A_0$ and $D_0$ are constant and all other components are assumed to vanish. In that case, the bounce averaged transport equation becomes
%\begin{align}
%\{\mathcal{D}f\} = \frac{\partial}{\partial r}\left[A_0\{v \xi\} f +D_0\left\{|\nabla r|^2v\xi\right\}\frac{\partial f}{\partial r}\right].
%\end{align}
%The integrals can be given explicitly as
%\begin{align}
%\{v |\xi|\} &= (2\pi)^2 \frac{vp^2}{2} \oint \rd \theta \, \mathcal{J} B \nonumber \\
%\{|\nabla r|^2 v|\xi|\} &= (2\pi)^2 \frac{vp^2}{2} \oint \rd \theta \, \mathcal{J} |\nabla r|^2B.
%\end{align}
%Since these bounce-averaged transport coefficients become independent of $\lambda$ except for in the integration limits, 
%
%
%\subsection{Bounce-averaged effective critical field}
%Hesslow ecritpaper considers the kinetic equation with the addition of bremsstrahlung and synchrotron radiation losses
%\begin{align}
%\frac{\partial f}{\partial t} &= \frac{1}{p^2}\frac{\partial}{\partial p}\left[ p^2\left(-\xi eE + p\nu_s + F\sub{br}+\frac{p\gamma}{\tau_s}(1-\xi^2)\right)f\right] \nonumber \\
%&+\frac{\partial}{\partial \xi}\left[ (1-\xi^2)\left( -\frac{1}{p}eE + \frac{\nu_D}{2}\frac{\partial}{\partial \xi} - \frac{\xi}{\tau_s \gamma} \right)f\right].
%\end{align}
%In terms of our normalized magnetic moment coordinate $\lambda = (1-\xi^2)/B$, this takes the form
%\begin{align}
%\frac{\partial f}{\partial t} &= \frac{1}{p^2}\frac{\partial}{\partial p}\left[ p^2\left(-\xi eE + p\nu_s + F\sub{br}+\frac{p\gamma}{\tau_s}\lambda B \right)f\right] \nonumber \\
%& +2\xi \frac{\partial}{\partial \lambda}\left[\lambda \left( \frac{1}{p}eE  + \sigma \sqrt{1-\lambda B} \nu_D\frac{\partial}{\partial \lambda} + \sigma\sqrt{1-\lambda B} \frac{1}{\tau_s\gamma} \right)f\right].
%\end{align}
%Bounce averaging and multiplying by $\sigma 2\pi v \partial \psi/\partial r$ as before, this takes the form
%\begin{align}
%\bar{L}\frac{\partial f}{\partial t} &= \frac{1}{p^2}\frac{\partial}{\partial p}\left[ p^2\left( -\sigma\Theta G(r) \left\langle\frac{1}{R^2}\right\rangle\frac{e V\sub{loop}}{2\pi} + p\nu_s \bar{L} + \frac{p\gamma}{B^2\tau_s}\bar{H}\lambda \right)f\right] \nonumber \\
%&+ \frac{\partial}{\partial \lambda}\left[ \lambda\left(\frac{2}{p}\sigma\Theta G(r) \left\langle\frac{1}{R^2}\right\rangle\frac{e V\sub{loop}}{2\pi} + 2 \bar{S}\nu_D\frac{\partial}{\partial \lambda} + \frac{2\bar{J}}{B^2\tau_s \gamma}  \right)f\right], \\ 
%%&= \frac{1}{p^2}\frac{\partial}{\partial p} \left( p^2 U(p
%\bar{H} &= \begin{cases}
%2\pi \int_0^{2\pi}\rd \theta \, \mathcal{J} \frac{B^4}{\sqrt{1-B\lambda}} = \left\langle\frac{B^4}{|\xi|}\right\rangle , & \text{passing}~(\lambda \leq \lambda_T) \\
%4\pi \sigma \int_{-\theta\sub{bounce}}^{\theta\sub{bounce}} \rd \theta \, \mathcal{J} \frac{B^4}{\sqrt{1-B\lambda}}, & \text{trapped}~(\lambda > \lambda_T),
%\end{cases} \nonumber \\
%\bar{J} &= \begin{cases}
%2\pi \int_0^{2\pi} \rd \theta \,\mathcal{J} B^2\sqrt{1-B\lambda}  = \langle B^2|\xi| \rangle, & \text{passing} \\
%4\pi\sigma \int_{-\theta\sub{bounce}}^{\theta\sub{bounce}} \rd \theta \,\mathcal{J}\sigma B^2\sqrt{1-B\lambda}, & \text{trapped} \\
%\end{cases}\nonumber \\
%\frac{1}{\tau_s} &= \frac{e^4 B^2}{6\pi \varepsilon_0 m_e^3 c^3},
%\end{align}
%where $\tau_s$ is the characteristic synchrotron energy loss time, and $1/(\tau_s B^2)$ is independent of the local magnetic field strength. If we follow the method outlined in ecritpaper, we first consider the dynamics near the O-point where the flux in the momentum direction nearly vanishes, so that the pitch-angle distribution is determined by the vanishing of the pitch-angle flux:
%\begin{align}
%f = F(p) \exp\left[ - \frac{\lambda}{p\nu_D \bar{S}(\lambda)}\sigma\Theta G(r) \left\langle\frac{1}{R^2}\right\rangle\frac{e V\sub{loop}}{2\pi} \right].
%\end{align}
%If we consider steady-state and integrate the bounce-averaged kinetic equation over $\lambda$ (taking $\sigma=1$), we obtain 
%\begin{align}
%0 &= \frac{1}{p^2}\frac{\partial }{\partial p}\Big[p^2 U(p; \,V\sub{loop}) \alpha(p;\,V\sub{loop}) F(p)\Big], \nonumber \\
%U &= -\Theta G(r) \left\langle\frac{1}{R^2}\right\rangle\frac{e V\sub{loop}}{2\pi} + p\nu_s \bar{L}(\lambda) + \frac{\beta(p;\,V\sub{loop})}{\alpha(p;\,V\sub{loop})}\frac{p\gamma}{B^2\tau_s}\bar{H}(\lambda) \\
%\alpha &=  \int_0^{1/B\sub{min}}\rd \lambda \,  \exp\left[ - \frac{\lambda}{p\nu_D \bar{S}(\lambda)}\sigma\Theta G(r) \left\langle\frac{1}{R^2}\right\rangle\frac{e V\sub{loop}}{2\pi} \right] \nonumber \\
%\beta &= \int_0^{1/B\sub{min}}\rd \lambda \,  \lambda \exp\left[ - \frac{\lambda}{p\nu_D \bar{S}(\lambda)}\sigma\Theta G(r) \left\langle\frac{1}{R^2}\right\rangle\frac{e V\sub{loop}}{2\pi} \right] \nonumber.
%\end{align}
%There is a minimum value of $V\sub{loop}$ for which $U$ has real zeros in $p$; this loop voltage defines the critical effective field (voltage).
%




\subsection{Analytic runaway generation rate from source function $S$}
Runaway generation that is not due to Dreicer or hot tail mechanisms are generated by the source function $S$, which represents for example large-angle collisions, beta decay of tritium or Compton scattering on gamma rays. These can be evaluated if we assume quasi steady state, where we order radial transport and the time derivative as even higher order, i.e. $\partial/\partial t \sim A^r \sim D^{rr} \sim \delta^3$. In that case, the second-order equation takes the form
\begin{align}
\frac{1}{p^2}\frac{\partial}{\partial p}\left[ p^2\left(A^pf + D^{pp}\frac{\partial f}{\partial p}\right)\right] + S &= 0.
\end{align}
With this equation, the runaway generation rate can be defined as the particle flux to infinity,
\begin{align}
\frac{\partial n\sub{RE}}{\partial t} &= -4\pi p^2\left(A^pf + D^{pp}\frac{\partial f}{\partial p}\right)_{\!\!p=\infty}.
\end{align}
If we then integrate the second-order equation from $p$ to $\infty$, we obtain
\begin{align}
 p^2\left(A^pf + D^{pp}\frac{\partial f}{\partial p}\right) = -\frac{1}{4\pi}\frac{\partial n\sub{RE}}{\partial t} + \int_p^\infty \rd p \, p^2S .
\end{align}
Dividing by $p^2 D^{pp}$ and introducing an integrating factor $G$, we obtain the equation
\begin{align}
\frac{\partial e^Gf}{\partial p} &= \frac{ e^G}{p^2 D^{pp}}\left(-\frac{1}{4\pi}\frac{\partial n\sub{RE}}{\partial t} + \int_p^\infty \rd p' \, p'^2S(p')\right), \nonumber \\
G &= -\int_p^\infty \,\frac{A^p(p')}{D^{pp}(p')}\,\rd p'.
\end{align}
If we assume that $f$ is well-behaved so that $e^G f|_{p=\infty} = e^G f|_{p=0} = 0$ (the latter following from the fact that $\lim_{p\to\infty} G = -\infty$), if we integrate the equation over all $p$ from 0 to $\infty$, we obtain an equation for the runaway density:
\begin{align}
\frac{1}{4\pi} \frac{\partial n\sub{RE}}{\partial t} \int_0^\infty \rd p \,\frac{e^G}{p^2 D^{pp}} &= \int_0^\infty \rd p \, \frac{e^G}{p^2 D^{pp}} \int_p^\infty \rd p' \, p'^2S(p') \nonumber \\
&=\int_0^\infty \rd p \, p^2S(p) \int_0^p \rd p' \, \frac{e^{G(p')}}{p'^2 D^{pp}(p')}.
\end{align}
This allows us to give the final expression
\begin{align}
\frac{\partial \langle n\sub{RE}\rangle}{\partial t} &= 4\pi \int \rd p \,p^2S(p)h(p) \nonumber \\
&= \left\langle 2\pi\int_0^\infty \rd p \,p^2\int_{-1}^1 \rd \xi \,S h(p) \right\rangle \nonumber \\
&= \left\langle  \int S(t,\,\b{x},\,\b{p}) h(p) \, \rd\b{p}\right\rangle,  \nonumber \\
h(p) &=  \frac{ \int_0^p \rd p \,\frac{e^G}{p^2 D^{pp}}}{ \int_0^\infty \rd p \,\frac{e^G}{p^2 D^{pp}}} \nonumber \\
&= \frac{ \int_0^p \rd p \,\frac{\nu_D e^G}{p^2}}{ \int_0^\infty \rd p \,\frac{\nu_D e^G}{p^2}}, \nonumber \\
G &= -\int_p^\infty \rd p' \,\frac{A^p}{D^{pp}} \nonumber \\
&= - \frac{4 \int_p^\infty p'\nu_s(p')\nu_D(p') \,\rd p'}{e^2 \langle \b{E}\cdot\b{B}\rangle^2 \int_0^{\lambda_T} \frac{\lambda\rd\lambda}{\langle\sqrt{1-\lambda B}\rangle}},
\end{align}
where we have interpreted $n\sub{RE} = \langle n\sub{RE}\rangle$ (or rather, flux surface averaged the equation again, since $f_0$, and hence its flux to infinity, is uniform on flux surfaces due to isotropy, but beam-like runaways would have $n\sub{RE}\propto B$). An asymptotically matched formula, analogous to the one developed by Linnea, can be obtained by in $G$ replacing $\langle \b{E}\cdot\b{B}\rangle \mapsto \langle (\b{E}-(2\pi)^{-1}V\sub{loop,c}^\text{eff}\nabla\varphi)\cdot\b{B}\rangle$ and $\nu_D\mapsto \nu_D+4\nu_s$, where the critical effective loop voltage is calculated as in section 2.5.

\subsubsection*{Special case: non-relativistic ideal plasma}
In a pure plasma characterized by a density and effective charge $Z$, in the non-relativistic limit, we have
\begin{align}
\nu_D &= eE_c (1+Z)\frac{m_e^2 c^2}{p^3}, \nonumber \\
\nu_s &= eE_c  \frac{m_e^2 c^2}{p^3},
\end{align}
in which case
\begin{align}
G &= -\frac{ 4(1+Z)}{ \langle \frac{\b{E}}{E_c}\cdot\b{B}\rangle^2 \int_0^{\lambda_T} \frac{\lambda\rd\lambda}{\langle\sqrt{1-\lambda B}\rangle}}  \int_p^\infty \frac{1}{(p/m_e c)^5} \,\rd (p/m_ec) \nonumber \\
&= -\left(\left\langle \frac{\b{E}}{E_c}\cdot\b{B}\right\rangle^2 \int_0^{\lambda_T} \frac{\lambda\rd\lambda}{\langle\sqrt{1-\lambda B}\rangle}\right)^{-1}  \frac{1+Z}{p^4}, \nonumber \\
h &= \frac{\int_0^p \rd p \, \frac{e^G}{p^5}}{\int_0^\infty \rd p \, \frac{e^G}{p^5}}  %\nonumber \\ &
= e^{G(p)} %\nonumber \\ &
= \exp\left(-\left(\left\langle \frac{\b{E}}{E_c}\cdot\b{B}\right\rangle^2 \int_0^{\lambda_T} \frac{\lambda\rd\lambda}{\langle\sqrt{1-\lambda B}\rangle}\right)^{-1}  \frac{1+Z}{p^4} \right).
\end{align}
The large-angle collision operator can be integrated exactly with this $h$ in this approximation, which yields the result given in Rosenbluth-Putvinski's paper (Eq 13).

Since $G$ changes very rapidly with $p$, we can approximate $h$ with a step function at the point where $G$ passes $-1$:
\begin{align}
h(p) &\approx \Theta(p-p_\star),\nonumber \\
-1 &= G(p_\star) = -\left(\left\langle \frac{\b{E}}{E_c}\cdot\b{B}\right\rangle^2 \int_0^{\lambda_T} \frac{\lambda\rd\lambda}{\langle\sqrt{1-\lambda B}\rangle}\right)^{-1}  \frac{1+Z}{p_\star^4} ,
\end{align}
yielding
\begin{align}
\frac{1}{p_\star^2} &= \frac{1}{\sqrt{1+Z}} \left\langle \frac{\b{E}}{E_c}\cdot\b{B}\right\rangle \sqrt{ \int_0^{\lambda_T} \frac{\lambda\rd\lambda}{\langle\sqrt{1-\lambda B}\rangle}} \nonumber \\
&= \sqrt{\frac{4}{3}}\frac{E/E_c}{\sqrt{1+Z}} \times (1-0.73\sqrt{\epsilon} + \Ordo(\epsilon))
\end{align}
where the last line includes only the leading-order contribution in finite-aspect ratio (and $E$ varies with $\Ordo(\epsilon)$ along the orbit).

\subsubsection*{General case}
In the general case, we can write
\begin{align}
\nu_D &= eE_c m_e^2 c^2 \bar\nu_D \frac{ \gamma^2}{p^3}, \nonumber \\
\nu_s &= eE_c m_e^2 c^2 \bar\nu_s \frac{ \gamma}{p^3},
\end{align}
for which
\begin{align}
G &=  -\frac{ 4}{ \langle \frac{\b{E}}{E_c}\cdot\b{B}\rangle^2 \int_0^{\lambda_T} \frac{\lambda\rd\lambda}{\langle\sqrt{1-\lambda B}\rangle}}  \int_{p/m_e c}^\infty \bar\nu_s\bar\nu_D\frac{(1+q^2)^{3/2}}{q^5} \,\rd q.
\end{align}
Since the dominant contribution to the integral will be obtained from the region near the lower limit $q \approx p/m_e c$, and since $\bar\nu_s$ and $\bar\nu_D$ are relatively weakly varying with $p$, we can take them outside of the integral and have
\begin{align}
G(p) &\approx -\frac{ 4 \bar\nu_s(p)\bar\nu_D(p)}{ \langle \frac{\b{E}}{E_c}\cdot\b{B}\rangle^2 \int_0^{\lambda_T} \frac{\lambda\rd\lambda}{\langle\sqrt{1-\lambda B}\rangle}} \int_{p/m_e c}^\infty \frac{(1+q^2)^{3/2}}{q^5} \,\rd q \nonumber \\
&= -\frac{  \bar\nu_s(p)\bar\nu_D(p)}{ \langle \frac{\b{E}}{E_c}\cdot\b{B}\rangle^2 \int_0^{\lambda_T} \frac{\lambda\rd\lambda}{\langle\sqrt{1-\lambda B}\rangle}}\frac{1}{p^4}\left[\left(1+\frac{5}{2}p^2\right)\sqrt{1+p^2} + \frac{3}{2}p^4\ln\left(\frac{1+\sqrt{1+p^2}}{p}\right)\right],
\end{align}
where all the $p$'s in this expression are in units of $m_e c$. Then, $p_\star$ is again defined from $G(p_\star)=-1$, which must be determined numerically.



\section{Impurity dynamics}
We shall assume that all background species (all charge states of impurities, cold-electron density and cold-electron heat) are uniformly distributed over the flux surfaces. This is valid when the runaway density is small, since the kinetic hot electrons (which drive ionization) are uniformly distributed in the banana limit that we consider. The runaway density, which is proportional to $B$, is mildly inhomogeneous, and could in principle drive some poloidal asymmetries in the ion charge states when the parallel fluxes are sufficiently small. 

We describe the impurity dynamics by introducing the densities $n_i^{(j)}$ which describe the number density of an ion with atomic number $Z = Z_i$ and charge number $Z_0 = Z_{0j} = j$. Thus, $n\sub{Ar}^{(2)}$ would denote the number density of Ar$^{+2}$. We model the time evolution of the ion densities via
\begin{align}
\frac{\partial n_i^{(j)}}{\partial t} &=\left( I_{i}^{(j-1)}n\sub{cold}+\mathcal{I}_i^{(j-1)}\right)n_i^{(j-1)} - \left(I_i^{(j)}n\sub{cold} + \mathcal{I}_i^{(j)}\right)n_i^{(j)}  \nonumber \\
&+ R_i^{(j+1)} n_i^{(j+1)}n\sub{cold} - R_i^{(j)}n_i^{(j)}n\sub{cold}+ \frac{1}{V'}\frac{\partial}{\partial r}\left[V'\left(-A_i^{(j)}n_i^{(j)}+D_i^{(j)}\frac{\partial n_i^{(j)}}{\partial r}\right)\right],
\end{align} 
where $I_i^{(j)}$ denote the ionization rate coefficients (e.g. from ADAS) due to collisions with the fluid-like cold electrons, $\mathcal{I}_i^{(j)}$ the fast-electron impact ionization coefficient calculated later in this section, and $R_i^{(j)}$ are radiative recombination rates.
We have neglected charge-exchange terms with neutral hydrogenic species, which would take the form $+C_{i}^{(j+1)}n_i^{(j+1)}n\sub{H}^{(0)} - C_{i}^{(j)}n_i^{(j)}n\sub{H}^{(0)} $.
The number density of cold electrons $n\sub{cold}$ must be such that the overall plasma is charge neutral:
\begin{align}
n\sub{cold} = \sum_i \sum_{j=0}^{Z_i} Z_{0j} n_i^{(j)} - n\sub{RE} - \int \rd \b{p} \, f,
\end{align}
where the sum over $i$ is taken over all atomic species present in the plasma.  

\subsection{Heat equation for cold electron population}
The heat transport of the cold fluid electrons is formulated in terms of the evolution of the total energy content $W_c$ of the cold electron population, 
\begin{align}
W_c &= \frac{3}{2}n\sub{cold}T_{e,\text{cold}} + W\sub{binding}, \nonumber \\
W\sub{binding} &= -\sum_i \sum_{j=0}^{Z_i} W_i^{(j)} n_i^{(j)}
\end{align}
where $W_i^{(j)}$ is the total binding energy of the ion $i,j$, and $W\sub{binding}$ represents the total binding energy of the plasma. 

The rate of change of energy is governed by the equation
\begin{align}
\frac{\partial W_c}{\partial t} &= E_\parallel j_\Omega + \int  \frac{p^2}{2m_e}\nu_{E,ee} f \,\rd \b{p}  + n\sub{RE} ceE_c^\text{(cold)} - 
n\sub{cold} \sum_i \sum_{j=0}^{Z_i-1} n_i^{(j)} L_{i}^{(j)}(T\sub{cold},\,n\sub{cold})  \nonumber \\
&+ Q_c +  \frac{1}{V'}\frac{\partial}{\partial r}\left[V'\left(-A_W W_c + D_W\frac{\partial W_c}{\partial r}\right)\right], \nonumber \\
\nu_{E,ee} &= \frac{2 n\sub{cold}\ln\Lambda e^4}{4\pi\varepsilon_0^2 p^2 v} \left[\phi\left(\frac{v}{v\sub{Te}}\right) - 2\frac{v}{v\sub{Te}}\phi'\left(\frac{v}{v\sub{Te}}\right)\right], \nonumber \\
E_c^\text{(cold)} &= \frac{n\sub{cold}\ln\Lambda e^3}{4\pi\varepsilon_0^2 m_e c^2}, \nonumber \\
Q_c &= \sum_j Q_{ej}, \nonumber\\
Q_{ij} &= \frac{2}{3}\frac{m_i}{m_j} \frac{Z_i^2 Z_j^2}{(2\pi)^{3/2}\varepsilon_0^2 m_i^2}\frac{n_iW_j-n_jW_i}{\left(\frac{T_i}{m_i}+\frac{T_j}{m_j}\right)^{3/2}},
\end{align}
plus terms that vanish upon flux surface averaging. $E_c^\text{(cold)}$ here should be evaluated with the Coulomb logarithm characteristic for the runaway population, and $Q_{ij}$ is the collisional energy transfer between Maxwellians of species $i$ and $j$. The sum over $j$ is taken over all ion species, where we assume that the ion heat is governed by 
\begin{align}
\frac{\partial W_i}{\partial t} = -Q_{ei} + \sum_j Q_{ij} +\frac{1}{V'}\frac{\partial}{\partial r}\left[V'\left(-A_W^{(i)} W_i + D_W^{(i)}\frac{\partial W_i}{\partial r}\right)\right].
\end{align}
\textcolor{red}{[We may consider defining the transport so that it is only driven by temperature gradients and not heat gradients (which can be set up by density gradients)]} All impurities of the same atomic number $Z_i$ should be represented by a single $W_i = (3/2)T_i \sum_j n_i^{(j)}$ summed over charge states. 

%We approximate the ionization heat loss $Q_\mathcal{I}$ by summing the ionization rates multiplied by the ionization energy thresholds plus the thermal energy of the cold population:
%\begin{align}
%Q_\mathcal{I} &= \sum_i \sum_{j=0}^{Z_i-1}n_i^{(j)}\left\{  \left[\Delta I_i^{(j)} + \frac{3}{2}T_{e,\text{cold}}\right]  I_i^{(j)}n\sub{cold} +  \frac{3}{2}T_{e,\text{cold}}\mathcal{I}_i^{(j)} \right\} .
%\left[ \left( I_{i}^{(j-1)}n\sub{cold}+\mathcal{I}_i^{(j-1)}\right)n_i^{(j-1)} - \left(I_i^{(j)}n\sub{cold} + \mathcal{I}_i^{(j)}\right)n_i^{(j)}\right].
%\end{align}
%This energy loss assumes that the majority of ionizations by cold electrons occur from the weakest bound valence electrons, and does not account for the deep ionizations. We assume that all newly ionized electrons have negligible energy, so that they must gain an energy $3T_{e,\text{cold}}/2$ as they equilibrate with the Maxwellian population.
%\textcolor{red}{[It is probably more accurate not to keep the $\Delta I_i^{(j)}$ term with the $\mathcal{I}_i^{(j)}$ contributions since that energy is lost by the fast electrons, which is already accounted for in the $\nu_s$ term in the kinetic equation]}

The flux surface average of  the electron heat equation yields, inserting $j_\Omega/B = \sigma \langle \b{E}\cdot\b{B}\rangle/\langle B^2\rangle$, 
\begin{align}
\frac{\partial W_c}{\partial t} &= \sigma \frac{\langle \b{E}\cdot\b{B}\rangle^2}{\langle B^2\rangle}+ \left\langle\int  \frac{p^2}{2m_e}\nu_{E,ee} f \,\rd \b{p}\right\rangle -n\sub{cold} \sum_i \sum_{j=0}^{Z_i-1} n_i^{(j)} L_{i}^{(j)}(T\sub{cold},\,n\sub{cold})  \nonumber \\
&+ \frac{j\sub{RE}}{B}\langle B\rangle E_c^\text{(cold)} + Q_c + Q_\mathcal{I} + \frac{1}{V'}\frac{\partial}{\partial r}\left[V'\left(-A_W W_c + D_W\frac{\partial W_c}{\partial r}\right)\right].
\end{align}
With our reduced isotropic kinetic description of $f$, the collisional energy transfer can be evaluated using only $f_0$ which is constant on flux surfaces, making the flux surface average trivial.
%
%where a flux-surface average of the equation only changes \mbox{$E_\parallel j_\Omega \mapsto (j_\Omega/B)\langle \b{E}\cdot\b{B}\rangle$} and $n\sub{RE} \mapsto (n\sub{RE}/B)\langle B\rangle$ since all other quantities are (assumed to be) uniformly distributed on the flux surface. Note that if the runaways heat the cold electrons on a time scale faster than the parallel heat conductivity, you end up with some poloidal asymmetries in the cold-electron temperature and ionization states which we have neglected in the fast-electron collision operator (since $\nu_s$ and $\nu_D$ were taken to be constant on flux surfaces).
%
Since the mean speed of $f$ is much greater than the cold thermal speed $v\sub{Te}$ in most (all?) situations of interest, we can write
\begin{align}
\frac{p^2}{2m}\nu_{E,ee} &\approx \frac{n\sub{cold}\ln\Lambda e^4}{4\pi\varepsilon_0^2 m_e v} = ceE_c^\text{(cold)}\frac{c}{v},
\end{align}
so that the collisional energy transfer between hot electrons and free cold electrons is \textcolor{red}{[this could be revised with the proper relativistic stopping power, but it is probably not going to turn out any different]}
\begin{align}
\left\langle \int  \frac{p^2}{2m_e}\nu_{E,ee} f \,\rd \b{p} \right\rangle &\approx \left\langle ceE_c^\text{(cold)} \int \frac{c}{v}f\,\rd\b{p}\right\rangle\nonumber \\
 &\approx ceE_c^\text{(cold)}4\pi \int_0^{p\sub{max}} \rd p  \, mcp f_0(p)
\end{align}

%The heat transport could possibly be calculated from the transport model assumed for the electron populations, but I'm not sure if that makes sense since the cold electron particle transport does not follow that model; why would the heat?

Finally, the radiated power from atomic-physics processes $L_i^{(j)}$ denotes line radiated power (due to excitations) as well as recombination radiation. To better understand the role of the binding potential term, we can express its effective cooling rate as
\begin{align}
P\sub{binding} &= -\frac{\partial W\sub{binding}}{\partial t} = \sum_i \sum_{j=0}^{Z_i} W_i^{(j)} \frac{\partial n^{(j)}}{\partial t}\nonumber \\
&=  \sum_i \sum_{j=0}^{Z_i} W_i^{(j)} \left[ \mathcal{I}_i^{(j-1)}n_i^{(j-1)} - \mathcal{I}_i^{(j)}n_i^{(j)} + n\sub{cold} R_i^{(j+1)}n_i^{(j+1)} -n\sub{cold} R_i^{(j)}n_i^{(j)} \right] \nonumber \\
&= \sum_i \sum_{j=0}^{Z_i-1} (W_i^{(j+1)}-W_i^{(j)})\mathcal{I}_i^{(j)}n_i^{(j)} -  n\sub{cold}\sum_i \sum_{j=1}^{Z_i} (W_i^{(j)} - W_i^{(j-1)})R_i^{(j)}n_i^{(j)} ,
\end{align}
where we let $\mathcal{I}_i^{(j)}$ denote the net ionisation rate due to all mechanisms, and the same for the recombination $R_i^{(j)}$. Now, $W_i^{(j+1)}-W_i^{(j)} = E_i^{(j)}$ denotes the ionisation threshold for ion $(i,j)$, and we recognize the first term as the ionisation energy loss. The second term is (potential) energy gain due to the recombination, but which will cancel against the recombination radiation term if the recombination is radiative. Collisional three-body recombination would not cause radiation, and would correctly yield an increase in the kinetic energy of the plasma.
Note that ``external'' ionisation (say, by fast electrons) will cause a loss of kinetic energy of the cold electron population, unless a source term is added to the equation which describes the energy transfer.

\subsubsection*{Mini appendix: collisional heat exchange expression}
The collisional heat exchange between different Maxwellian populations is given by (Hinton 1983)
\begin{align}
\left(\frac{\partial W_i}{\partial t}\right)_c &= \sum_j Q_{ij}, \nonumber\\
Q_{ij} &= \frac{3}{\tau_{ij}}\frac{m_i}{m_i+m_j}n_i(T_j-T_i), \nonumber\\
\frac{1}{\tau_{ij}} &= \frac{4}{3\sqrt{\pi}}\frac{m_i+m_j}{m_j} n_j Z_j^2\Gamma_i \frac{1}{2^{3/2}\left(\frac{T_i}{m_i}+\frac{T_j}{m_j}\right)^{3/2}}, \nonumber\\
\Gamma_i &= \frac{Z_i^2 e^4 \ln\Lambda}{4\pi\varepsilon_0^2 m_i^2}.
\end{align}
We can put the pieces together as
\begin{align}
Q_{ij} &=\frac{m_i}{m_j} \frac{n_i n_j Z_i^2 Z_j^2}{(2\pi)^{3/2}\varepsilon_0^2 m_i^2}\frac{T_j-T_i}{\left(\frac{T_i}{m_i}+\frac{T_j}{m_j}\right)^{3/2}} \nonumber \\
&= \frac{2}{3}\frac{m_i}{m_j} \frac{Z_i^2 Z_j^2}{(2\pi)^{3/2}\varepsilon_0^2 m_i^2}\frac{n_iW_j-n_jW_i}{\left(\frac{T_i}{m_i}+\frac{T_j}{m_j}\right)^{3/2}}.
\end{align}
In the limit $T_i/m_i \gg T_j/m_j$, i.e. when the ``other'' species $j$ is much heavier, Helander gives the expression
\begin{align}
Q_{ij} \approx \frac{m_i}{m_j}\frac{n_in_j Z_j^2 e^4\ln\Lambda}{(2\pi)^{3/2}\varepsilon_0^2 \sqrt{m_i}T_i^{3/2}},
\end{align}
which we see is in agreement with the general formula when $Z_i=1$ (Helander considered the $i$ species as electrons). 

\subsection{Fast electron impact ionization}
\textcolor{red}{[when the runaways have a large influence on the ionization states, they will not be distributed evenly across the flux surface since the recombination time can be much shorter than the cold-electron transit time -- we neglect this scenario and assume that the charge states are the flux-surface-averaged values everywhere]}

For electron impact ionization by the kinetic electrons, we will use the Garland model for the ionization cross section. We ignore fine structure within the shells (which split the p-shell energy levels by $\sim 1\%$).

The electron-impact ionization cross section of a bound electron in subshell $\alpha$ in an ion with charge number $Z_0$ is given by
\begin{align}
\sigma_{Z_0,\alpha \to  Z_0+1}& = (1-S(p))\sigma_{Z_0,\alpha \to  Z_0+1}^\text{NR} + S(p) \sigma_{Z_0,\alpha\to Z_0+1}^\text{R}, \nonumber \\
S(p) &= \frac{1}{1+ \exp[ 1 - (\sqrt{m_e^2 c^4+c^2 p^2}-m_e c^2)/(10^5\,\text{eV})]}, \nonumber \\
\sigma_{Z_0,\alpha\to Z_0+1}^\text{NR} &= \pi a_0^2 C \mathcal{N}_\alpha \left(\frac{\text{Ry}}{\Delta I_Z^\alpha}\right)^2\frac{(\ln U)^{1+\beta^\star/U}}{U} \nonumber \\
U &= \frac{p^2}{2m\Delta I_Z^\alpha}, \nonumber \\
\beta^\star &= 0.25\sqrt{\frac{100Q_n+91}{4Q_n+3}}-1.25, \nonumber \\
Q_n &=  Z - \sum_{m<n} \sigma(n,m) P_m - \frac{1}{2}\sigma(n,n)\text{max}(0,\,P_n-1), \nonumber \\
%\Delta I_Z^i &= \frac{Q_i^2 \text{Ry}}{i^2} \left[ 1+  \left(\frac{\alpha Q_n}{i}\right)^2\left(\frac{2i}{i+1}-\frac{3}{4}\right)\right] ,\nonumber  \\
\sigma_{Z_0,\alpha\to Z_0+1}^\text{R} &= \pi r_0^2 C \mathcal{N}_\alpha\left(\frac{\text{Ry}}{\Delta I_Z^\alpha}\right) \left[\ln\left(\frac{p^2}{2m\Delta I_Z^\alpha}\right) - \frac{p^2}{\sqrt{m^2c^2+p^2}}\right],
\end{align}
where $\text{Ry} \approx 13.6\,$eV is the Rydberg energy, $C \approx 2$ an undetermined constant, $\mathcal{N}_\alpha$ the number of electrons in subshell $\alpha$ (for neutral argon, $\mathcal{N}_1 = 2$ for the two electrons in the 1s state, $\mathcal{N}_2 = 2$ for the two electrons in 2s, $\mathcal{N}_3=6$ for the 2p shell, $\mathcal{N}_4 = 2$ and $\mathcal{N}_5=6$), and $\Delta I_Z^\alpha$ is the ionization threshold energy for the subshell $i$; this data for the atomic binding energy of electrons in each shell of all ionization states of impurities from $Z=1$ to $Z=30$ is given in [W Lotz, \emph{Subshell Binding Energies of Aoms and Ions from Hydrogen to Zinc}, Journal of the Optical Society of America {\bf 58} (1968)]. This model uses the simpler hydrogenic shell model for the effective charge number $Q_n$ of the shell with principal quantum number $n$ and corresponding occupation number $P_n$ (where, for neutral argon, $P_1=2$, $P_2 = 8$ and $P_3 = 8$), such that the total energy of the ion is approximatetly $E\sub{ion}(\text{Ry}) \approx - \sum_i Q_n^2\mathcal{N}_n/n^2$. In the expression for $Q_n$, $\sigma(n,m)$ are hydrogenic screening constants which are given in [R Marchand, S Caille. ``Improved screening coefficients for the hydrogenic ion model''. Journal of Quantitative Spectroscopy and Radiative Transfer {\bf 43} (1990)], and presented in the table below.
%which are supposed to be published in [R M More. "Applied Atomic Collision Physics." Vol. II (Massey, 1983) (1982)], but the reference does not appear to be available. Screening constants given by [H Mayer. ``Methods of Opacity Calcutions'' (1949)] are supposedly less accurate for nearly neutral ions, but are available and given in the table below for the first three shells (covers all ions up to argon). A model more recent than More has been given in [R Marchand, S Caille. ``Improved screening coefficients for the hydrogenic ion model''. Journal of Quantitative Spectroscopy and Radiative Transfer {\bf 43} (1990)]. I can't find the ref through the library, but I think this is the one that Garland is using. 

The ionization rate coefficient for fast-electron impact ioniztion is finally given by
\begin{align}
\mathcal{I}_i^{(j)} = \int \rd\b{p}\,vf \sum_\alpha \sigma_{i,\alpha, Z_{0j}  \to  Z_{0j}+1}(p),
\end{align}
where we sum the cross section over all subshells $\alpha$ of the ion species $n_i^{(j)}$.

The runaway contribution to $\mathcal{I}_i^{(j)}$ may be approximated by evaluating the cross sections at a characteristic mean energy $p_0$ of the runaway distribution, multiplied by $n\sub{RE}c$. It will probably not be impactful to replace the actual average RE energy by the asymptotic average RE energy $p_0 \approx eE/\Gamma\sub{ava}(E)$.


%\begin{center}
\begin{table}[h]
\caption{Screening constants $\sigma(i,j)$ given by R Marchand \& S Caille (1990). \vspace{-6mm}}
\begin{center}
\begin{tabular}{|c|c|c|c|c|c|c|c|c|}
\hline 
\diagbox{i}{j}  & 1 & 2 & 3 & 4 & 5 \\
\hline
1 & 0.5966 &  &  & & \\
\hline
2  & 0.8597  & 0.6888  &  & & \\
\hline
3 & 0.9923  & 0.8877 & 0.7322  & & \\
\hline
4 & 0.9800  & 0.9640 & 0.9415 & 0.6986 & \\
\hline
5 & 0.9725  & 1.0000 & 0.9897  & 0.8590 & 0.8502 \\
\hline
\end{tabular}
\end{center}
\end{table}
%\end{center}


%%\begin{center}
%\begin{table}[h]
%\caption{Screening constants $\sigma(i,j)$ given by H Mayer (1949). Not very accurate for weakly ionized ions, and getting our hands on the R M More or (better yet) R Marchand tables would probably be useful. Higher-shell data is available if we would need it. \vspace{-6mm}}
%\begin{center}
%\begin{tabular}{|c|c|c|c|c|c|c|c|c|}
%\hline 
%i/j  & 1s & 2s & 2p \\
%\hline
%1s & 0.6250 & 0.2099& 0.2428 \\
%\hline
%2s  & 0.8395  & 0.6016 & 0.6484  \\
%\hline
%2p &0.9712   & 0.6484 & 0.7266 \\
%\hline
%\end{tabular}
%\end{center}
%\end{table}
%%\end{center}



\section{Poloidal flux and electric field}
In an axisymmetric geometry, the poloidal flux (closely related to the toroidal vector potential) evolves in time according to
\begin{align}
\frac{\partial \psi}{\partial t} = V\sub{loop} = 2\pi \frac{\langle \b{E}\cdot\b{B} \rangle }{\left\langle \b{B}\cdot\nabla \varphi \right\rangle } .
\label{eq:poloidal flux evolution}
\end{align}

From Ampere's law, it also follows that the poloidal flux $\psi$ is related to the plasma current $I$ via
\begin{align}
(2\pi)^2 \mu_0 I(t,\,r) &= V'\left\langle \frac{|\nabla r|^2}{R^2}\right\rangle \frac{\partial \psi}{\partial r},
\end{align}
or, by differentiating with respect to $r$ and writing it on divergence form,
\begin{align}
2\pi \mu_0 \frac{j_\parallel}{B} \langle \b{B}\cdot\nabla\varphi\rangle = \frac{1}{V'}\frac{\partial }{\partial r}\left[V'\left\langle \frac{|\nabla r|^2}{R^2}\right\rangle \frac{\partial \psi}{\partial r}\right].
\end{align}
The time derivative of this equation reduces to the $E$-field diffusion equation similar to the one solved by GO.

If there is a vacuum region, i.e.~a radius $a$ for which $j_\parallel(r\geq a) = 0$, the poloidal flux will take the solution 
\begin{align}
\psi(r\geq a) = \psi(a) + (2\pi)^2\mu_0 I(t,\,a) \int_a^r \frac{\rd r}{V'\left\langle \frac{|\nabla r|^2}{R^2}\right\rangle },
\end{align}
and in this region the loop voltage is given by inserting this  into Eq.~(\ref{eq:poloidal flux evolution}).

\subsection{Safety factor $q$}
The safety factor $q$ is defined as
\begin{align}
q &= \frac{\langle \b{B}\cdot\nabla\varphi\rangle}{\langle \b{B}\cdot\nabla\theta\rangle},
\end{align}
where
\begin{align}
\langle \b{B}\cdot\nabla\theta \rangle &= \frac{1}{V'} \int_0^{2\pi} \rd \varphi \int_0^{2\pi}\rd\theta \,\mathcal{J} \b{B}\cdot\nabla \theta \nonumber \\
&=\frac{1}{V'}\frac{\partial \psi}{\partial r} \int_0^{2\pi} \rd \varphi \int_0^{2\pi}\rd\theta  \nonumber \\
&= \frac{(2\pi)^2}{V'}\frac{\partial \psi}{\partial r}.
\end{align}
From the relation between poloidal flux and plasma current, we can solve for
\begin{align}
\frac{\partial \psi}{\partial r} &= \frac{(2\pi)^2 \mu_0 I(t,\,r)}{V'\langle\frac{|\nabla r|^2}{R^2}\rangle},
\end{align}
which yields an expression for the safety factor $q$ as
\begin{align}
q(t,\,r) &= \frac{V'\langle \b{B}\cdot\nabla\varphi\rangle}{(2\pi)^2 \mu_0 I(t,\,r)}V'\left\langle\frac{|\nabla r|^2}{R^2}\right\rangle.
\end{align}

\subsection{Electric field and Ohm's law}
Given the poloidal flux, the plasma current density can be calculated as above. The current density in term defines the loop voltage via an Ohm's law,% which we express in terms of $I' = \partial I_p/\partial r$ as
\begin{align}
%I' = I'_\Omega + I'\sub{fast} + I'\sub{RE},
\frac{j_\parallel}{B} = \left(\frac{j_\parallel}{B}\right)_\Omega + \left(\frac{j_\parallel}{B}\right)\sub{fast} + \left(\frac{j_\parallel}{B}\right)\sub{RE}.
\end{align}
where
\begin{align}
\frac{j_\Omega}{B} &= \sigma \frac{\langle \b{E}\cdot\b{B} \rangle}{\langle B^2\rangle}, \nonumber \\
\frac{j_\parallel}{B} &= -\pi  e\langle \b{E}\cdot\b{B}\rangle \int_0^{\lambda_T}\frac{\lambda\,\rd \lambda}{\langle\sqrt{1-\lambda B}\rangle}\int \rd p \, \frac{vp^2}{\nu_D}\frac{\partial f_0}{\partial p}, \nonumber\\
\frac{j\sub{RE}}{B} &= \frac{ecn\sub{RE}}{B}.
\end{align}

The expression for $j\sub{fast}/B$ can, however, exceed the value that would be obtained if all particles moved with $\xi=1$, which is $j=\int vf \,\rd\b{p}$, or:
\begin{align}
\left(\frac{j}{B}\right)\sub{max} &= \frac{4\pi e}{B} \int_0^\infty \rd p \,vp^2 f  \nonumber \\
&\leq  \frac{4\pi e}{B\sub{min}} \int_0^\infty \rd p \,vp^2 f 
\end{align}
%\textcolor{red}{[is the bounding correct here? feels a bit counter-intuitive]} \\
In order for the current density never to exceed this value, we introduce a cutoff $p\sub{cut}$ as
\begin{align}
\frac{j\sub{fast}}{B} &=-\pi  e\langle \b{E}\cdot\b{B}\rangle \int_0^{\lambda_T}\frac{\lambda\,\rd \lambda}{\langle\sqrt{1-\lambda B}\rangle}\int_0^{p\sub{cut}} \rd p \, \frac{vp^2}{\nu_D}\frac{\partial f_0}{\partial p}, \nonumber \\
&+\frac{4\pi e }{B\sub{min}} \int_{p\sub{cut}}^\infty \rd p \,vp^2f,
\end{align}
where the cut-off $p\sub{cut} = p\sub{cut}(V\sub{loop},\,n,\,...)$ is chosen so that the integrand is continuous, and must in general be determined iteratively so that the resulting total current is the desired value (based on the poloidal flux profile which is fixed in a given time step).

In the numerical solution for $p\sub{cut}$, it can be useful to give the partial derivative
\begin{align}
\frac{\partial}{\partial p\sub{cut}} \frac{j_\parallel}{B} &= -\pi e c \frac{p\sub{cut}^3}{\sqrt{m_e^2 c^2+p\sub{cut}^2}} \biggl( %\nonumber \\ &
\int_0^{\lambda_T}\frac{\lambda\,\rd \lambda}{\langle\sqrt{1-\lambda B}\rangle} \frac{\langle\b{E}\cdot\b{B}\rangle  }{\nu_D(p\sub{cut})}\frac{\partial f_0}{\partial p}(p\sub{cut}) 
%& \hspace{70mm}
 + \frac{f(p\sub{cut})  }{B\sub{max}}\biggr)
\end{align}

\subsection{Boundary condition and wall current}
Following [A~Boozer, Rev Mod Phys {\bf 76} (2004), Eq (122) + above], the poloidal flux $\psi_w = \psi(r=r_w)$ at the wall satisfies the circuit equation
\begin{align}
\frac{\partial \psi_w}{\partial t} &= R_w I_w
\end{align}
where $R_w$ is the wall resistivity and $I_w$ the current flowing through the wall. The poloidal flux at the wall can in turn be expressed in terms of the external inductances $L_w^\text{(ext)}$ and $L_p^\text{(ext)}$ and for the wall and plasma, respectively, as
\begin{align}
\psi_w = -(L_w^\text{(ext)} I_w + L_p^\text{(ext)} I_p),
\end{align}
where we assume that 
\begin{align}
L_w^\text{(ext)} \approx L_p^\text{(ext)} \approx \mu_0 R_0 \ln\frac{R_0}{r_w},
\end{align}
with $R_0$ is the plasma major radius, where the expression can be obtained by integrating Ampere's law for a circular plasma over radius from $r_w$ to $R_0$. A characteristic wall time $\tau_w$ is given by the ratio
\begin{align}
\tau_w = \frac{L_w^\text{(ext)}}{R_w}.
\end{align}

\subsection{Hyperresistivity and magnetic helicity}
Boozer (Pivotal issues on relativistic electrons in ITER, NF 58 (2018)) gives the mean-field equation for the poloidal flux in a non-axisymmetric system as
\begin{align}
\frac{\partial \psi}{\partial t} = V\sub{loop} + \frac{\partial}{\partial \psi_t}\left(\psi_t \Lambda \frac{\partial^2 I}{\partial \psi_t^2}\right),
\end{align}
where the toroidal flux is the quantity satisfying $\partial \psi_t/\partial \psi = q$, conventionally defined such that $\psi_t(r=0) = 0$, allowing us to write (with previous definitions) 
\begin{align}
\frac{\partial \psi_t}{\partial r} &= q\frac{\partial \psi}{\partial r} = \frac{V'\langle \b{B}\cdot\nabla\varphi \rangle}{(2\pi)^2}, \nonumber \\
\psi_t(r) &= \int q \,\rd \psi = \int_0^r \frac{\partial \psi}{\partial r}q\,\rd r \nonumber \\
&= \frac{1}{(2\pi)^2} \int_0^r V'\langle \b{B}\cdot\nabla\varphi\rangle\,\rd r
\end{align}
so that
\begin{align}
\frac{\partial I_p}{\partial \psi_t} = 2\pi \frac{j_\parallel}{B},
\end{align}
and finally 
\begin{align}
\frac{\partial \psi}{\partial t} = V\sub{loop} + \frac{(2\pi)^5}{ V'\langle \b{B}\cdot\nabla\varphi\rangle}\frac{\partial}{\partial r}\left( \frac{\psi_t\Lambda(t,\,r)}{V'\langle \b{B}\cdot\nabla\varphi\rangle}  \frac{\partial}{\partial r}\frac{j_\parallel}{B}\right).
\end{align}

%The magnetic helicity $K$ is defined by
%\begin{align}
%K(t) &= \int \b{A}\cdot\b{B} \,\rd \b{x},
%\end{align}
%which in an axisymmetric system can be written
%\begin{align}
%K &= \int (\psi_t - q\psi)\,\rd \psi = \int \,\frac{\partial \psi}{\partial r}(\psi_t - q\psi) \,\rd r,
%\end{align}
%where $\psi_t$ is the toroidal flux and $\psi$ the poloidal flux as before. Since the toroidal flux is the quantity satisfying $\partial \psi_t/\partial \psi = q$, which allows us to write
%\begin{align}
%\psi_t(r) &= \psi_t(r=0) + \int q \,\rd \psi = \psi_t(r=0) + \int_0^r \frac{\partial \psi}{\partial r}q\,\rd r \nonumber \\
%&=\psi_t(0) + \frac{1}{(2\pi)^2} \int_0^r G(r) \left\langle\frac{1}{R^2}\right\rangle \,\rd r
%\end{align}



\subsection{TODO: Self-consistent equilibrium}
I think that it would be neat to investigate how $\psi(r)$, $j(r)$, $G(r)$ as well as the shape parameters $\delta$, $\kappa$, $\Delta$ can be chosen in a way that makes the equilibrium consistent with the Grad-Shafranov equation, or at least approximately so. Now there are a bit too many free parameters in the model that aren't ``free'' in reality. %, and it would be interesting to be able to say something about stability of the beam.

We could at least do something like Freidberg's ``One size fits all'' paper, and set $\delta$, $\kappa$ and $\Delta$ on the plasma boundary for $t=0$, and then obtain an equilibrium that is consistent with, at least, the total initial plasma current \footnote{In principle you get the full radial current profile from this treatment, I think, but I don't think we can easily choose a plasma boundary that reproduces any desired plasma current. Down the line you could probably write an iterative method that finds the optimum $\delta$, $\kappa$ and $\Delta$ for the boundary that gives a good fit to the initial plasma profile -- however you would also need to start playing around with pressure profiles which feels like a slippery slope}. From this equilibrium $\psi(R,\,z)$ we can fit $\delta(r)$, $\kappa(r)$ and $\Delta(r)$ for all radii.

For example, at least, from Ampere's law it follows that the function $G(r)$ is determined by
\begin{align}
\frac{\partial G}{\partial r} &= \mu_0\frac{j_\parallel}{B}\frac{\partial \psi}{\partial r},
\end{align}
where, typically, the plasma current carried by non-runaway electrons is directly proportional to $G(r)$, which you would probably want to account for in a fully self-consistent treatment. Otherwise, if you just have a prescribed current profile, you can insert this into the relation between poloidal flux and current given previously:
\begin{align}
\mu_0 \frac{j_\parallel}{B} G(r) V'\left\langle \frac{1}{R^2}\right\rangle &= \frac{\partial }{\partial r}\left[V'\left\langle \frac{|\nabla r|^2}{R^2}\right\rangle \frac{\partial \psi}{\partial r}\right] \nonumber \\
&=\frac{1}{\mu_0}\frac{\partial }{\partial r}\left[V'\left\langle \frac{|\nabla r|^2}{R^2}\right\rangle \frac{\partial G/\partial r}{j_\parallel/B}\right] 
%\frac{\partial \psi}{\partial r} &= \frac{2\pi\mu_0}{\left\langle\frac{|\nabla r|^2}{R^2}\right\rangle}I_p \nonumber \\
%&=  \frac{\mu_0}{ \left\langle\frac{|\nabla r|^2}{R^2}\right\rangle} \int_0^r \frac{j_\parallel}{B}G(r)\left\langle\frac{1}{R^2}\right\rangle \, \rd r
\end{align}
which takes the form of a second-order linear ODE in $G$, with the current profile appearing in the coefficients.

\newpage
\appendix

\section{Kinetic equation advection and diffusion terms}
\label{app:kinetic equation terms}
To aid the bounce averaging -- as in Eq.~(\ref{eq:general bounce average}) -- we can give the components of the various transport terms that appear in the kinetic equation.

\subsubsection*{Electric field}
The electric field corresponds to an advection term $\b{A}_E$ which is given by 
\begin{align}
\b{A}_E &= eE \frac{\partial \b{p}}{\partial p_\parallel}  \nonumber \\
&=eE \hat{p}_\parallel.
\end{align}
In the various (orbit-constant) coordinate systems, we can give:

\paragraph{$p$, $\lambda$:}
\begin{align}
A_E^p &= eE \xi, \nonumber \\
A_E^\lambda &= - \frac{2}{p}\frac{1-\xi^2}{B} eE\xi \nonumber \\
&= -\frac{2}{p}\lambda eE\xi. 
\end{align}

\paragraph{$p$, $\xi_0$:}
\begin{align}
A_E^p &= eE \xi, \nonumber \\
A_E^{\xi_0} &=  \frac{1-\xi_0^2}{p\xi_0} eE\xi 
\end{align}

\paragraph{$p_{\parallel,0}$, $p_{\perp,0}$:}
\begin{align}
A_E^{p_{\parallel,0}} &= eE\left( \xi_0\xi + \frac{B\sub{min}}{B}\frac{\xi}{\xi_0} (1-\xi^2) \right), \nonumber \\
&=eE \frac{\xi}{\xi_0} = eE \frac{\sqrt{p_{\parallel,0}^2+p_{\perp,0}^2}}{p_{\parallel,0}}\xi,\nonumber \\% \left( \xi_0 + (1-\xi_0^2)/\xi_0  \right) \nonumber \\
A_E^{p_{\perp,0}} &= 0. %eE\left(  \sqrt{1-\xi_0^2}\xi - \frac{B\sub{min}}{B}\frac{p\xi_0}{\sqrt{1-\xi_0^2}}\frac{\xi}{\xi_0}\frac{1-\xi^2}{p}\right) \nonumber \\
%&= 0.%eE\left(  \sqrt{1-\xi_0^2}\xi - \xi \sqrt{1-\xi_0^2}\right)
\end{align}



\subsubsection*{Friction term}
The isotropic drag force corresponds to an advection term $\b{A}_F$ which is given by
\begin{align}
\b{A}_F &= -p\nu_s\frac{\partial \b{p}}{\partial p} \nonumber \\
&= -p\nu_s \hat{p},
\end{align}
which may also include bremsstrahlung losses. The components of this term, in the different coordinate systems, are given by

\paragraph{$p$, $\lambda$:}
\begin{align}
A_F^p &= -p\nu_s \nonumber \\
A_F^\lambda &=  0. 
\end{align}

\paragraph{$p$, $\xi_0$:}
\begin{align}
A_F^p &= -p\nu_s, \nonumber \\
A_F^{\xi_0} &=  0.
\end{align}

\paragraph{$p_{\parallel,0}$, $p_{\perp,0}$:}
\begin{align}
A_F^{p_{\parallel,0}} &= -p_{\parallel,0}\nu_s \nonumber \\
A_F^{p_{\perp,0}} &= -p_{\perp,0}\nu_s.
\end{align}


\subsubsection*{Pitch-angle scattering term}
The pitch-angle scattering term corresponds to a diffusion term $\mathsf{D}_d$ which is given by
\begin{align}
\mathsf{D}_d &=  \frac{\nu_D}{2} p^2\left(\mathsf{I}  - \hat{p}\hat{p}\right).
\end{align} 
The non-vanishing contravariant components of this tensor are, in the various coordinate systems:


\paragraph{$p$, $\lambda$:}
\begin{align}
\mathsf{D}_d^{\lambda \lambda} &=  \frac{\nu_D}{2}p^2 |\nabla_\b{p}\lambda|^2 \nonumber \\
&= \nu_D \frac{2\xi^2}{B^2}(1-\xi^2) \nonumber \\
&= 2\nu_D \lambda \frac{1-\lambda B}{B}
\end{align}

\paragraph{$p$, $\xi_0$:}
\begin{align}
\mathsf{D}_d^{\xi_0 \xi_0} &=  \frac{\nu_D}{2}p^2 |\nabla_\b{p}\xi_0|^2  \nonumber \\
&= \frac{\nu_D}{2} \frac{B\sub{min}^2}{B^2}\frac{\xi^2}{\xi_0^2}(1-\xi^2) \nonumber \\
&= \frac{\nu_D}{2} \frac{B\sub{min}}{B}\frac{1-\xi_0^2}{\xi_0^2}\xi^2 \nonumber \\
&= \frac{\nu_D}{2}\frac{1-\xi_0^2}{\xi_0^2} \left[\xi_0^2 - \left(1-\frac{B\sub{min}}{B}\right)\right] \nonumber \\
&= \frac{\nu_D}{2}(1-\xi_0^2) - \frac{\nu_D}{2}\frac{1-\xi_0^2}{\xi_0^2}\left(1-\frac{B\sub{min}}{B}\right)
\end{align}

\paragraph{$p_{\parallel,0}$, $p_{\perp,0}$:}
Here, it is useful to first give
\begin{align}
\nabla p_{\parallel,0} \cdot (\mathsf{I}-\hat{p}\hat{p}) &= \frac{B\sub{min}}{B}p\frac{\xi}{\xi_0}\nabla_\b{p}\xi \nonumber \\ %\nabla p_{\parallel,0} - \xi_0\hat{p} \nonumber \\ 
%&= \left(\frac{1-\xi_0^2}{\xi_0}-\frac{B\sub{min}}{B}\right)\hat{p} + \frac{B\sub{min}}{B}\frac{\xi}{\xi_0}\hat{p}_\parallel \nonumber \\
%&= \frac{B\sub{min}}{B}\left[ \left(\frac{1-\xi^2}{\xi_0} - 1\right)\hat{p} + \frac{\xi}{\xi_0}\hat{p}_\parallel\right] \nonumber \\
\nabla p_{\perp,0} \cdot (\mathsf{I}-\hat{p}\hat{p}) &= -\frac{B\sub{min}}{B}\frac{p\xi}{\sqrt{1-\xi_0^2}}\nabla_\b{p}\xi.
\end{align}
With these, we can give 
\begin{align}
\mathsf{D}_d^{p_{\parallel,0}p_{\parallel,0}} &= \frac{\nu_D}{2} p^2 \frac{B\sub{min}^2}{B^2}p^2\frac{\xi^2}{\xi_0^2}|\nabla_\b{p}\xi|^2 \nonumber \\
&= \frac{\nu_D}{2}p^2\frac{B\sub{min}^2}{B^2} \frac{\xi^2(1-\xi^2)}{\xi_0^2} \nonumber \\
&= \frac{\nu_D}{2}\frac{B\sub{min}}{B}\frac{\xi^2}{\xi_0^2}p_{\perp,0}^2 \nonumber \\
%&=  \frac{\nu_D}{2}\frac{B\sub{min}}{B} p_{\perp,0}^2\frac{1-\frac{B}{B\sub{min}}(1-\xi_0^2)}{\xi_0^2}  \nonumber \\
&=  \frac{\nu_D}{2}p_{\perp,0}^2-  \frac{\nu_D}{2}\frac{p_{\perp,0}^2(p_{\parallel,0}^2+p_{\perp,0}^2)}{p_{\parallel,0}^2}\left(1-\frac{B\sub{min}}{B}\right)  \nonumber \\
\mathsf{D}_d^{p_{\perp,0}p_{\perp,0}} &= p^2 \frac{\nu_D}{2}\frac{B\sub{min}^2}{B^2} \xi^2\frac{1-\xi^2}{1-\xi_0^2} \nonumber \\
&= \frac{\nu_D}{2}\frac{B\sub{min}}{B}\frac{\xi^2}{\xi_0^2}p_{\parallel,0}^2, \nonumber \\
%&= p^2  \frac{\nu_D}{2}\left[\frac{B\sub{min}}{B}-(1-\xi_0^2)\right] \nonumber \\
&= \frac{\nu_D}{2}p_{\parallel,0}^2 -  \frac{\nu_D}{2}(p_{\parallel,0}^2+p_{\perp,0}^2) \left(1-\frac{B\sub{min}}{B}\right), \nonumber \\
\mathsf{D}_d^{p_{\parallel,0}p_{\perp,0}} = \mathsf{D}_d^{p_{\perp,0}p_{\parallel,0}} &= -p^2 \frac{\nu_D}{2}\frac{B\sub{min}^2}{B^2}\frac{\xi^2}{\xi_0\sqrt{1-\xi_0^2}}(1-\xi^2) \nonumber \\
&= -\frac{\nu_D}{2}\frac{B\sub{min}}{B}\frac{\xi^2}{\xi_0^2}p_{\parallel,0}p_{\perp,0} \nonumber \\
%&= -p^2 \frac{\nu_D}{2}\frac{B\sub{min}}{B}\frac{\sqrt{1-\xi_0^2}}{\xi_0}\left[1-\frac{B}{B\sub{min}}(1-\xi_0^2)\right] \nonumber \\
&=- \frac{\nu_D}{2} p_{\perp,0}p_{\parallel,0} + \frac{\nu_D}{2} \frac{p_{\perp,0}}{p_{\parallel,0}}\left(1-\frac{B\sub{min}}{B}\right)(p_{\parallel,0}^2+p_{\perp,0}^2). \nonumber
\end{align}



\subsubsection*{Radial transport terms}
The radial transport lacks a first-principle description, and we assume that it can be described by an advection term $\b{A}_r$ and a diffusion term $\mathsf{D}_r$ of the form
\begin{align}
\b{A}_r &= A(r) \frac{\partial \b{x}}{\partial r} , \nonumber \\
\mathsf{D}_r &= D(r)\frac{\partial \b{x}}{\partial r}\frac{\partial \b{x}}{\partial r}.
\end{align}
The non-vanishing contravariant components are then simply
\begin{align}
A_r^r &= A(r), \nonumber \\
D_r^{rr} &= D(r).
\end{align}

\subsubsection*{Synchrotron loss term}
The synchrotron losses can be described by an advection term $\b{A}_s$ of the form
\begin{align}
\b{A}_s &= -\frac{1}{\gamma \tau_s}\left(p_\perp\frac{\partial \b{p}}{\partial p_\perp} + p\frac{p_\perp^2}{m_e^2c^2}\frac{\partial \b{p}}{\partial p}\right), \nonumber \\
&= -\frac{1}{\gamma \tau_s}\left( p_\perp \hat{p}_\perp + p\frac{p_\perp^2}{m_e^2c^2}\hat{p}\right), \nonumber \\
&= -\frac{1}{\gamma\tau_s}\left[ \frac{p_\perp^2}{m_e^2c^2}p_\parallel \hat{p}_\parallel + \left(1+\frac{p_\perp^2}{m_e^2c^2}\right)p_\perp\hat{p}_\perp\right] \nonumber \\
\frac{1}{\tau_s} &= \frac{e^4 B^2}{6\pi\varepsilon_0 m_e^3c^3}.
\end{align}

The components are as follows:
\begin{align}
A_s^p &=    -\frac{1 }{p \gamma\tau_s} \left(  p_\perp^2  + p_\perp^2 \frac{p^2}{m_e^2c^2} \right), \nonumber \\
&= -\frac{ \gamma p_\perp^2}{p \tau_s}\nonumber \\
&=- B\lambda \frac{p \gamma}{\tau_s}\nonumber \\
&= -\frac{\lambda p \gamma}{B^2\tau_s}B^3, \nonumber \\
A_s^\lambda &= -\frac{p_\perp}{\gamma\tau_s} \nabla_\b{p}\lambda\cdot\hat{p}_\perp\nonumber \\
&= \frac{2\xi p\sqrt{1-\xi^2}}{\gamma\tau_s B} \nabla_\b{p} \xi \cdot \hat{p}_\perp \nonumber \\
&= -\frac{2\lambda}{\gamma\tau_s}\xi^2 \nonumber \\
&= -\frac{2\lambda}{\gamma B^2\tau_s} B^2(1-\lambda B). 
\end{align}

\paragraph{$p$, $\xi_0$:}
\begin{align}
A_s^p%&=- B\lambda \frac{p^2}{\tau_s}\nonumber \\
&= -\frac{\lambda p\gamma}{B\sub{min}^2\tau_{s,\text{min}}}B^3, \nonumber \\
&= -\frac{p\gamma}{\tau_{s,\text{min}}} (1-\xi_0^2)\frac{B^3}{B\sub{min}^3}, \nonumber \\
A_s^{\xi_0} &= -\frac{1}{2}\frac{B\sub{min}}{\xi_0}A_s^{\lambda} \nonumber \\
&= \frac{1-\xi_0^2}{\gamma\tau_{s,\text{min}}} \frac{B^2}{B\sub{min}^2}\frac{\xi^2}{\xi_0} \nonumber \\
&= \frac{1}{\gamma B^2 \tau_s} \frac{1-\xi_0^2}{\xi_0} \left(1-\frac{B}{B\sub{min}}(1-\xi_0^2)\right) B^2\nonumber \\
&= \frac{ \xi_0(1-\xi_0^2)}{\gamma \tau_{s,\text{min}}} \frac{B^3}{B\sub{min}^3} -  \frac{1}{\gamma \tau_{s,\text{min}}}\frac{1-\xi_0^2}{\xi_0}\frac{B^2}{B\sub{min}^2}\left(\frac{B}{B\sub{min}}-1\right) 
%p\frac{(1-\xi_0^2)}{\xi_0}\xi^2 \nonumber \\
%&= p\frac{1-\xi_0^2}{\xi_0} \left(1-\frac{B}{B\sub{min}} + \frac{B}{B\sub{min}}\xi_0^2 \right)   \nonumber \\
%&=\frac{B}{B\sub{min}} p\xi_0(1-\xi_0^2)  - p\frac{1-\xi_0^2}{\xi_0}\left(\frac{B}{B\sub{min}}-1\right).
\end{align}

\paragraph{$p_{\parallel,0}$, $p_{\perp,0}$:}
\begin{align}
A_s^{p_{\parallel,0}} &= - \frac{p_\perp^2}{\gamma\tau_sp_{\parallel,0}} \left[\frac{p_\parallel^2}{m_e^2c^2}+\left(1+\frac{p_\perp^2}{m_e^2c^2}\right)\left(1-\frac{B\sub{min}}{B}\right)\right] \nonumber \\
&= -\frac{p_{\perp,0}^2}{\gamma B^2\tau_s p_{\parallel,0}}B^2\left(\frac{B}{B\sub{min}}-1 + \frac{B}{B\sub{min}}\frac{p_{\parallel,0}^2}{m_e^2 c^2}\right) \nonumber\\
%&= -\frac{p_{\perp,0}^2}{\gamma B^2\tau_s p_{\parallel,0}}B^2 \left[\frac{B}{B\sub{min}} \frac{p_\parallel^2}{m_e^2c^2}+\left(1+\frac{B}{B\sub{min}}\frac{p_{\perp,0}^2}{m_e^2c^2}\right)\left(\frac{B}{B\sub{min}} -1\right)\right] \nonumber \\
%&= -\frac{p_{\perp,0}^2}{\gamma B^2\tau_s p_{\parallel,0}}B^2 \left[\frac{B}{B\sub{min}} \left(\frac{p_{\parallel,0}^2}{m_e^2c^2} - \left(\frac{B}{B\sub{min}}-1\right)\frac{p_{\perp,0}^2}{m_e^2c^2}\right)+\left(1+\frac{B}{B\sub{min}}\frac{p_{\perp,0}^2}{m_e^2c^2}\right)\left(\frac{B}{B\sub{min}} -1\right)\right] \nonumber \\
A_s^{p_{\perp,0}} &= -\sqrt{\frac{B\sub{min}}{B}}\frac{p_\perp}{\gamma\tau_s}\left(1+\frac{p_\perp^2}{m_e^2c^2}\right) \nonumber \\
&=  -\frac{p_{\perp,0}}{\gamma B^2\tau_s}B^2\left(1+\frac{B}{B\sub{min}}\frac{p_{\perp,0}^2}{m_e^2c^2}\right) 
\end{align}



\subsubsection*{Energy diffusion term}
From the next appendix, the energy diffusion term can be represented by an operator
\begin{align}
\mathsf{D} &= m_e  T\sub{cold} \gamma \nu_s \hat{p}\hat{p}, \nonumber \\
&= (m_e c)^2 \nu_\parallel \hat{p}\hat{p} , \\
\nu_\parallel &= \frac{T}{m_e c^2}\gamma \nu_s .
\end{align}

The components are:

\paragraph{$p$, $\xi_0$:}
\begin{align}
D^{pp} = (m_e c)^2 \nu_\parallel .
\end{align}

\paragraph{$p_{\parallel,0}$, $p_{\perp,0}$:}
We can first give
\begin{align}
\nabla p_{\parallel,0} \cdot \hat{p}\hat{p} &= \xi_0 \hat{p} , \nonumber \\
\nabla p_{\perp,0} \cdot \hat{p}\hat{p} &= \sqrt{1-\xi_0^2}\hat{p},
\end{align}
which gives us
\begin{align}
D^{p_{\parallel,0},p_{\parallel,0}} &= (m_e c)^2 \nu_\parallel \xi_0^2, \nonumber \\
D^{p_{\parallel,0},p_{\perp,0}} = D^{p_{\perp,0},p_{\parallel,0}} &= (m_e c)^2 \nu_\parallel \xi_0\sqrt{1-\xi_0^2}, \nonumber \\
D^{p_{\perp,0},p_{\perp,0}} &= (m_e c)^2 \nu_\parallel(1- \xi_0^2), \nonumber \\
\end{align}

\subsection{Collision frequencies $\nu_s$ and $\nu_\parallel$ when resolving thermal bulk}
A new way of matching the collision frequencies to the thermal region while retaining a Maxwellian distribution as exact solution to the collision operator, is to use the following representation of the momentum part of the collision operator (which holds exactly for the ideal relativistic test-particle Fokker-Planck operator)

\begin{align}
C_p = \frac{1}{p^2}\frac{\partial}{\partial p} \left[ p^4 \nu\sub{coll}(p) f\sub{Me}\frac{\partial}{\partial p}\left(\frac{f}{f\sub{Me}}\right)\right],
\end{align}
where we could for example take a relativistic Maxwellian $f\sub{Me}\propto e^{-\gamma  m_e c^2/T}$ which satisfies 
\begin{align}
\frac{1}{f\sub{Me}}\frac{\partial f\sub{Me}}{\partial p} = -\frac{p}{\gamma}\frac{c}{T} = -\frac{2 p/p\sub{Te}^2}{\gamma}, 
\end{align}
in which case we find
\begin{align}
C_p = \frac{1}{p^2}\frac{\partial}{\partial p} \left[ \frac{p^5}{\gamma}\nu\sub{coll} \frac{c}{T} f  + p^4 \nu\sub{coll} \frac{\partial f}{\partial p}\right].
\end{align}
If we wish to write this on the standard form
\begin{align}
C_p &=\frac{1}{p^2}\frac{\partial}{\partial p} \left[ p^3\nu_s f + \frac{\nu_\parallel}{2}p^4\frac{\partial f}{\partial p}\right],
\end{align}
we can identify
\begin{align}
\nu_s &= \frac{p^2}{\gamma} \frac{c}{T}\nu\sub{coll}, \nonumber \\
\nu_\parallel &= 2\nu\sub{coll} \nonumber \\
&= \frac{2T}{c}\frac{\gamma}{p^2}\nu_s,
\end{align}
where we can for example choose $\nu\sub{coll}$ as Linnea did, by making it match the Bethe stopping power for high energies and matching to the well-known completely screened result for thermal energies.




\section{Non-linear self-collision operator}
If the cooling is sufficiently slow, self-collisions will Maxwellianize the hot population and prevent a hot tail from forming (which does not really occur unless the self-collisions are accounted for). In order to describe this, we need to derive the self-collision operator for the isotropic hot electron poplation (since we order the self-collisions as $\delta^2$, i.e. being assumed to be equally important as collisions with the background). 

\subsection{Non-relativistic case}
Rosenbluth-MacDonald-Judd gives the non-relativistic self-collision operator for isotropic potentials $g$ and $h$ as
\begin{align}
C_{ee,nl} =  -\frac{1}{p^2}\frac{\partial}{\partial p}\left(p^2\frac{\partial h}{\partial p}f+\frac{\partial g}{\partial p}f\right) + \frac{1}{p^2}\frac{\partial^2}{\partial p^2} \left(p^2\frac{1}{2}\frac{\partial^2 g}{\partial p^2} f\right) +\frac{1}{2p^3}\frac{\partial g}{\partial p}\frac{\partial}{\partial \xi}\left[(1-\xi^2)\frac{\partial f}{\partial \xi}\right],
%- \frac{1}{p^2}\frac{\partial}{\partial p}\left( \right),
\end{align}
which we can write as
\begin{align}
C_{ee,nl} &= \frac{1}{p^2}\frac{\partial}{\partial p}\left[p^2\left( p\nu_s^{nl} f+ \frac{p^2\nu_\parallel^{nl}}{2}\frac{\partial f}{\partial p}\right)\right] + \frac{\nu_D^{nl}}{2}\frac{\partial}{\partial \xi}\left[(1-\xi^2)\frac{\partial f}{\partial\xi}\right], \\
\Gamma_e^{-1}p\nu_s^{nl} &= -\frac{\partial h}{\partial p} - \frac{1}{p^2}\frac{\partial g}{\partial p} +\frac{1}{p^2}\frac{\partial}{\partial p}\left(\frac{p^2}{2}\frac{\partial^2 g}{\partial p^2}\right) \nonumber \\
\Gamma_e^{-1}\frac{p^2}{2}\nu_\parallel^{nl} &= \frac{1}{2}\frac{\partial^2 g}{\partial p^2} \nonumber\\
\Gamma_e^{-1}\nu_D^{nl} &= \frac{1}{p^3}\frac{\partial g}{\partial p}  \nonumber, \\
\Gamma_e &= \frac{e^4 \ln\Lambda}{4\pi\varepsilon_0^2 m_e^2}, \nonumber
\end{align}
and where the potentials satisfy
\begin{align}
\nabla^2 h &= -8\pi f \nonumber \\
\nabla^2 g &= h,
\end{align}
from which we find
\begin{align}
\frac{1}{p^2}\frac{\partial}{\partial p}p^2\frac{\partial h}{\partial p} &= -8\pi f, \nonumber \\
\frac{1}{p^2}\frac{\partial}{\partial p}p^2\frac{\partial g}{\partial p} &= h , \nonumber \\
g'' &= h-\frac{2}{p}g'.
\end{align}
With the last relation, we can simplify
\begin{align}
p\nu_s^{nl} &=  -\frac{1}{2}\frac{\partial h}{\partial p}. %\frac{2}{p^2}g' - \frac{2}{p}h - \frac{3}{2}h' .%- \frac{1}{p^2}g' + \frac{h'}{2} + \frac{g'}{p^2}%\frac{1}{p^2}\frac{\partial}{\partial p}\left(\frac{p^2}{2}(h-2g'/p)\right)
\end{align}
The potentials $h$ and $g$ are in turn given by
\begin{align}
h &= 8\pi \left( \frac{1}{p}\int_0^p \rd p' \, p'^2 f(p') + \int_p^\infty \rd p' \, p'f(p')\right), \nonumber \\
\frac{\partial h}{\partial p} &= -\frac{8\pi}{p^2}\int_0^p \rd p' \, p'^2 f(p'), \nonumber \\
g &= \frac{4\pi}{3}\left( \frac{1}{p}\int_0^ p \rd p' \,(3p^2+p'^2)p'^2f(p')  + \int_p^\infty \rd p' \,(p^2+3p'^2)p'f(p')\right), \nonumber \\
\frac{\partial g}{\partial p} &= \frac{4\pi}{3}\left[ \int_0^p \rd p'\,p'^2\left(3-\frac{p'^2}{p^2}\right)f(p') + 2p\int_p^\infty \rd p' \, p' f(p')\right] .
\end{align}

We can therefore give the final expressions
\begin{align}
-A^p &= p\nu_s = -\frac{1}{2}h' \nonumber \\
&= \Gamma_e 4\pi \int_0^p \rd p'\,\frac{p'^2}{p^2}f(p'), \nonumber \\
D^{pp} &= \frac{p^2\nu_\parallel}{2} = \frac{1}{2} g'' = \frac{1}{2}h - \frac{1}{p}g' \nonumber \\
&= \Gamma_e \frac{4\pi}{3}\left( \int_0^p \rd p' \, \frac{p'^4}{p^3} f(p') + \int_p^\infty \rd p' \,p' f(p')\right), \nonumber \\
\nu_D &= \frac{1}{p^3}g' \nonumber \\
&= \Gamma_e  \frac{4\pi}{3}\frac{1}{p}\left[ \int_0^p \rd p'\,\frac{p'^2}{p^2}\left(3-\frac{p'^2}{p^2}\right)f(p') +\int_p^\infty \rd p' \, \frac{2p'}{p} f(p')\right] .
\end{align}

\subsection{Trapezoidal quadrature}
Consider an integral of the form 
\begin{align*}
\int_0^p  F(p') \, \rd p', 
\end{align*}
where $F$ is known on the distribution grid $p'$ and $p$ is evaluated on the flux grid. These satisfy $p_i < p'_i < p_{i+1}$. In that case, we can write the integral evaluated at $p=p_i$ as
\begin{align}
\int_0^p F(p') \, \rd p' = \int_0^{p_0'} F(p') \,\rd p' + \int_{p_0'}^{p_{i-1}'} F(p') \,\rd p' + \int_{p_{i-1}'}^{p_i} F(p') \,\rd p'.
\end{align}
For the first term, we can use the physically-motivated boundary condition that $f'(0) = 0$ to taylor expand 
\begin{align}
\int_0^{p_0'} p'^n f(p') \, \rd p' = f(p_0') \frac{p_0'^{n+1}}{n+1}.
\end{align}
For the middle term, we use a regular trapezoidal rule
\begin{align}
\int_{p_0'}^{p_{i-1}'} F(p')\,\rd p' &= \sum_{k=1}^{i-1} \frac{F(p_{k-1}')+F(p_k')}{2}(p_k'-p_{k-1}') \nonumber \\
&= \frac{p_1'-p_0'}{2}F(p'_0) + \frac{p_{i-1}'-p_{i-2}'}{2}F(p_{i-1}') + \sum_{k=1}^{i-2}\frac{p_{k+1}-p_{k-1}}{2}F(p_k').
\end{align}
For the final bit, we note that between $p_{i-1}'$ and $p_i'$ we can interpolate linearly
\begin{align}
F(p') = F(p_{i-1}') + \frac{p' - p_{i-1}'}{p_i' - p_{i-1}'}[F(p_i') - F(p_{i-1}')],
\end{align}
which integrates to
\begin{align}
\int_{p_{i-1}'}^{p_i} F(p') \,\rd p' &= (p_i-p_{i-1}')F(p'_{i-1}) + \frac{1}{p_i'-p_{i-1}'}\left[\frac{p_i^2-p_{i-1}'^2}{2} - p_{i-1}'(p_i - p_{i-1})\right][F(p_i')-F(p_{i-1}')] \nonumber \\
&=\frac{p_i-p_{i-1}'}{p_i'-p_{i-1}'}\left[ F(p_{i-1}')\left(p_i' - p_{i-1}' - \frac{p_i+p_{i-1}'}{2}+p_{i-1}' \right) +F(p_i')  \left(\frac{p_i+p_{i-1}'}{2}-p_{i-1}'\right)\right] \nonumber \\
&=\frac{p_i-p_{i-1}'}{p_i'-p_{i-1}'}\left[ \frac{2p_i'-p_i-p_{i-1}'}{2} F(p_{i-1}')+ \frac{p_i-p_{i-1}'}{2}F(p_i')\right].
\end{align}
At $p\sub{max}$ on the flux grid, i.e. when $i=N+1$, we can assume a boundary condition $F(p_{N+1}') = F(p_N')$.

In a similar way, we write
\begin{align}
\int_p^\infty F(p') \, \rd p' = \int_{p_i}^{p_i'} F(p') \, \rd p' + \int_{p_i'}^{p_N'} F(p')\,\rd p' + \int_{p_N'}^{p_{N+1}}F(p') \,\rd p'.
\end{align}
For the first bit, we use the same linear interpolation as above to obtain
\begin{align}
 \int_{p_i}^{p_i'} F(p') \, \rd p' = \frac{p_i'-p_i}{p_i'-p_{i-1}'}\left[\frac{p_i'-p_i}{2}F(p_{i-1}')+\frac{p_i+p_i'-2p_{i-1}'}{2}F(p_i') \right].
\end{align}
The second piece is again given by the trapezoidal rule
\begin{align}
\int_{p_i'}^{p_N'} F(p')\,\rd p' =  \frac{p_{i+1}'-p_i'}{2}F(p'_i) + \frac{p_N'-p_{N-1}'}{2}F(p_N') + \sum_{k=i+1}^{N-1}\frac{p_{k+1}-p_{k-1}}{2}F(p_k'),
\end{align}
and we neglect the last piece since the grid is presumably chosen large enough that the dynamics are not sensitive to the boundary.


%With the last relation, we can significantly simplify.
%\begin{align}
%p\nu_s^{nl} &= -\frac{1}{2}\frac{\partial h}{\partial p}.%- \frac{1}{p^2}g' + \frac{h'}{2} + \frac{g'}{p^2}%\frac{1}{p^2}\frac{\partial}{\partial p}\left(\frac{p^2}{2}(h-2g'/p)\right)
%\end{align}
%We can integrate the coupled equations defining $h$ and $g$ to obtain
%\begin{align}
%h' &= -8\pi \frac{1}{p^2}\int_0^p \rd p' \, p'^2 f(p'), \nonumber \\
%h &= 8\pi \left( \frac{1}{p}\int_0^p \rd p' p'^2 f(p') + \int_p^\infty \rd p' \, p'f(p')\right) \nonumber \\
%g' &= \frac{1}{p^2} \int_0^p \rd p'\,p'^2 h(p') \nonumber \\
%&= \frac{1}{3p^2}\int_0^p \rd p' \, p'^4 f(p') + \frac{p}{3}\int_p^\infty \rd p'\,p'f(p'),
%\end{align}
%which fully describes the non-linear self-collision operator for an isotropic field particle distribution.



\end{document}
