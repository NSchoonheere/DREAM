\documentclass[11pt,a4paper]{article}
\usepackage{diagbox}
\usepackage{wrapfig}
\usepackage[utf8]{inputenc}
%\usepackage[swedish]{babel}
\usepackage{graphicx}
\usepackage{amsmath}
\usepackage{amssymb}
\usepackage{units}
\usepackage{ae}
\usepackage{icomma}
\usepackage{color}
\usepackage{graphics} 
\usepackage{bbm}
\usepackage{float}

\usepackage{caption}
\usepackage{subcaption}

\usepackage{hyperref}
\usepackage{epstopdf}
\usepackage{epsfig}
\usepackage{braket}
\usepackage{pdfpages}

\usepackage{tcolorbox}

\usepackage{MnSymbol}

\newcommand{\N}{\ensuremath{\mathbbm{N}}}
\newcommand{\Z}{\ensuremath{\mathbbm{Z}}}
\newcommand{\Q}{\ensuremath{\mathbbm{Q}}}
\newcommand{\R}{\ensuremath{\mathbbm{R}}}
\newcommand{\C}{\ensuremath{\mathbbm{C}}}
\newcommand{\id}{\ensuremath{\,\mathrm{d}}}
\newcommand{\rd}{\ensuremath{\mathrm{d}}}
\newcommand{\Ordo}{\ensuremath{\mathcal{O}}}% Stora Ordo
\renewcommand{\L}{\ensuremath{\mathcal{L}}}% Stora Ordo
\newcommand{\sub}[1]{\ensuremath{_{\text{#1}}}}
\newcommand{\ddx}[1]{\ensuremath{ \frac{\partial}{\partial #1} }}
\newcommand{\ddxx}[2]{\ensuremath{ \frac{\partial^2}{\partial #1 \partial #2} }}
%\newcommand{\sup}[1]{\ensuremath{^{\text{#1}}}}
\renewcommand{\b}[1]{\ensuremath{ {\bf #1 } }}
\renewcommand{\arraystretch}{1.5}

\begin{document}

\begin{center}
\Large \bf Discretization of a 3D delta source in DREAM
\end{center}

A pellet may be treated as a point source which moves through three-dimensional space. In this note, we outline the numerical discretization of such a source within the one-dimensional (in radius) DREAM formulation.

The pellet can be viewed as a source term for an unknown plasma parameter $x$, depending locally on all unknowns $y$ in the system. The pellet is located in the position $\b{x}_s(t)$, and is assumed to deposit material in a location $\b{x}_d = \b{x}_d(\b{x}_s(t),\,y(\b{x}_d))$\footnote{We assume that the spreading is a pure translation which depends only on the plasma at the target location in order to make the source local - in reality there would likely be spreading which would need to be modelled more accurately}, which may be separate from the pellet location itself as the ablated material can drift before spreading out over the flux surface and becoming confined. 

Then, the SPI contribution to the evolution of quantity $x$ is
\begin{align}
\left(\frac{\partial x}{\partial t}\right)_\mathrm{SPI} &= F(y) \delta^3( \b{x} - \b{x}_d(t,\,y)),
\end{align}
where $y$ denotes local plasma parameters at the location $\b{x}_d$ where the material is deposited and $F$ is the source amplitude.  The numerical challenge lies in the discretization of the delta function, to which we now turn our attention.

\section*{Flux surface average of delta function}
In DREAM we utilise the toroidal coordinates $(r,\,\varphi,\,\theta)$ where $r$ is a (minor-)radial coordinate labelling the flux surface, $\varphi$ the toroidal angle and $\theta$ a poloidal angle. Refer to \texttt{DREAM/doc/notes/theory.tex} for a more detailed description of the geometry.  The delta function expressed in toroidal coordinates becomes
\begin{align}
\delta^3( \b{x} - \b{x}_d(t,\,y)) = \frac{\delta(r-r_d)\delta(\theta-\theta_d)\delta(\varphi - \varphi_d)}{\mathcal{J}},
\end{align}
where $\mathcal{J}$ is the jacobian of the coordinate system. The flux surface average of this becomes
\begin{align}
\langle\delta^3( \b{x} - \b{x}_d(t,\,y))\rangle = \frac{\delta(r-r_d)}{V'},
\end{align}
where $V(r)$ denotes the total volume enclosed by the flux surface labelled $r$. In DREAM, $V'(r)$ is denoted \texttt{VpVol} and is stored in the \texttt{RadialGrid} object. Here $r_d(\b{x}_s(t), y(r_d))$ denotes the radius (i.e.~flux surface) on which the material is deposited, and $\b{x}_s(t)$ still denotes the pellet location.

\section*{Time averaged delta source}
Since a pellet may pass several radial grid points in one time step, a purely explicit or implicit time step would cause all ablated material to only be deposited in the location of the pellet at the beginning or end of the time step, respectively. In order to spread the material across all radial locations crossed by the pellet during a time step, we may perform a time average where we evaluate the unknown quantities $y_0 = y(t_0)$ at a given time (previous time step for explicit stepping, next time step for implicit), but carry out a time average over the full pellet motion
\begin{align}
\llangle\delta^3( \b{x} - \b{x}_d(t,\,y)) \rrangle &= \frac{1}{\Delta t} \int_t^{t+\Delta t}  \langle \delta^3( \b{x} - \b{x}_d(t,\,y))\rangle \, \rd t \nonumber \\
&= \frac{1}{V'\Delta t} \int_t^{t+\Delta t} \delta(r-r_d(t,y_0)) \, \rd t  \nonumber\\
&= \frac{1}{V'\Delta t} \int_{r_d(t)}^{r_d(t+\Delta t)}\frac{\delta(r-r_d)}{\dot{r}_d} \, \rd r_d \nonumber \\
&= \frac{1}{V'|\dot{r}| \Delta t } \times \begin{cases}
1, & r_{dl} < r < r_{du}  \\
0, & \text{otherwise}
\end{cases} \nonumber \\
r_{dl} &= \text{min}[r_d(t),r_d(t+\Delta t)], \nonumber \\
r_{du} &=  \text{max}[r_d(t),r_d(t+\Delta t)],
\end{align}
where $\dot{r}$ denotes the radial velocity of the pellet deposition location at radial location $r$. This way, we have transformed the 3D delta source to a box shaped source in radius which will behave well even with longer time steps, particularly when combined with implicit time stepping where $t_0 = t_{n+1}$, and $\partial x/\partial t = (x_{n+1}-x_n)/\Delta t$.

An alternative, at the same order of accuracy in $\Delta t$, would be to replace
\begin{align}
|\dot{r}|\Delta t \mapsto r_{du}-r_{dl}
\end{align}
which may be a more attractive option since $r_{dl}$ and $r_{du}$ need to be evaluated regardless.

\section*{Evaluation of radius $r$}
Given the deposition location $\b{x}_d(\b{x}_s,\,y)$, we need to evaluate the radius $r_d(\b{x}_d)$. In DREAM, it would be appropriate to let the \texttt{RadialGridGenerator} class perform this calculation, since this is the class where the magnetic geometry of the system is specified.
In particular, the radial grid generators are implemented in the two classes \texttt{CylindricalRadialGridGenerator} (describing a cylindrical geometry such as used in GO) or \texttt{AnalyticBRadialGridGenerator} which describes a toroidal geometry with shaped flux surfaces parametrised by
\begin{align}
\b{x} &= R\hat{R} + z\hat{z}, \nonumber \\
R &= R_m + \Delta(r) + r\cos[\theta +\delta(r) \sin\theta], \nonumber \\
z &= r \kappa(r) \sin\theta, \nonumber \\
\hat{R} &= \cos\varphi \hat{x} + \sin\varphi \hat{y},
\label{eq:analyticB system}
\end{align}
where $\kappa$ is the elongation, $\Delta$ the Shafranov shift, $\delta$ a triangularity parameter, $R=\sqrt{x^2+y^2}$ the major radius and $R_m$ the radius of the magnetic axis. The \texttt{CylindricalRadialGridGenerator} effectively implements the special case $R_m=\infty$, $\delta=0$, $\kappa=1$, representing an infinite straight cylinder.

A way to accomplish this could be to equip \texttt{RadialGridGenerator} with a function
\begin{verbatim}
real_t RadialGridGenerator::evaluateRadialPosition(real_t x, real_t y, real_t z)
\end{verbatim}
which would be implemented separately in the derived classes (\texttt{Cylindrical}, \texttt{AnalyticB} and other future generators).

In the cylindrical radial grid generator, a suitable definition could be to return $r = \sqrt{x^2+y^2}$ (if we were to assume that the cylinder has $\hat{z}$ as its symmetry axis), whereas in the \texttt{AnalyticB} generator, the equation system (\ref{eq:analyticB system}) above needs to be solved numerically. 

In the future we will likely also support a \texttt{NumericalRadialGridGenerator} that loads numerical magnetic field data and interpolates to determine $r$ numerically.

\subsection*{(optional) Evaluation of radial velocity $\dot{r}$}
The radial velocity of the pellet deposition location is given by
\begin{align}
\dot{r} &= \nabla r \cdot \frac{\rd \b{x}_d(\b{x}_s(t), y_0)}{\rd t} \nonumber \\
&= \nabla r \cdot \left( \frac{\partial \b{x}_d}{\partial \b{x}_s} \cdot \dot{\b{x}}_s\right) 
%&= \frac{\partial r}{\partial x}\frac{\rd x}{\rd t} + \frac{\partial r}{\partial y}\frac{\rd y}{\rd t} + \frac{\partial r}{\partial z}\frac{\rd z}{\rd t}.
\end{align}
The cartesian components of the pellet velocity $\dot{\b{x}}_s = (\dot{x}_s,\,\dot{y}_s,\,\dot{z}_s)$ will likely be held constant as the pellet is assumed to move along a line at constant speed, although this could be generalised if we wish $(x,y,z)$ to satisfy some equations of motion (say, by adding drag force which depends on plasma parameters or jet propulsion due to anisotropic ablation). The Jacobian matrix $\partial \b{x}_d/\partial \b{x}_s$ connecting deposition location to pellet location will depend on the model imposed for the drift of the ablated material -- in the simplest model, we can assume the material to be deposited locally so that this becomes the identity matrix; otherwise it will probably depend on the plasma parameters $y$.

For the \texttt{AnalyticB} generator, according to \texttt{doc/notes/theory}, we can evaluate $\nabla r$ using
\begin{align}
\nabla r &= \frac{\frac{\partial z}{\partial \theta} \hat{R} - \frac{\partial R}{\partial \theta} \hat{z}}{\frac{\partial R}{\partial r}\frac{\partial z}{\partial \theta} - \frac{\partial R}{\partial \theta}\frac{\partial z}{\partial r}},
\end{align}
where
\begin{align}
\frac{\partial R}{\partial r} &= \Delta' + \cos[\theta+\delta\sin\theta]-r\delta'\sin\theta\sin[\theta+\delta\sin\theta], \nonumber \\
\frac{\partial R}{\partial \theta} &= -r(1+\delta\cos\theta)\sin[\theta+\delta\sin\theta],  \nonumber \\
\frac{\partial z}{\partial r} &=  \kappa\left(1+ \frac{r\kappa'}{\kappa}\right)\sin\theta, \nonumber \\
\frac{\partial z}{\partial \theta} &= r\kappa\cos\theta.
\end{align}
This provides a full description of the geometry of pellet trajectories in shaped magnetic fields in DREAM. 

This could be explicitly implemented by, for example, equipping RadialGridGenerator with functions
\begin{verbatim}
real_t RadialGridGenerator::evaluateDrDx(real_t r, real_t theta);
real_t RadialGridGenerator::evaluateDrDy(real_t r, real_t theta);
real_t RadialGridGenerator::evaluateDrDz(real_t r, real_t theta);
\end{verbatim}
which would be implemented in the various derived classes. 

\section*{TODO: Finite volume discretization of time-flux-averaged source}
The final step of the discretization is to carry out the finite volume discretization of the source function in radius. This is relatively straight forward, as we assume that the plasma parameters $y$ are evaluated at the cell center and it remains only to carefully average the box function $\llangle\delta^3( \b{x} - \b{x}_d(t,\,y)) \rrangle $ over the cell (a simple average over the radial interval $r_{i-1/2} < r < r_{i+1/2}$ where everything is held fixed except for the ``case'' block of Eq (4)).





\section*{NGS pellet model}
In the NGS ablation model, we solve the system of equations
\begin{align}
\frac{\partial N}{\partial t}(t) &= - K(y(t,\b{x})) N(t)^{a}, \\
\frac{\partial n_i}{\partial t}(t,\,\b{x}) &= - \frac{\partial N}{\partial t}(t)\delta^3(\b{x} - \b{x}_s(t)) + ...\\
\frac{\partial W}{\partial t}(t,\,\b{x}) &= I_d \frac{\partial N}{\partial t}(t) \delta^3(\b{x}-\b{x}_s(t)) + ...,
\end{align}
where $N =(4/3)\pi r_s^3 \rho$ is the total number of particles in the pellet (/shard), $n_i$ the number density of the ion density that it is composed of (say, neutral hydrogen), $W$ the local energy density of the plasma, $K>0$ and $a=4/9$ parameters of the NGS model which set the ablation rate, $I_d$ the (positive) disocciation energy per particle in the pellet (possibly adding the ionization energy if we take $n_i$ as an ionized species), $\b{x}_s(t)$ the pellet location and $y$ the set of plasma parameters which influence the ablation rate and `...' denote equation terms that are independent of the pellets.

An option to model the motion of the ablated cloud before it becomes confined by the plasma is to add another fluid field $n_\text{abl}(t,\,\b{x})$ of newly ablated material, such that
\begin{align}
\frac{\partial n\sub{abl}}{\partial t}(t,\,\b{x}) &= - \frac{\partial N}{\partial t}(t) \delta^3(\b{x}-\b{x}_s(t)) - \frac{n\sub{abl}}{\tau} + \frac{\partial}{\partial \b{x}}\cdot \Phi, \nonumber \\
\frac{\partial n_i}{\partial t}(t,\,\b{x}) &=  \frac{n\sub{abl}}{\tau} + ... ,
\end{align}
where $\tau$ is the time-scale for the ablated material to become confined (probably characterising some parallel-flow rate?) providing a delay before the ablated material contributes to cooling the plasma and raising collisionality, and $\Phi$ a flux of ablated material, for example of the form of an advection-diffusion term 
\begin{align}
\Phi &= \b{A}\sub{abl} n\sub{abl} + \mathsf{D}\sub{abl}\cdot \frac{\partial n\sub{abl}}{\partial \b{x}}
\end{align}
with \b{A} an advective flow velocity and $\mathsf{D}$ a diffusion tensor, which can be used to model the flow of ablated material from the point of ablation to the point where it becomes confined to the flux surfaces. 

\paragraph{Solution strategy:} For particle and energy conservation, we wish to discretize the second and third equation as 
\begin{align}
\frac{n_i(t_{n+1}) - n_i(t_n)}{\Delta t} = -\frac{N(t_{n+1}) - N(t_n)}{\Delta t}\delta^3(\b{x}-\b{x}_s) + ...,
\end{align}
so that the change in particle number in the pellet directly contributes to the ion density in a number-conserving manner which is independent of resolution -- then, once the pellet has ablated, the correct number of particles will have been added to the plasma. In the first equation, which sets the ablation rate, we are free to discretize any way we see fit. A convenient choice is the implicit scheme
\begin{align}
\frac{N(t_{n+1})^{1-a} - N(t_n)^{1-a}}{\Delta t} + (1-a)K(y(t_{n+1},\,\b{x}) = 0
\label{eq:Ntnp1}
\end{align}
where we integrated the equation exactly under the assumption that $y$ is constant at its value in the next time step, $t_{n+1}$. For extensions of this idealized model, where we may want to add additional non-linear terms $H(y,\,N)$ to the right-hand side of (\ref{eq:Ntnp1}), this form of the equation solved with a Newton method (according to the DREAM recipe) will be suitable, however for this simple case the solution can in principle be explicitly given 
\begin{align}
N(t_{n+1}) = \left[ N(t_n)^{1-a} - \Delta t (1-a)K(y(t_{n+1},\,\b{x})  \right]^{1/(1-a)},
\end{align}
which provides the solution without the need of any additional iterations in a Newton solver.


A numerical challenge is posed by the fact that (\ref{eq:Ntnp1}) will yield undefined $N(t_{n+1})$ in the final time step as the pellet is completely ablated (as $N^{1-a}$ goes negative, and the jacobian diverges). A possible resolution is as follows: in each iteration $k$ of the Newton solver, if
\begin{align}
N(t_n)^{1-a} - (1-a)\Delta t K(y(t_{n+1}^{(k)},\b{x}) < 0,
\end{align}
\textcolor{red}{[is this a sufficient constraint to guarantee that the solver will converge?]} replace equation (\ref{eq:Ntnp1}) by $N(t_{n+1}) = 0$ , otherwise use the full expression.

\subsection*{More careful time average of the equation system}
\textcolor{red}{[this section is a bit of a mess, I just jotted it down quickly to have the main pieces in place, but I don't think I really believe in it because of its complexity]}

In case we wish to take such long time steps that multiple radial cells are traversed by a pellet in a single time step, we may wish to perform a more careful time average of the equations that preserves conservation of particle number.

When flux-surface averaging, and carrying out the FVM cell average, the delta function is replaced by
\begin{align}
\delta^3(\b{x}-\b{x}_s) \mapsto \begin{cases}
\frac{1}{V' \Delta r_i}, & r_{i-1/2} < r(\b{x}_s(t)) \leq r_{i+1/2}, \\
0 & \text{otherwise}.
\end{cases}
\end{align}
Then, we can again solve exactly for the ablation of the pellet under the assumption that all plasma quantities are evaluated at their value in the next time step, and also that they are piecewise constant, taking the constant value $K_i = K(y(t_{n+1}, \,r_i)$ within a given cell with radial index $i$:
\begin{align}
\frac{N(t_{k+1})^{1-a} - N(t_k)^{1-a}}{\Delta t} + (1-a) K_{i_k} = 0, \quad k = 0,~1,~...,~k\sub{max}
\end{align}
where $\Delta t_k = t_{k+1} - t_k$ is the time during the time step that the pellet spent in the cell with index $i_k$, where $i_0$ is the cell which contained the pellet $\b{x}_s(t_0 = t_n)$ at the beginning of the time step, $i_{k\sub{max}}$ contains $\b{x}_s(t_{k\sub{max}} = t_{n+1})$, and we solve for all grid cells $k$ that the pellet moves across. Summing over all $k$ up to some particular value, the equation takes the form
\begin{align}
N(t_{k+1}) = \left[ N(t_n)^{1-a} - (1-a)\sum_{m=0}^k\Delta t_m K_{i_m} \right]^{1/(1-a)}.
\end{align}
If at any $k_c\leq k\sub{max}$ the situation occurs where $N(t_{k_c+1})^{1-a}$ becomes negative, then we should set $N(t_{k+1})=0$ for all $k_c \leq k \leq k\sub{max}$.

All this becomes more complicated than the local theory presented before, because (1) we must determine the time $\Delta t_k$ spent by the particle within each cell, which requires the solution of a nonlinear equation system by itself (at least in the case of general toroidal geometry), (2) the change $N(t_{k+1}) - N(t_k)$ of the density within a given cell $i_k$ depends on the values of the plasma parameters in all cells $i_m$ between $i_k$ and $i_0$, so that the jacobian picks up off-diagonal contributions, and (3) this theory does not appear to be (easily) amenable to generalizations with additional non-linear contributions in $N$ (at least not without solving a nonlinear set of equations for all $\{N(t_k)\}$ to obtain the various $N(t_k)$ in terms of $N(t_n)$).

\end{document}
